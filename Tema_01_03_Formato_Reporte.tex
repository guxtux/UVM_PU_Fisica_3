\documentclass[14pt]{beamer}
\usepackage{./Estilos/BeamerUVM}
\usepackage{./Estilos/ColoresLatex}
\usetheme{Madrid}
\usecolortheme{default}
%\useoutertheme{default}
\setbeamercovered{invisible}
% or whatever (possibly just delete it)
\setbeamertemplate{section in toc}[sections numbered]
\setbeamertemplate{subsection in toc}[subsections numbered]
\setbeamertemplate{subsection in toc}{\leavevmode\leftskip=3.2em\rlap{\hskip-2em\inserttocsectionnumber.\inserttocsubsectionnumber}\inserttocsubsection\par}
% \setbeamercolor{section in toc}{fg=blue}
% \setbeamercolor{subsection in toc}{fg=blue}
% \setbeamercolor{frametitle}{fg=blue}
\setbeamertemplate{caption}[numbered]

\setbeamertemplate{footline}
\beamertemplatenavigationsymbolsempty
\setbeamertemplate{headline}{}


\makeatletter
% \setbeamercolor{section in foot}{bg=gray!30, fg=black!90!orange}
% \setbeamercolor{subsection in foot}{bg=blue!30}
% \setbeamercolor{date in foot}{bg=black}
\setbeamertemplate{footline}
{
  \leavevmode%
  \hbox{%
  \begin{beamercolorbox}[wd=.333333\paperwidth,ht=2.25ex,dp=1ex,center]{section in foot}%
    \usebeamerfont{section in foot} {\insertsection}
  \end{beamercolorbox}%
  \begin{beamercolorbox}[wd=.333333\paperwidth,ht=2.25ex,dp=1ex,center]{subsection in foot}%
    \usebeamerfont{subsection in foot}  \insertsubsection
  \end{beamercolorbox}%
  \begin{beamercolorbox}[wd=.333333\paperwidth,ht=2.25ex,dp=1ex,right]{date in head/foot}%
    \usebeamerfont{date in head/foot} \insertshortdate{} \hspace*{2em}
    \insertframenumber{} / \inserttotalframenumber \hspace*{2ex} 
  \end{beamercolorbox}}%
  \vskip0pt%
}
\makeatother

\makeatletter
\patchcmd{\beamer@sectionintoc}{\vskip1.5em}{\vskip0.8em}{}{}
\makeatother

% \usefonttheme{serif}
\usepackage[clock]{ifsym}

\sisetup{per-mode=symbol}
\resetcounteronoverlays{saveenumi}

\title{\Large{Reporte de Laboratorio} \\ \normalsize{Física III}}
\date{11 de septiembre de 2023}

\renewcommand\cellset{\renewcommand\arraystretch{0.7}%
\setlength\extrarowheight{0pt}}

\begin{document}
\maketitle

\section*{Contenido}
\frame{\frametitle{Contenido} \tableofcontents[currentsection, hideallsubsections]}

\section{El reporte de Laboratorio}
\frame{\tableofcontents[currentsection, hideothersubsections]}
\subsection{Evidencia de trabajo}

\begin{frame}
\frametitle{La finalidad del reporte}
Un reporte de Laboratorio es un texto en donde cada alumno menciona la experiencia que le ha dejado la realización de un experimento.
\end{frame}
\begin{frame}
\frametitle{La finalidad del reporte}
El alumno desarrollará la habilidad no solo de anotar lo que hizo, cómo lo hizo, \pause sino también corroborar el objetivo así como la hipótesis de la cual parte para estudiar el fenómeno.
\end{frame}
\begin{frame}
\frametitle{La finalidad del reporte}
Debiendo interpretar sus resultados, \pause es decir, extender su trabajo práctico y relacionar sus propuestas con lo que desarrolló.
\end{frame}
\begin{frame}
\frametitle{Reporte como entrenamiento}
El alumno contará con la evidencia de sus primeros reportes y podrá comparar con las últimas entregas, reconociendo una mejora, no solo en la redacción de su evidencia, sino con el manejo del formato.
\end{frame}
\begin{frame}
\frametitle{Trabajo en equipo - Reporte individual}
En cada práctica de Laboratorio se tendrá trabajo en equipo, \pause mientras que la elaboración y entrega del reporte será INDIVIDUAL.
\end{frame}

\section{Elementos del reporte}
\frame[allowframebreaks]]\tableofcontents[currentsection, hideothersubsections]}
\subsection{Formato a seguir}

\begin{frame}
\frametitle{Formato del reporte}
El formato que seguiremos durante el ciclo escolar será el mismo, \pause debiendo de seguirlo en todo momento, con la finalidad adicional de tener una estructura homogénea.
\end{frame}
\begin{frame}
\frametitle{Formato del reporte}
Cada apartado del reporte se evalúa de manera integral, \pause por lo que se espera que la entrega del mismo sea completa y en todo momento, cubriendo cada apartado.
\end{frame}

\subsection{Carátula}

\begin{frame}
\frametitle{Identificación de la actividad}
El primer apartado corresponde a la \textocolor{byzantium}{carátula} del reporte.
\end{frame}

\begin{frame}
\frametitle{Datos de la portada}
\setbeamercolor{item projected}{bg=cordovan,fg=white}
\setbeamertemplate{enumerate items}{%
\usebeamercolor[bg]{item projected}%
\raisebox{1.5pt}{\colorbox{bg}{\color{fg}\footnotesize\insertenumlabel}}%
}
\begin{enumerate}[<+->]
\item Grupo.
\item Sección.
\item Horario.
\seti
\end{enumerate}
\end{frame}
\begin{frame}
\frametitle{Datos de la portada}
\setbeamercolor{item projected}{bg=cordovan,fg=white}
\setbeamertemplate{enumerate items}{%
\usebeamercolor[bg]{item projected}%
\raisebox{1.5pt}{\colorbox{bg}{\color{fg}\footnotesize\insertenumlabel}}%
}
\begin{enumerate}[<+->]
\conti
\item Sesión.
\item Unidad.
\item Temática.
\seti
\end{enumerate}
\end{frame}
\begin{frame}
\frametitle{Datos de la portada}
\setbeamercolor{item projected}{bg=cordovan,fg=white}
\setbeamertemplate{enumerate items}{%
\usebeamercolor[bg]{item projected}%
\raisebox{1.5pt}{\colorbox{bg}{\color{fg}\footnotesize\insertenumlabel}}%
}
\begin{enumerate}[<+->]
\conti
\item Nombre de la sesión.
\item Número de sesiones a realizar.
\item Nombre completo del alumno.
\seti
\end{enumerate}
\end{frame}

\subsection{Planteamiento}

\begin{frame}
\frametitle{Planteamiento del problema}
Se elaborarán preguntas o problemas sobre un fenómeno físico, \pause que se resolverán de manera concreta, \pause determinada \pause y sin ambigüedades.
\end{frame}
\begin{frame}
\frametitle{Planteamiento del problema}
En la primera sesión de la práctica se llevará a cabo esta actividad.
\end{frame}

\subsection{Marco teórico}

\begin{frame}
\frametitle{El marco teórico}
A partir de una breve investigación bibliográfica, se redactara un texto que ponga en contexto el problema a resolver, apoyándose con la física.
\end{frame}
\begin{frame}
\frametitle{El marco teórico}
Este texto servirá para contrastar los resultados de la práctica, \pause revisando la congruencia con el mismo.
\end{frame}
\begin{frame}
\frametitle{El marco teórico}
Las referencias se indicarán en formato APA, dejando una guía general para su elaboración.
\end{frame}
\begin{frame}
\frametitle{Sobres los materiales de consulta}
Los materiales de consulta deben de ser libros y revistas.
\\
\bigskip
\pause
Consultar páginas de internet puede ser una primera aproximación al problema, pero no se considerarán como fuentes de consulta.
\end{frame}

\subsection{Objetivos}

\begin{frame}
\frametitle{Objetivo general}
Es la meta a realizar durante la actividad práctica.
\\
\bigskip
\pause
Este objetivo se detallará en la primera sesión.
\end{frame}
\begin{frame}
\frametitle{Objetivos específicos}
Son las actividades puntuales que favorecerán el alcanzar el objetivo general.
\end{frame}


\subsection{Hipótesis}


\begin{frame}
\frametitle{La hipótesis}
La hipótesis es una propuesta del resultado de la práctica.
\\
\bigskip
\pause
Se deberá de corroborar si es correcta o no.
\end{frame}

\subsection{Plan de investigación}

\begin{frame}
\frametitle{Tipo de investigación}
Para el curso de Física III, el tipo de investigación será \textocolor{red}{Experimental.}
\end{frame}
\begin{frame}
\frametitle{Programa de actividades}
\setbeamercolor{item projected}{bg=bananayellow,fg=black}
\setbeamertemplate{enumerate items}{%
\usebeamercolor[bg]{item projected}%
\raisebox{1.5pt}{\colorbox{bg}{\color{fg}\footnotesize\insertenumlabel}}%
}
\begin{enumerate}[<+->]
\item Sesión 1. Encuadre de la práctica.
\item Sesión 2. Montaje experimental.
\item Sesión 3. Discusión de resultados.
\end{enumerate}
\end{frame}

\subsection{Material y equipo}

\begin{frame}
\frametitle{Listando el material}
Para cada práctica se identificará de manera previa el material y equipo necesario para el montaje experimental.
\end{frame}
\begin{frame}
\frametitle{Listando el material}
En algunas actividades, el equipo de alumnos deberá de proporcionar material para realizar la práctica.
\end{frame}
\begin{frame}
\frametitle{Listando el material}
Para cada práctica se identificará de manera previa el material y equipo necesario para el montaje experimental.
\end{frame}

\subsection{Procedimiento}

\begin{frame}
\frametitle{El montaje experimental}
El alumno llegará a la sesión de montaje conociendo las actividades a realizar, \pause así como la información que recabará en el tiempo de la práctica.
\end{frame}
\begin{frame}
\frametitle{Recabando los datos}
El alumno deberá de haber preparado los elementos para registrar los datos experimentales: tablas, listas, etc.
\end{frame}
\begin{frame}
\frametitle{Análisis preliminar}
Luego de concluida la actividad de montaje, el alumno deberá de realizar un análisis preliminar de los resultados que registró durante la sesión.
\end{frame}

\subsection{Análisis y discusión}

\begin{frame}
\frametitle{Tercera sesión}
En la última sesión de cada práctica se discutirán los resultados de las prácticas.

\end{frame}
\begin{frame}
\frametitle{En la tercera sesión}
Se eligirá un equipo al azar y presentará sus datos, interpretación y determinación del alcance de objetivos, así como de la hipótesis planteada.
\end{frame}
\begin{frame}
\frametitle{En la tercera sesión}
Todos los equipos presentarán sus resultados, es decir, habrá rotación y participación constante.
\end{frame}

\subsection{Conclusiones}

\begin{frame}
\frametitle{Elaboración de las Conclusiones}
Con la discusión de resultados, cada integrante deberá de redactar las conclusiones de la práctica.
\end{frame}

\subsection{Entrega del reporte}

\begin{frame}
\frametitle{Entrega del reporte}
Considerando que en cada sesión se va avanzando en la realización del contenido para el reporte, \pause el alumno entregará por asignación en Teams su reporte individual.
\end{frame}
% \begin{frame}
% \frametitle{Calendario de entregas.}
% \setbeamercolor{item projected}{bg=auburn,fg=white}
% \setbeamertemplate{enumerate items}{%
% \usebeamercolor[bg]{item projected}%
% \raisebox{1.5pt}{\colorbox{bg}{\color{fg}\footnotesize\insertenumlabel}}%
% }
% \begin{enumerate}[<+->]
% \item Grupo 43B. Lunes 9 - 10 am. Sábados 5 pm.
% \item Grupo 43C. Lunes 10 - 11 am. Sábados 5 pm.
% \end{enumerate}
% \end{frame}
\begin{frame}
\frametitle{Calendario de entregas.}
\setbeamercolor{item projected}{bg=auburn,fg=white}
\setbeamertemplate{enumerate items}{%
\usebeamercolor[bg]{item projected}%
\raisebox{1.5pt}{\colorbox{bg}{\color{fg}\footnotesize\insertenumlabel}}%
}
\begin{enumerate}[<+->]
\item Grupo 47B. Miércoles 7 - 8 am. Domingos 5 pm.
\item Grupo 47C. Miércoles 8 - 9 am. Domingos 5 pm.
\end{enumerate}
\end{frame}


\end{document}