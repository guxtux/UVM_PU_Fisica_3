\documentclass[14pt]{extarticle}
\usepackage[utf8]{inputenc}
\usepackage[T1]{fontenc}
\usepackage[spanish,es-lcroman]{babel}
\usepackage{amsmath}
\usepackage{amsthm}
\usepackage{physics}
\usepackage{tikz}
\usepackage{float}
\usepackage[autostyle,spanish=mexican]{csquotes}
\usepackage[per-mode=symbol]{siunitx}
\usepackage{gensymb}
\usepackage{multicol}
\usepackage{enumitem}
\usepackage[left=2.00cm, right=2.00cm, top=2.00cm, 
     bottom=2.00cm]{geometry}

%\renewcommand{\questionlabel}{\thequestion)}
\decimalpoint
\sisetup{bracket-numbers = false}

\renewcommand*{\theenumi}{\thesection.\arabic{enumi}}
\renewcommand*{\theenumii}{\theenumi.\arabic{enumii}}

\title{\vspace*{-2cm} Ejercicios de repaso \\  Evaluación Continua - Física III\vspace{-5ex}}
\date{}

\begin{document}
\maketitle

\textbf{Indicaciones:} Al inicio de cada hoja que utilices para tu solución, deberás de anotar tu nombre completo.
\par
Resuelve de manera detallada cada uno de los siguientes ejercicios, en donde deberás de indicar el paso (o pasos necesarios) para llegar al resultado.
\par
Esta actividad otorgará hasta \textbf{8 puntos}. Si se reporta el resultado directo, sin presentar el desarrollo del ejercicio, éste no aporta puntaje.

\section{Segunda ley de Newton.}

\begin{enumerate}
\item Determina la fuerza que se necesita aplicar a un camión de \SI{2800}{\kilo\gram} para que éste se acelere \SI{6.5}{\meter\per\square\second}
\item La fuerza resultante de las fuerzas que actúan sobre un cuerpo de \SI{70}{\kilo\gram}, es de \SI{123}{\newton}. ¿Cuál es el valor de la aceleración que posee este cuerpo?
\item ¿Cuál es la masa de un cuerpo si al aplicarle una fuerza de \SI{750}{\newton} adquiere una aceleración de \SI{9.3}{\meter\per\square\second}?
\item Qué fuerza ejerce el motor de un automóvil de \SI{1300}{\kilo\gram} para detenerlo completamente después de \SI{7}{\second} si iba a una velocidad de \SI{65}{\kilo\meter}?
\end{enumerate}

\section{Ley de gravitación universal.}

\begin{enumerate}
\item Determina la fuerza gravitacional entre la Tierra y la Luna, sabiendo que sus masas son \SI{5.98d24}{\kilo\gram} y \SI{7.35d22}{\kilo\gram}, respectivamente, y están separadas una distancia de \SI{3.8d8}{\meter}.
\item Calcula la distancia promedio entre la Tierra y el Sol, cuyas masas son \break \hfill \SI{5.98d24}{\kilo\gram} y \SI{2d30}{\kilo\gram}, respectivamente, si entre ellos existe una fuerza gravitacional de \SI{3.6d22}{\newton}.
\item Un cuerpo de \SI{60}{\kilo\gram} se encuentra a una distancia de \SI{3.5}{\meter} de otro cuerpo, de manera que entre ellos se produce una fuerza de \SI{6.5d-7}{\newton}. Calcula la masa del otro cuerpo.
\item Calcula la distancia a la que se deben de colocar dos masas iguales de \SI{1}{\kilo\gram} para que se atraigan con una fuerza de \SI{1}{\newton}.
\end{enumerate}

\end{document}