\documentclass[14pt]{extarticle}
\usepackage[utf8]{inputenc}
\usepackage[T1]{fontenc}
\usepackage[spanish,es-lcroman]{babel}
\usepackage{amsmath}
\usepackage{amsthm}
\usepackage{physics}
\usepackage{tikz}
\usepackage{float}
\usepackage[autostyle,spanish=mexican]{csquotes}
\usepackage[per-mode=symbol]{siunitx}
\usepackage{gensymb}
\usepackage{multicol}
\usepackage{enumitem}
\usepackage[left=2.00cm, right=2.00cm, top=2.00cm, 
     bottom=2.00cm]{geometry}

%\renewcommand{\questionlabel}{\thequestion)}
\decimalpoint
\sisetup{bracket-numbers = false}

\renewcommand*{\theenumi}{\thesection.\arabic{enumi}}
\renewcommand*{\theenumii}{\theenumi.\arabic{enumii}}

\title{\vspace*{-2cm} Actividad Grupo 47 \\  Evaluación Continua - Física III\vspace{-5ex}}
\date{16 de febrero de 2024}

\begin{document}
\maketitle

\textbf{Nombre:} \rule{8cm}{0.1mm}

\vspace*{0.75cm}
\textbf{Importante: } La hoja deberás de devolverla antes de que concluya la clase, ya que será la evidencia de tu asistencia a la clase de Física III.

\vspace*{0.5cm}
Esta actividad otorgará hasta \textbf{4 puntos}, un punto por cada ejercicio. Deberás de entregar el desarrollo completo de cada ejercicio, si se reporta el resultado directo, sin presentar el desarrollo, sin manejo de unidades en cada paso, no se contará como ejercicio resuelto.

\vspace*{0.75cm}
Podrás apoyarte con tus notas.

\section{Ejercicios.}

Considera que este tipo de ejercicios son los que se incluirán en el siguiente examen parcial.

\begin{enumerate}
\item ¿Qué fuerza se requiere para llevar un automóvil de 1500 kg al reposo, que lleva una velocidad de \SI{100}{\kilo\meter\per\hour} en una distancia de \SI{55}{\meter}?
\item Una fuerza neta de \SI{265}{\newton} acelera a una persona en bicicleta a \SI{2.30}{\meter\per\square\second}. ¿Cuál es la masa de la persona junto con la bicicleta?
\item Superman debe detener un tren que viaja a \SI{120}{\kilo\meter\per\hour} en \SI{150}{\meter} para evitar que choque contra un automóvil parado sobre las vías. Si la masa del tren es de \SI{3.6d5}{\kilo\gram}. ¿Cuánta fuerza debe ejercer el superhéroe?
% \item Una pelota de béisbol de \SI{0.140}{\kilo\gram} que viaja a \SI{35.0}{\meter\per\second} golpea el guante del catcher, que al llevarla al reposo, se mueve hacia atrás \SI{11.0}{\centi\meter}. ¿Cuál fue la fuerza promedio aplicada por la pelota al guante?
\item Dos trasatlánticos, cada uno con \num{40000} toneladas de masa, se mueven en rutas paralelas separadas \SI{100}{\meter}. ¿Cuál es la magnitud de la aceleración de uno de los trasatlánticos hacia el otro debido a su atracción gravitacional mutua?
\end{enumerate}
\end{document}