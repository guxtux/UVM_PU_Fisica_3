\documentclass[14pt]{beamer}
\usepackage{./Estilos/BeamerUVM}
\usepackage{./Estilos/ColoresLatex}
\input{Preambulos/preambulo_Beamer_Berlin_beaver}
% \usefonttheme{serif}
\usepackage[clock]{ifsym}
\DeclareSIUnit\erg{erg}
\DeclareSIUnit[number-unit-product = {\,}]\cal{cal}

\sisetup{per-mode=symbol}
\resetcounteronoverlays{saveenumi}

% Macro para agregar el logo de UVM en cada slide de la presentación

\addtobeamertemplate{frametitle}{}{%
\begin{tikzpicture}[remember picture,overlay]
\coordinate (logo) at ([xshift=-1.5cm,yshift=-0.8cm]current page.north east);
% \fill[devryblue] (logo) circle (.9cm);
% \clip (logo) circle (.75cm);
\node at (logo) {\includegraphics[width=2.1cm]{Imagenes/logo_UVM.png}};
\end{tikzpicture}}


\title{\Large{Transformaciones de Energía} \\ \normalsize{Física III}}
\date{}

\begin{document}
\maketitle

\section*{Contenido}
\frame[allowframebreaks]{\frametitle{Contenido} \tableofcontents[currentsection, hideallsubsections]}

\section{Transformaciones de energía}
\frame{\tableofcontents[currentsection, hideothersubsections]}
\subsection{Energía constante}

\begin{frame}
\frametitle{¿Qué son las trasnformaciones?}
Las \textocolor{carnelian}{transformaciones de energía} se refieren a los cambios en la forma de la energía en un sistema.
\end{frame}
\begin{frame}
\frametitle{Conservación de energía}
La energía no se crea ni se destruye, pero puede cambiar de una forma a otra.
\end{frame}




\end{document}