\documentclass[14pt]{beamer}
\usepackage{./Estilos/BeamerUVM}
\usepackage{./Estilos/ColoresLatex}
\usetheme{Madrid}
\usecolortheme{default}
%\useoutertheme{default}
\setbeamercovered{invisible}
% or whatever (possibly just delete it)
\setbeamertemplate{section in toc}[sections numbered]
\setbeamertemplate{subsection in toc}[subsections numbered]
\setbeamertemplate{subsection in toc}{\leavevmode\leftskip=3.2em\rlap{\hskip-2em\inserttocsectionnumber.\inserttocsubsectionnumber}\inserttocsubsection\par}
% \setbeamercolor{section in toc}{fg=blue}
% \setbeamercolor{subsection in toc}{fg=blue}
% \setbeamercolor{frametitle}{fg=blue}
\setbeamertemplate{caption}[numbered]

\setbeamertemplate{footline}
\beamertemplatenavigationsymbolsempty
\setbeamertemplate{headline}{}


\makeatletter
% \setbeamercolor{section in foot}{bg=gray!30, fg=black!90!orange}
% \setbeamercolor{subsection in foot}{bg=blue!30}
% \setbeamercolor{date in foot}{bg=black}
\setbeamertemplate{footline}
{
  \leavevmode%
  \hbox{%
  \begin{beamercolorbox}[wd=.333333\paperwidth,ht=2.25ex,dp=1ex,center]{section in foot}%
    \usebeamerfont{section in foot} {\insertsection}
  \end{beamercolorbox}%
  \begin{beamercolorbox}[wd=.333333\paperwidth,ht=2.25ex,dp=1ex,center]{subsection in foot}%
    \usebeamerfont{subsection in foot}  \insertsubsection
  \end{beamercolorbox}%
  \begin{beamercolorbox}[wd=.333333\paperwidth,ht=2.25ex,dp=1ex,right]{date in head/foot}%
    \usebeamerfont{date in head/foot} \insertshortdate{} \hspace*{2em}
    \insertframenumber{} / \inserttotalframenumber \hspace*{2ex} 
  \end{beamercolorbox}}%
  \vskip0pt%
}
\makeatother

\makeatletter
\patchcmd{\beamer@sectionintoc}{\vskip1.5em}{\vskip0.8em}{}{}
\makeatother

% \usefonttheme{serif}
\usepackage[clock]{ifsym}

\sisetup{per-mode=symbol}
\resetcounteronoverlays{saveenumi}

\title{\Large{Revisión inicial} \\ \normalsize{Física III}}
\date{}

\renewcommand\cellset{\renewcommand\arraystretch{0.7}%
\setlength\extrarowheight{0pt}}

\begin{document}
\maketitle

\section*{Contenido}
\frame[allowframebreaks]{\frametitle{Contenido} \tableofcontents[currentsection, hideallsubsections]}

\section{¿Qué es la física?}
\frame{\tableofcontents[currentsection, hideothersubsections]}
\subsection{Lo que entendemos por física}

\begin{frame}
\frametitle{¿Qué es la física?}
La Física es una ciencia basada en las \textocolor{cobalt}{observaciones} y \textocolor{cadmiumgreen}{medidas} de los fenómenos físicos.
\end{frame}
\begin{frame}
\frametitle{Naturaleza de la física}
La física es una ciencia experimental.
\\
\bigskip
\pause
Los físicos observan los fenómenos naturales e intentan encontrar los patrones y principios que los describen.
\end{frame}
\begin{frame}
\frametitle{Naturaleza de la física}
Tales patrones se denominan \textocolor{byzantine}{teorías físicas} \pause o, si están muy bien establecidos y se usan ampliamente, \textocolor{cordovan}{leyes o principios físicos}.
\end{frame}

\section{Conceptos importantes}
\frame{\tableofcontents[currentsection, hideothersubsections]}
\subsection{Medición}

\begin{frame}
\frametitle{¿Qué es medir?}
Es \textocolor{lava}{comparar} una magnitud con otra de la misma especie, llamada patrón.
\end{frame}
\begin{frame}
\frametitle{¿Qué es una magnitud?}
Es una cantidad medible de un sistema físico a la que se le pueden asignar distintos valores como resultado de una medición o una relación de medidas.
\end{frame}
\begin{frame}
\frametitle{¿Qué es una unidad?}
Es una cantidad de una determinada magnitud física, definida y adoptada por convención o por ley.
\\
\bigskip
\pause
Cualquier valor de una cantidad física puede expresarse como un múltiplo de la unidad de medida.
\end{frame}

\subsection{Unidades fundamentales}

\begin{frame}
\frametitle{Las unidades fundamentales}
Las \textocolor{airforceblue}{unidades fundamentales} del \textocolor{awesome}{Sistema Internacional de Unidades} (SI), son magnitudes físicas básicas que pueden medirse y son independientes de todas las demás.
\end{frame}
\begin{frame}
\frametitle{Las unidades fundamentales}
\begin{table}
\renewcommand{\arraystretch}{1.1}
\centering
\begin{tabular}{l | c | c}
Magnitud & Unidad & Símbolo \\ \hline 
Longitud & metro & \unit{m} \\ \hline \pause
Masa & kilogramo & \unit{\kilo\gram} \\ \hline
Tiempo & segundo & \unit{\second} \\ \hline
Temperatura & Kelvin & \unit{\kelvin} \\ \hline
\end{tabular}
\end{table}
\end{frame}
\begin{frame}
\frametitle{Las unidades fundamentales}
\begin{table}
\renewcommand{\arraystretch}{1.1}
\centering
\begin{tabular}{l | c | c}
Magnitud & Unidad & Símbolo \\ \hline 
Intensidad eléctrica & Ampere & \unit{\ampere} \\ \hline
Intensidad luminosa & candela & cd \\ \hline
Cantidad de sustancia & mol & \unit{mol}\\ \hline
\end{tabular}
\end{table}
\end{frame}
\begin{frame}
\frametitle{Unidades derivadas}
Son las unidades que provienen de una combinación de las unidades fundamentales.
\end{frame}
\begin{frame}
\frametitle{Unidades derivadas}
\begin{table}
\renewcommand{\arraystretch}{1.1}
\centering
\begin{tabular}{l | c | c | c}
Magnitud & Unidad & Símbolo & Equivalencia\\ \hline 
Área & \makecell{metro \\ cuadrado} & \unit{\square\meter} &  \\ \hline
Volumen & \makecell{metro \\ cúbico} & \unit{\cubic\meter} &  \\ \hline
Velocidad & \makecell{metro \\ por segundo} & \unit{\meter\per\second} & \\ \hline
\end{tabular}
\end{table}
\end{frame}
\begin{frame}
\frametitle{Unidades derivadas}
\begin{table}
\renewcommand{\arraystretch}{1.4}
\centering
\begin{tabular}{l | c | c | c}
Magnitud & Unidad & Símbolo & Equivalencia\\ \hline 
Fuerza & newton & \unit{\newton} & $\displaystyle \unit[per-mode=fraction]{\kilo\gram\meter\per\square\second} $\\ \hline \pause
Presión & Pascal & \unit{\pascal} & $\displaystyle \unit[per-mode=fraction]{\newton\per\square\meter}$ \\ \hline \pause
Trabajo & Joule & \unit{\joule} & $\displaystyle \unit[per-mode=fraction]{\newton\meter}$ \\ \hline
\end{tabular}
\end{table}
\end{frame}

\section{Tipos de cantidades}
\frame{\tableofcontents[currentsection, hideothersubsections]}
\subsection{Cantidad escalar}

\begin{frame}
\frametitle{Cantidad escalar}
Es la que queda definida con solo indicar su cantidad en \textocolor{amethyst}{número y unidad de medida}.
\\
\bigskip
\pause
Ejemplos: \SI{5}{\kilo\gram}, \pause \SI{20}{\degreeCelsius}, \pause \SI{250}{\square\meter}, \pause \SI{40}{\milli\gram}
\end{frame}
\begin{frame}
\frametitle{Cantidad vectorial}
Es la que además de definir cantidad en número y unidad de medida, se requiere indicar la dirección y sentido en que actúan.
\\
\bigskip
\pause
Se representan de manera gráfica por \textocolor{ao(english)}{vectores}, \pause los cuales deben tener: \textocolor{blue}{magnitud}, \pause \textocolor{red}{dirección} \pause y \textocolor{cadmiumorange}{sentido}.
\end{frame}

\section{Manejando unidades}
\frame{\tableofcontents[currentsection, hideothersubsections]}
\subsection{Los prefijos de unidades}

\begin{frame}
\frametitle{Los prefijos de las unidades}
Una vez definidas las unidades fundamentales, es fácil introducir unidades más grande y más pequeñas para las mismas cantidades físicas.
\end{frame}
\begin{frame}
\frametitle{Los prefijos de las unidades}
En el sistema métrico, estas otras unidades siempre se relacionan con las fundamentales por múltiplos y submúltiplos de $10$.
\end{frame}
\begin{frame}
\frametitle{Múltiplos de $10$}
\begin{table}
\renewcommand{\arraystretch}{1}
\centering
\begin{tabular}{c | c | c}
Potencia & Prefijo & Abreviatura \\ \hline
\num{d3} & kilo & k \\ \hline
\num{d6} & mega & M \\ \hline
\num{d9} & giga & G \\ \hline
\num{d12} & tera & T \\ \hline
\num{d15} & peta & P \\ \hline
\num{d18} & exa & E \\ \hline
\end{tabular}
\end{table}
\end{frame}
\begin{frame}
\frametitle{Submúltiplos de $10$}
\begin{table}
\renewcommand{\arraystretch}{1}
\centering
\begin{tabular}{c | c | c}
Potencia & Prefijo & Abreviatura \\ \hline
\num{d-1} & deca & d \\ \hline
\num{d-2} & centi & c \\ \hline
\num{d-3} & mili & m \\ \hline
\num{d-6} & micro & $\mu$ \\ \hline
\num{d-9} & nano & n \\ \hline
\num{d-12} & pico & p \\ \hline
\end{tabular}
\end{table}
\end{frame}

\section{Conversión de unidades}
\frame{\tableofcontents[currentsection, hideothersubsections]}
\subsection{Procedimiento a realizar}

\begin{frame}
\frametitle{¿Para qué hacer conversión de unidades?}
En ocasiones se va a requerir expresar una cantidad en términos de otras unidades, ya sea por conveniencia o por que se requiere mantener un manejo homogéneo.
\end{frame}
\begin{frame}
\frametitle{¿Para qué hacer conversión de unidades?}
La regla básica es muy sencilla: \pause utilizar ya sea un \textocolor{blue}{factor de conversión}, o los factores de conversión necesarios.
\end{frame}
\begin{frame}
\frametitle{El factor de conversión}
Estos factores de conversión \enquote{ajustan} las unidades, dejando entonces un problema de tipo aritmético, \pause es decir, donde tenemos que hacer multiplicaciones o divisiones.
\end{frame}
\begin{frame}
\frametitle{El factor de conversión}
El factor de conversión es un \enquote{uno} que manejamos de manera conveniente.
\end{frame}
\begin{frame}
\frametitle{Ejemplo 1}
Consideremos el siguiente problema:
\\
\bigskip
\pause
Convertir 100 km a metros.
\end{frame}
\begin{frame}
\frametitle{El facto de conversión}
Siempre será necesario contar de manera previa con el(los) factor(es) de conversión:
\pause
\begin{align*}
\SI{1}{\kilo\meter} = \SI{1000}{\meter}
\end{align*}
\end{frame}
\begin{frame}
\frametitle{EL procedimiento}
Procedemos a escribir el valor inicial que se multiplicará por el factor de conversión:
\pause
\begin{eqnarray*}
\begin{aligned}
\SI{100}{\kilo\meter} \pause \left( \dfrac{\SI{1000}{\meter}}{\SI{1}{\kilo\meter}} \right) = \pause \SI{100000}{\meter}
\end{aligned}
\end{eqnarray*}
\end{frame}
\begin{frame}
\frametitle{Segundo ejercicio}
Ahora consideremos el siguiente problema:
\\
\bigskip
\pause
Convertir \SI{250}{\kilo\meter\per\hour} a m/s
\end{frame}
\begin{frame}
\frametitle{El procedimiento}
Los factores de conversión son:
\pause
\begin{eqnarray*}
\begin{aligned}
\SI{1}{\kilo\meter} &= \SI{1000}{\meter} \\[0.5em] \pause
\SI{1}{\hour} &= \SI{3600}{\second}
\end{aligned}
\end{eqnarray*}
\end{frame}
\begin{frame}
\frametitle{Realizando la conversión}
\begin{eqnarray*}
\begin{aligned}
\SI[per-mode=fraction]{250}{\kilo\meter\per\hour} \pause &\left( \dfrac{\SI{1000}{\meter}}{\SI{1}{\kilo\meter}} \right) \pause \left( \dfrac{\SI{1}{\hour}}{\SI{3600}{\second}} \right) = \pause \dfrac{\SI{250000}{\meter}}{\SI{3600}{\second}} = \\[0.5em] \pause
&= \SI[per-mode=fraction]{69.44}{\meter\per\second}
\end{aligned}
\end{eqnarray*}
\end{frame}
\begin{frame}
\frametitle{Otro ejercicio - Volumen}
Convertir $7$ galones a \unit{\cubic\centi\meter}.
\\
\bigskip
Los factores de conversión son:
\pause
\begin{align*}
1 \, \text{galón} &= \SI{3.785}{\liter} \\[0.5em]
\SI{1}{\liter} &= \SI{1000}{\cubic\centi\meter}
\end{align*}
\end{frame}
\begin{frame}
\frametitle{Realizando la conversión}
\begin{eqnarray*}
\begin{aligned}
7 \, \text{ga} \pause \left( \dfrac{\SI{3.785}{\liter}}{1 \, \text{ga}} \right) \pause \left( \dfrac{\SI{1000}{\cubic\centi\meter}}{\SI{1}{\liter}} \right) = \pause \SI{26945}{\cubic\centi\meter}
\end{aligned}
\end{eqnarray*}
\end{frame}
\begin{frame}
\frametitle{Siguiente ejercicio}
Convertir \SI{96500}{\cubic\centi\meter\per\minute} a  galones/s
\end{frame}
\begin{frame}
\frametitle{Factores de conversión necesarios}
Listamos los factores de conversión:
\pause
\begin{eqnarray*}
\begin{aligned}
1 \, \text{ga} &= \SI{3.785}{\liter} \\[0.5em] \pause
\SI{1}{\liter} &= \SI{1000}{\cubic\centi\meter} \\[0.5em] \pause
\SI{1}{\minute} &= \SI{60}{\second}
\end{aligned}
\end{eqnarray*}
\end{frame}
\begin{frame}
\frametitle{Realizando la conversión}
\begin{eqnarray*}
\begin{aligned}
&\SI[per-mode=fraction]{96500}{\cubic\centi\meter\per\minute} \pause  \left( \dfrac{\SI{1}{\liter}}{\SI{1000}{\cubic\centi\meter}} \right) \pause \left( \dfrac{1 \, \text{ga}}{\SI{3.785}{\liter}} \right) \pause \left(  \dfrac{\SI{1}{\minute}}{\SI{60}{\second}} \right) = \\[1em] \pause
&= \dfrac{96500 \, \text{ga}}{\SI{227100}{\second}} = \\[1em] \pause
&= 0.424 \dfrac{\text{ga}}{\unit{\second}}
\end{aligned}
\end{eqnarray*}
\end{frame}
\begin{frame}
\frametitle{Otro ejercicio}
Convertir \SI{10}{\cubic\meter} a pies${}^{3}$
\\
\bigskip
\pause
El factor de conversión es:
\begin{align*}
1 \, \text{pie} = \SI{0.3048}{\meter}
\end{align*}
\end{frame}
\begin{frame}
\frametitle{Procedimiento para la solución}
\begin{eqnarray*}
\begin{aligned}
1 \, \text{pie} &= \SI{0.3048}{\meter} \\[1em] \pause
(1 \, \text{pie})^{3} &= (\SI{0.3048}{\meter})^{3} \pause \hspace{0.2cm} \Rightarrow \hspace{0.2cm} 1 \, \text{pie}^{3} = \SI{0.02831}{\cubic\meter} \\[1em] \pause
\SI{10}{\cubic\meter} \pause &\left( \dfrac{1 \, \text{pie}}{\SI{0.02831}{\cubic\meter}} \right) = \pause 353.23 \, \text{pie}^{3}
\end{aligned}
\end{eqnarray*}
\end{frame}

\begin{frame}
\frametitle{Trabajo Evaluación Continua}
Tendrás que elaborar una \textocolor{byzantine}{línea de tiempo} sobre la física, que deberá de abarcar desde el tiempo de los griegos hasta el siglo XX.
\end{frame}
\begin{frame}
\frametitle{Puntos por la Actividad}
La actividad otorgará hasta $6$ puntos de Evaluación Continua, siempre y cuando se cumpla el nivel de desempeño \textocolor{darkblue}{Excelente}.
\end{frame}
\begin{frame}
\frametitle{Rúbrica de evaluación}
La evaluación del trabajo será con una rúbrica, en donde se mencionarán los elementos a considerar y el nivel de desempeño. 
\\
\bigskip
\pause
Recuerda que es una guía muy clara en cuanto a lo que se espera que resuelvas.
\end{frame}
\begin{frame}
\frametitle{Entrega de la Actividad}
%El plazo para enviar por Teams la actividad es el jueves 21 de septiembre a las 8 pm.
El plazo para enviar por Teams la actividad es el martes 19 de septiembre a las 8 pm.
\\
\bigskip
\pause
No se recibirán entregas extemporáneas, por correo o por mensaje directo al Profesor.
\end{frame}

\section{Notación científica}
\frame{\tableofcontents[currentsection, hideothersubsections]}
\subsection{Ejercicios}

\begin{frame}
\frametitle{Actividad a realizar}
Escribe las siguientes cifras en notación científica:
\setbeamercolor{item projected}{bg=cobalt,fg=white}
\setbeamertemplate{enumerate items}{%
\usebeamercolor[bg]{item projected}%
\raisebox{1.5pt}{\colorbox{bg}{\color{fg}\footnotesize\insertenumlabel}}%
}
\begin{enumerate}[<+->]
\item $16200000000000 =$% \num{1.62d13}$
\item $0.00000045 = $ %\num{4.5d-7}$
\item $0.00000000123 = $ % \num{1.23d-9}$
\item $384500000000000 = $ % \num{3.845d14}$
\end{enumerate}
\end{frame}
\begin{frame}
\frametitle{Ejercicio a resolver}
La velocidad de la luz en el vacío es de \SI[per-mode=symbol]{300000}{\kilo\meter\per\second}.
\\
\bigskip
\pause
Escribe en notación científica esta cifra pero expresada en pulgadas por segundo.
\end{frame}
\begin{frame}
\frametitle{Factores de conversión}
Para responder este ejercicio necesitamos los factores de conversión:
\pause
\begin{eqnarray*}
\begin{aligned}
\SI{1}{\kilo\meter} &= \pause \SI{1000}{\meter} = \pause \SI{d3}{\meter} \\[0.5em] \pause
1 \, \text{pulgada} &= \pause \SI{2.54}{\centi\meter} = \pause \SI{2.54d-2}{\meter}
\end{aligned}
\end{eqnarray*}
\end{frame}
\begin{frame}
\frametitle{Expresando la velocidad de la luz}
La velocidad de la luz en notación científica se escribe como:
\pause
\begin{eqnarray*}
\begin{aligned}
\SI[per-mode=symbol]{300000}{\kilo\meter\per\second} = \pause \SI[per-mode=fraction]{3d5}{\kilo\meter\per\second}
\end{aligned}
\end{eqnarray*}
\end{frame}
\begin{frame}
\frametitle{Haciendo las operaciones}
Ya podemos ocupar los factores de conversión:
\pause
\begin{eqnarray*}
\begin{aligned}
\SI[per-mode=fraction]{3d5}{\kilo\meter\per\second} \, \pause &\left( \dfrac{\SI{d3}{\meter}}{\SI{1}{\kilo\meter}} \right) \pause \, \left( \dfrac{1 \, \text{pulgada}}{\SI{2.54d-2}{\meter}} \right) = \\[0.5em] \pause
&= \dfrac{\SI{3d8}{\meter} \, \text{pulgadas}}{\SI{2.54d-2}{\meter}} = \\[0.5em] \pause 
&= \num{1.1811d10} \, \dfrac{\text{pulgadas}}{\unit{\second}}
\end{aligned}
\end{eqnarray*}
\end{frame}

\end{document}