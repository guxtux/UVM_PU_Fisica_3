\documentclass[14pt]{extarticle}
\usepackage[utf8]{inputenc}
\usepackage[T1]{fontenc}
\usepackage[spanish,es-lcroman]{babel}
\usepackage{amsmath}
\usepackage{amsthm}
\usepackage{physics}
\usepackage{siunitx}
\usepackage{tikz}
\usepackage{float}
\usepackage[autostyle,spanish=mexican]{csquotes}
\usepackage[per-mode=symbol]{siunitx}
\usepackage{gensymb}
\usepackage{multicol}
\usepackage{enumitem}
\usepackage[left=2.00cm, right=2.00cm, top=2.00cm, 
     bottom=2.00cm]{geometry}

%\renewcommand{\questionlabel}{\thequestion)}
\decimalpoint
\sisetup{bracket-numbers = false}

\DeclareSIUnit{\nothing}{\relax}

\title{\vspace*{-2cm} Ejercicios de Notación Científica \\  Solución - Física III\vspace{-5ex}}
\date{}

\begin{document}
\maketitle

\begin{enumerate}
\item (\textbf{1 punto}) Realiza las siguientes operaciones con números en notación científica, ocupa tu calculadora científica (la finalidad es que vayas reconociendo su uso), el resultado debe de quedar con un dígito antes del punto decimal y cuatro dígitos después del punto.
\begin{enumerate}
    \item (\num{3.4d23})(\num{1.1d11}) = \num{3.74d34}
    \item (\num{2.45d-12})(\num{2.45d-4}) = \num{6.0025d-16}
    \item (\num{4.56d11})(\num{0.655d-20}) = \num{2.9868d-9}
    \item \num{2.2d5} / \num{0.6d45} = \num{3.66666d-40}
    \item \num{0.68d-8} / \num{1.9d-13} = \num{35789.47} = \num{3.578947d4}
    \item
    \begin{eqnarray*}
    \begin{aligned}
    &\dfrac{(\num{4.3d5})(\num{2.23d2})(\sqrt{\num{1.4d-6}})}{(\num{9d4})^{2}} = \\[0.5em]
    &= \dfrac{(\num{9.589d7})(\sqrt{\num{1.4d-6}})}{(\num{9d4})^{2}} = \\[0.5em]
    &= \dfrac{(\num{9.589d7})(\num{1.832d-3})}{\num{8.1d9}} = \\[0.5em]
    &= \dfrac{\num{1.7567d5}}{\num{8.1d9}} = \\[0.5em]
    &= \num{2.1687d-5}
    \end{aligned}
    \end{eqnarray*}
\end{enumerate}
\item (\textbf{1 punto}) Con notación científica de base $10$, expresa:
\begin{enumerate}
\item Un área de \SI{2}{\square\kilo\meter} en \unit{\square\centi\meter}.

Tomemos en cuenta que no hay un factor de conversión \enquote{directo} para pasar de \unit{\square\kilo\meter} a \unit{\square\centi\meter}, por lo que hay que construirlo:
\begin{align*}
    \SI{1}{\meter} &= \SI{100}{\centi\meter} \\
    \SI{1}{\kilo\meter} &= \SI{100000}{\centi\meter} \\
    (\SI{1}{\kilo\meter})^{2} &= (\SI{100000}{\centi\meter})^{2} = \\
    \SI{1}{\square\kilo\meter} &= \SI{1d10}{\square\centi\meter}    
\end{align*}
Una vez obtenido el factor de conversión, lo ocupamos:
\begin{align*}
    \SI{2}{\square\kilo\meter} \left( \dfrac{\SI{1d10}{\square\centi\meter}}{\SI{1}{\square\kilo\meter}} \right) = \SI{2d10}{\square\centi\meter}
\end{align*}
\item Un volumen de \SI{5}{\cubic\centi\meter} en \unit{\cubic\meter}.

Al encontrar nuevamente un factor que no es directo, tenemos que \enquote{construirlo}:
\begin{align*}
\SI{1}{\meter} &= \SI{100}{\centi\meter} \\
(\SI{1}{\meter})^{3} &= (\SI{100}{\centi\meter})^{3} \\
\SI{1}{\cubic\meter} &= \SI{1d6}{\cubic\centi\meter}
\end{align*}

Ya podemos usar este factor de conversión:
\begin{align*}
\SI{5}{\cubic\centi\meter} \left( \dfrac{\SI{1}{\cubic\meter}}{\SI{1d6}{\cubic\centi\meter}} \right) = \SI{5d-6}{\cubic\meter}
\end{align*}
\item Un volumen de \SI{4}{\liter} en \unit{\cubic\milli\meter}, considera que en \SI{1}{\liter} hay \SI{d3}{\cubic\centi\meter}.

Procedemos como se ha hecho en los dos ejercicios anteriores:
\begin{align*}
    \SI{1}{\centi\meter} &= \SI{10}{\milli\meter} \\
    (\SI{1}{\centi\meter})^{3} &= (\SI{10}{\milli\meter})^{3} \\
    \SI{1}{\cubic\centi\meter} &= \SI{d3}{\cubic\milli\meter}
\end{align*}
Por lo que tenemos ahora dos factores de conversión:
\begin{align*}
\SI{4}{\liter} \left( \dfrac{\SI{d3}{\cubic\centi\meter}}{\SI{1}{\liter}} \right) \left( \dfrac{\SI{d3}{\cubic\milli\meter}}{\SI{1}{\cubic\centi\meter}} \right) = \SI{4d6}{\cubic\milli\meter}
\end{align*}
\item Una masa de \SI{8}{\gram} en \unit{\kilo\gram}.
Este ejercicio es directo:
\begin{align*}
\SI{1}{\kilo\gram} = \SI{d3}{\gram}
\end{align*}
Ocupamos este factor de conversión:
\begin{align*}
\SI{8}{\gram} \left( \dfrac{\SI{1}{\kilo\gram}}{\SI{d3}{\gram}} \right) = \SI{8d-3}{\kilo\gram}
\end{align*}
\end{enumerate}
\item (\textbf{1 punto}) Considera que el protón tiene una forma cúbica y cuya arista es de \SI{10d-13}{\centi\meter}:

Recordamos que la arista de un cubo es el lado común de dos caras:
\begin{figure}[H]
\centering
\begin{tikzpicture}
        \draw[thick](2,2,0)--(0,2,0)--(0,2,2)--(2,2,2)--(2,2,0)--(2,0,0)--(2,0,2)--(0,0,2)--(0,2,2);
        \draw[thick](2,2,2)--(2,0,2);
        \draw[gray](2,0,0)--(0,0,0)--(0,2,0);
        \draw[gray](0,0,0)--(0,0,2);
        \draw (0.35, -1.3, 0) node {\footnotesize{$l = \SI{10d-13}{\meter}$}};
        \draw (2.8, 1.5, 1) node {\footnotesize{$l$}};
        \draw (2.1, -0.3, 0.5) node {\footnotesize{$l$}};
\end{tikzpicture}
\end{figure}
\begin{enumerate}
\item Calcula su volumen.

El volumen lo obtenemos al multiplicar el largo por el ancho y la altura del cubo:
\begin{align*}
V &=  l \times\ l \times l = (\SI{10d-13}{\centi\meter})(\SI{10d-13}{\centi\meter})(\SI{10d-13}{\centi\meter}) = \\[0.5em]
&=(\SI{10d-13}{\centi\meter})^{3} = \\[0.5em]
&= \SI{1d-36}{\cubic\centi\meter}
\end{align*}
\item Si la masa del protón es de \SI{d-24}{\gram}, determina su densidad. La densidad de un objeto se obtiene al dividir su masa entre el volumen (el que obtuviste en el inciso anterior).

Como ya tenemos el volumen del protón, solo resta usar la operación que se nos indica en el resultado y así obtener la densidad, que representamos con la letra griega \textbf{rho} ($\rho$):
\begin{align*}
\rho = \dfrac{m}{V} = \dfrac{\SI{d-24}{\gram}}{\SI{1d-36}{\cubic\centi\meter}} = \SI{1d12}{\gram\per\cubic\centi\meter}
\end{align*}
\end{enumerate}
\item (\textbf{1 punto}) Ordena las siguientes cinco cantidades de masa, de la más grande a la más pequeña (tendrás que ocupar la tabla de múltiplos y submúltiplos que presentamos en clase):
\begin{enumerate}
\item \SI{0.032}{\kilo\gram}
\item \SI{15}{\gram}
\item \SI{2.7d5}{\milli\gram}
\item \SI{4.1d-8}{\giga\gram}
\item \SI{2.7d8}{\micro\gram}
\end{enumerate}
Si dos masas son iguales, dales igual lugar en la lista.

Para resolver este ejercicio, debemos de elegir una unidad y convertir las restantes para ordenarlas de manera descendente, será muy conveniente usar la notación científica, elegimos el gramo (\unit{\gram}), recordemos además de los múltiplos y submúltiplos:
\begin{align*}
\text{micro} \, (\mu) = \num{d-6} \hspace{0.2cm} &\Rightarrow \hspace{0.2cm} \SI{1}{\micro\gram} = \SI{1d-6}{\gram} \\
\text{mili} \, (\unit{m})  = \num{d-3} \hspace{0.2cm} &\Rightarrow \hspace{0.2cm} \SI{1}{\milli\gram} = \SI{1d-3}{\gram} \\
\text{kilo} \, (\unit{k}) = \num{d3} \hspace{0.2cm} &\Rightarrow \hspace{0.2cm} \SI{1}{\kilo\gram} = \SI{d3}{\gram} \\
\text{Giga} \, (\unit{G}) = \num{d9} \hspace{0.2cm} &\Rightarrow \hspace{0.2cm} \SI{1}{\giga\gram} = \SI{d9}{\gram}
\end{align*}
Por lo que al ajustar todas las unidades a gramos, tenemos que:
\begin{enumerate}[label=\alph*)]
\item $\SI{0.032}{\kilo\gram} = \SI{0.032}{\kilo\gram} \left( \dfrac{\SI{d3}{\gram}}{\SI{1}{\kilo\gram}} \right) = \SI{32}{\gram}$
\item $\SI{15}{\gram} = \SI{15}{\gram}$
\item $\SI{2.7d5}{\milli\gram} = \SI{2.7d5}{\milli\gram} \left( \dfrac{\SI{d-3}{\gram}}{\SI{1}{\milli\gram}} \right) = \SI{270}{\gram}$
\item $\SI{4.1d-8}{\giga\gram} = \SI{4.1d-8}{\giga\gram} \left( \dfrac{\SI{d9}{\gram}}{\SI{1}{\giga\gram}} \right) = \SI{41}{\gram}$
\item $\SI{2.7d8}{\micro\gram} = \SI{2.7d8}{\micro\gram} \left( \dfrac{\SI{d-6}{\gram}}{\SI{1}{\micro\gram}} \right) = \SI{270}{\gram}$
\end{enumerate}
Ahora nos queda ordenar los incisos de mayor a menor:
\begin{center}
    c), \quad e), \quad d), \quad a), \quad b)
\end{center}
\item (\textbf{1 punto}) Supongamos que un cabello crece a una proporción de $1/32$ pulgada por cada día. Encuentra la proporción a la que crece en nanómetros por segundo. Recuerda que 1 pulgada equivale a \SI{2.54}{\centi\meter}.

Este ejercicio es un claro ejemplo en donde debemos de utilizar varios factores de conversión para responder la pregunta:
\begin{align*}
1 \, \text{pulgada} &= \SI{2.54}{\centi\meter} = \SI{2.54d-2}{\meter} \\
\SI{1}{\nano\meter} &= \SI{d-9}{\meter} \\
1 \, \text{día} &= (\SI{24}{\hour})({\SI{60}{\minute}})(\SI{60}{\second}) = \SI{8.64d4}{\second}
\end{align*}
Ahora hagamos la conversión:
\begin{align*}
&\dfrac{1}{32} \, \dfrac{\text{pulgada}}{\text{día}} \left( \dfrac{\SI{2.54d-2}{\meter}}{1 \, \text{pulgada}} \right) \left( \dfrac{\SI{1}{\nano\meter}}{\SI{d-9}{\meter}} \right) \left( \dfrac{1 \, \text{día}}{\SI{8.64d4}{\second}} \right) = \\[0.5em]
&= \dfrac{\SI{2.54d-2}{\nano\meter}}{\SI{2.7648d3}{\second}} = \\[0.5em]
&= \SI[per-mode=fraction]{9.1869}{\nano\meter\per\second}
\end{align*}
Que es la respuesta que se nos pidió en el enunciado.
\item (\textbf{1 punto}) Un auditorio mide $\SI{40.0}{\meter} \times \SI{20.0}{\meter} \times \SI{12.0}{\meter}$. La densidad del aire es \SI{1.20}{\kilo\gram\per\cubic\meter}.

Un pie equivale a \SI{30.48}{\centi\meter}, \SI{453.59}{\gram} equivalen a una libra, la densidad ($\rho$) de una sustancia está dada por la relación entre la masa ($m$) y el volumen (V):
\begin{align*}
\rho = \dfrac{m}{V}
\end{align*}

¿Cuáles son:
\begin{enumerate}
\item El volumen de la habitación en pies cúbicos.
Para responder este ejercicio, calculamos primero el volumen del auditorio en metros cúbicos:
\begin{align*}
V = \SI{40.0}{\meter} \times \SI{20.0}{\meter} \times \SI{12.0}{\meter} = \SI{9.6d3}{\cubic\meter}
\end{align*}
Hacemos la conversión a pies cúbicos, como recordarás, no tenemos un factor de conversión \enquote{directo}, por lo que tenemos que construirlo:
\begin{align*}
1 \, \text{pie} &= \SI{0.3048}{\meter} \\
(1 \, \text{pie})^{3} &= (\SI{0.3048}{\meter})^{3} \\
1 \, \text{pie}^{3} &= \SI{2.8316d-2}{\cubic\meter}
\end{align*}
Entonces la conversión queda:
\begin{align*}
\SI{9.6d3}{\cubic\meter} \left( \dfrac{1 \, \text{pie}^{3}}{\SI{2.8316d-2}{\cubic\meter}} \right) = \num{3.3903d5} \, \text{pie}^{3}
\end{align*}
\item La masa en libras del aire en la habitación?

Debemos de ocupar la expresión que ya conocemos:
\begin{align*}
\rho = \dfrac{m}{V} \hspace{0.3cm} \Rightarrow \hspace{0.3cm} m = \rho \, V
\end{align*}
Sustituimos los valores que ya tenemos:
\begin{align*}
m = \left( \SI[per-mode=fraction]{1.20}{\kilo\gram\per\cubic\meter} \right) \left( \SI{9.6d3}{\cubic\meter} \right) = \SI{1.152d4}{\kilo\gram}
\end{align*}
Este resulta queda expresado en kilogramos, pero nos piden la masa del aire contenido en el auditorio en libras, por lo que hay que hacer la conversión, el factor de libras a \unit{\kilo\gram} es conocido:
\begin{align*}
1 \, \text{libra} = \SI{0.45359}{\kilo\gram}
\end{align*}
Por lo que la conversión a realizar es:
\begin{align*}
m = \SI{1.152d4}{\kilo\gram} \left( \dfrac{1 \, \text{libra}}{\SI{0.45359}{\kilo\gram}} \right) = \num{2.5397d4} \, \text{libras}
\end{align*}
\end{enumerate}
\item (\textbf{1 punto}) Al colocar con mucho cuidado sobre una superficie libre de un recipiente con agua, una gota de aceite cuyo volumen es $V = \SI{6d-2}{\cubic\centi\meter}$, la misma gota se dispersa y forma una capa muy fina cuya área es $A = \SI{2d4}{\square\centi\meter}$. Calcula el espesor de esa lámina de aceite.

Para resolver esté ejercicio hay que plantearse la expresión que nos dará la respuesta: si todo el volumen se extiende sobre la superficie, entonces el espesor ($x$ en centímetros) de la lámina de aceite viene dado por:
\begin{align*}
x = \dfrac{V}{A} = \dfrac{\SI{6d-2}{\cubic\centi\meter}}{\SI{2d4}{\square\centi\meter}} = \SI{3d-6}{\centi\meter}
\end{align*}
Las unidades quedan en \unit{\centi\meter} ya que por la regla de los exponentes, tenemos que:
\begin{align*}
\dfrac{\unit{\cubic\centi\meter}}{\unit{\square\centi\meter}} = \unit{\centi\meter}^{3-2} = \unit{\centi\meter}
\end{align*}
\item (\textbf{1 punto}) Calcula el número de latidos que ha realizado el corazón de un paciente de $85$ años (en el mismo día de su cumpleaños). El promedio razonable en una persona sana y en reposo es de $80$ latidos por minuto. Expresa el resultado en notación científica.

La operación es directa:
\begin{align*}
&80 \, \dfrac{\text{latidos}}{\unit{\minute}} \left( \dfrac{\SI{60}{\minute}}{\SI{1}{\hour}} \right) \left( \dfrac{\SI{24}{\hour}}{1 \, \text{día}} \right) \left(\dfrac{\num{365} \, \text{día}}{1 \, \text{año}} \right) \left( \num{85} \, \text{años} \right) = \\[0.5em]
&= \num{3.57408d9} \, \text{latidos}
\end{align*}
\end{enumerate}

\end{document}