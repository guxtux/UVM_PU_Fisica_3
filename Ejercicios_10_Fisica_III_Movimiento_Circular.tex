\documentclass[14pt]{extarticle}
\usepackage[utf8]{inputenc}
\usepackage[T1]{fontenc}
\usepackage[spanish,es-lcroman]{babel}
\usepackage{amsmath}
\usepackage{amsthm}
\usepackage{physics}
\usepackage{tikz}
\usepackage{float}
\usepackage[autostyle,spanish=mexican]{csquotes}
\usepackage[per-mode=symbol]{siunitx}
\usepackage{gensymb}
\usepackage{multicol}
\usepackage{enumitem}
\usepackage[left=2.00cm, right=2.00cm, top=2.00cm, 
     bottom=2.00cm]{geometry}

%\renewcommand{\questionlabel}{\thequestion)}
\decimalpoint
\sisetup{bracket-numbers = false}

\renewcommand*{\theenumi}{\thesection.\arabic{enumi}}
\renewcommand*{\theenumii}{\theenumi.\arabic{enumii}}

\title{\vspace*{-2cm} Ejercicios Movimiento Circular \\  Evaluación Continua - Física III\vspace{-5ex}}
\date{9 de abril de 2024}

\begin{document}
\maketitle

\vspace*{0.75cm}
\textbf{Importante: } Esta actividad otorgará hasta \textbf{9 puntos}, un punto por cada ejercicio. Podrás apoyarte con tus notas.

Cada punto por ejercicio se divide de la siguiente forma: a) \textbf{Datos} del ejercicio, b) \textbf{Expresión(es)} que se necesite(n), si se requiere un manejo algebraico, aquí debe de detallarse, c) \textbf{Sustitución} de datos y operaciones artiméticas, y d) \textbf{Manejo de unidades}, que debe de estar presente en cada renglón que ocupes para la solución.

Deberás de entregar el desarrollo completo de cada ejercicio, si se reporta el resultado directo, sin presentar el desarrollo, como se indica en los incisos anteriomente mencionados, no contará como ejercicio resuelto.

\section{Ejercicios.}

\begin{enumerate}
\item Expresa en radianes los siguientes ángulos: a) \ang{45.0}, b) \ang{60.0}, c) \ang{90.0}, \break d) \ang{390.0} y e) \ang{445.0}? Reporta los valores como valores numéricos (valores con decimales) y como fracciones de $\pi$ (como múltiplos de $\pi$).
\item Un disco compacto tuvo una magnitud de aceleración angular de \SI{5}{\radian\per\square\second} durante \SI{6}{\second}. ¿Qué magnitud de velocidad angular final adquirió?
\item Si una hélice con una magnitud de velocidad angular inicial de \SI{15}{\radian\per\second} recibe una aceleración angular cuya magnitud es de \SI{7}{\radian\per\second} durante \SI{0.2}{\minute}, ¿cuáles son las magnitudes de la velocidad angular final y del desplazamiento angular que alcanzó a los \SI{0.2}{\minute}?
\item Un rehilete aumentó la magnitud de su velocidad angular de \SI{12}{\radian\per\second} a \SI{60}{\radian\per\second} en \SI{4}{\second}. ¿Cuál fue la magnitud de su aceleración angular?
\item Una rueda gira con una magnitud de velocidad angular inicial de \SI{12}{\radian\per\second} y recibe una aceleración angular cuya magnitud es de \SI{6}{\radian\per\square\second} durante \SI{13}{\second}.

Calcula:
\begin{enumerate}[label=\alph*)]
\item ¿Qué magnitud de velocidad angular lleva al cabo de los \num{13} segundos?
\item ¿Qué magnitud de desplazamiento angular tuvo?
\end{enumerate}
\item Un disco que gira a \num{2}  rev/s aumenta su frecuencia a \num{50} rev/s en \SI{3}{\second}. Determina cuál fue la magnitud de su aceleración angular en \unit{\radian\per\square\second}.
\item Una rueda de la fortuna gira inicialmente con una magnitud de velocidad angular de \SI{2}{\radian\per\second}. Si recibe una aceleración angular cuya magnitud es de \SI{1.5}{\radian\per\second} durante \num{5} segundos.

Calcula.
\begin{enumerate}[label=\alph*)]
\item ¿Cuál será la magnitud de su velocidad angular a los \SI{5}{\second}?
\item ¿Cuál será la magnitud de su desplazamiento angular?
\item ¿Cuántas revoluciones habrá dado al término de los \SI{5}{\second}?
\end{enumerate}
\item En un juego mecánico, los pasajeros viajan con velocidad constante en un círculo de \SI{5.0}{\meter} de radio, dando una vuelta completa cada \SI{4.0}{\second}. ¿Qué aceleración centrípeta tienen?
\item El radio de la órbita terrestre alrededor del Sol (suponiendo que fuera circular) es de \SI{1.5d8}{\kilo\meter} y la Tierra la recorre en \num{365} días.
\begin{enumerate}[label=\alph*)]
\item Calcula la magnitud de la velocidad tangencial de la Tierra.
\item Calcula la aceleración centrípeta de la Tierra hacia el Sol.
\end{enumerate}
\end{enumerate}

\end{document}