\documentclass[14pt]{extarticle}
\usepackage[utf8]{inputenc}
\usepackage[T1]{fontenc}
\usepackage[spanish,es-lcroman]{babel}
\usepackage{amsmath}
\usepackage{amsthm}
\usepackage{physics}
\usepackage{tikz}
\usepackage{float}
\usepackage[autostyle,spanish=mexican]{csquotes}
\usepackage[per-mode=symbol]{siunitx}
\usepackage{gensymb}
\usepackage{multicol}
\usepackage{enumitem}
\usepackage[left=2.00cm, right=2.00cm, top=2.00cm, 
     bottom=2.00cm]{geometry}

%\renewcommand{\questionlabel}{\thequestion)}
\decimalpoint
\sisetup{bracket-numbers = false}

\title{\vspace*{-2cm} Ejercicios de Notación Científica \\  Evaluación Continua - Física III\vspace{-5ex}}
\date{}

\begin{document}
\maketitle

\textbf{Indicaciones:} Resuelve de manera detallada cada uno de los siguientes ejercicios, en donde deberás de indicar el paso (o pasos necesarios) para llegar al resultado, en el caso de las conversiones de unidades, deberás de anotar el(los) factor(es) de conversión necesarios.
\par
Al inicio de cada hoja que utilices para tu solución, deberás de anotar tu nombre completo.
\par
Esta actividad otorga hasta \textbf{8 puntos}. Si se reporta el resultado directo, sin presentar el desarrollo del ejercicio, éste no aporta puntaje.
\par
La solución se deberá de enviar por Teams (ya sea en foto o la hoja escaneada).

\begin{enumerate}
\item (\textbf{1 punto}) Realiza las siguientes operaciones con números en notación científica, ocupa tu calculadora científica (la finalidad es que vayas reconociendo su uso), el resultado debe de quedar con un dígito antes del punto decimal y cuatro dígitos después del punto.
\begin{enumerate}
    \item (\num{3.4d23})(\num{1.1d11}) = 
    \item (\num{2.45d-12})(\num{2.45d-4}) =
    \item (\num{4.56d11})(\num{0.655d-20}) =
    \item \num{2.2d5} / \num{0.6d45} =
    \item \num{0.68d-8} / \num{1.9d-13} =
    \item $\dfrac{(\num{4.3d5})(\num{2.23d2})(\sqrt{\num{1.4d-6}})}{(\num{9d4})^{2}}=$
\end{enumerate}
\item (\textbf{1 punto}) Con notación científica de base $10$, expresa:
\begin{enumerate}
\item Un área de \SI{2}{\square\kilo\meter} en \unit{\square\centi\meter}.
\item Un volumen de \SI{5}{\cubic\centi\meter} en \unit{\cubic\meter}.
\item Un volumen de \SI{4}{\liter} en \unit{\cubic\milli\meter}, considera que en \SI{1}{\liter} hay \SI{d3}{\cubic\centi\meter}.
\item Una masa de \SI{8}{\gram} en \unit{\kilo\gram}.
\end{enumerate}
\item (\textbf{1 punto}) Considera que el protón tiene una forma cúbica y cuya arista es de \SI{10d-13}{\centi\meter}:
\begin{enumerate}
\item Calcula su volumen.
\item Si la masa del protón es de \SI{d-24}{\gram}, determina su densidad. La densidad de un objeto se obtiene al dividir su masa entre el volumen (el que obtuviste en el inciso anterior).
\end{enumerate}
\item (\textbf{1 punto}) Ordena las siguientes cinco cantidades de masa, de la más grande a la más pequeña (tendrás que ocupar la tabla de múltiplos y submúltiplos que presentamos en clase):
\begin{enumerate}
\item \SI{0.032}{\kilo\gram}
\item \SI{15}{\gram}
\item \SI{2.7d5}{\milli\gram}
\item \SI{4.1d-8}{\giga\gram}
\item \SI{2.7d8}{\micro\gram}
\end{enumerate}
Si dos masas son iguales, dales igual lugar en la lista.
\item (\textbf{1 punto}) Supongamos que un cabello crece a una proporción de $1/32$ pulgada por cada día. Encuentra la proporción a la que crece en nanómetros por segundo. Recuerda que 1 pulgada equivale a \SI{2.54}{\centi\meter}.
\item (\textbf{1 punto}) Un auditorio mide $\SI{40.0}{\meter} \times \SI{20.0}{\meter} \times \SI{12.0}{\meter}$. La densidad del aire es \SI{1.20}{\kilo\gram\per\cubic\meter}. ¿Cuáles son:
\begin{enumerate}
\item El volumen de la habitación en pies cúbicos.
\item La masa en libras del aire en la habitación?
\end{enumerate}
Un pie equivale a \SI{30.48}{\centi\meter}, \SI{453.59}{\gram} equivalen a una libra, la densidad ($\rho$) de una sustancia está dada por la relación entre la masa ($m$) y el volumen (V):
\begin{align*}
\rho = \dfrac{m}{V}
\end{align*}
\item (\textbf{1 punto}) Al colocar con mucho cuidado sobre una superficie libre de un recipiente con agua, una gota de aceite cuyo volumen es $V = \SI{6d-2}{\cubic\centi\meter}$, la misma gota se dispersa y forma una capa muy fina cuya área es $A = \SI{2d4}{\square\centi\meter}$. Calcula el espesor de esa lámina de aceite.
\item (\textbf{1 punto}) Calcula el número de latidos que ha realizado el corazón de un paciente de $85$ años (en el mismo día de su cumpleaños). El promedio razonable en una persona sana y en reposo es de $80$ latidos por minuto. Expresa el resultado en notación científica.
\end{enumerate}

\end{document}