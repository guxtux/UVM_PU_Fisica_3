\documentclass[14pt]{extarticle}
\usepackage[utf8]{inputenc}
\usepackage[T1]{fontenc}
\usepackage[spanish,es-lcroman]{babel}
\usepackage{amsmath}
\usepackage{amsthm}
\usepackage{physics}
\usepackage{tikz}
\usepackage{float}
\usepackage[autostyle,spanish=mexican]{csquotes}
\usepackage[per-mode=symbol]{siunitx}
\usepackage{gensymb}
\usepackage{multicol}
\usepackage{enumitem}
\usepackage[left=2.00cm, right=2.00cm, top=2.00cm, 
     bottom=2.00cm]{geometry}

%\renewcommand{\questionlabel}{\thequestion)}
\decimalpoint
\sisetup{bracket-numbers = false}

\title{\vspace*{-2cm} Los planetas del Sistema Solar \\  Evaluación Continua - Física III\vspace{-5ex}}
\date{}

\begin{document}
\maketitle

Para apoyar los temas de las leyes de Newton y las leyes de Kepler, se te pide que consultes ya sea libros, enciclopedias u otras fuentes de información, para elaborar una tabla con datos específicos de los planetas del Sistema Solar, recuerda que Plutón fue descatalogado como planeta por la Asociación Astronómica Internacional, pero bien podrías incluirlo.

\vspace*{0.5cm}
Los datos que se piden son:
\begin{enumerate}
\item La distancia promedio al Sol en \unit{\kilo\meter} o en un unidades astronómicas (habrá que revisar la definición de unidad astronómica UA).
\item El período de rotación sobre su propio eje, para el caso de la Tierra, sabemos que son $24$ horas.
\item El período de traslación alrededor del Sol, para el caso de la Tierra, sabemos que son $365$ días = $1$ año.
\item El valor de la aceleración debida a la gravedad, para la Tierra $g = \SI{9.81}{\meter\per\square\second}$
\item Responder el peso que tendría un objeto de \SI{1000}{\kilo\gram} en la superficie de cada planeta.
\end{enumerate}
Se comentó en la clase que la mejor manera de organizar los datos es haciendo una tabla.

\vspace*{0.5cm}
\textbf{Esta actividad te otorgará tres puntos}, siempre y cuando esté completa y se envíe oportunamente, antes del cierre de la asignación.

\vspace*{0.5cm}
El formato de entrega puede ser escrito a mano, o en computadora. Deberás de anotar la referencia completa que hayas utilizado (libro, revista, sitio web, enciclopedias, etc.)

\end{document} 