\documentclass[14pt]{extarticle}
\usepackage[utf8]{inputenc}
\usepackage[T1]{fontenc}
\usepackage[spanish,es-lcroman]{babel}
\usepackage{amsmath}
\usepackage{amsthm}
\usepackage{physics}
\usepackage{tikz}
\usepackage{float}
\usepackage[autostyle,spanish=mexican]{csquotes}
\usepackage[per-mode=symbol]{siunitx}
\usepackage{gensymb}
\usepackage{multicol}
\usepackage{multirow}
\usepackage{enumitem}
\usepackage[left=2.00cm, right=2.00cm, top=2.00cm, 
     bottom=2.00cm]{geometry}

%\renewcommand{\questionlabel}{\thequestion)}
\decimalpoint
\sisetup{bracket-numbers = false}

\title{\vspace*{-2cm} Calificación del examen parcial\vspace{-5ex}}
\date{}

\begin{document}
\maketitle

La calificación del examen parcial se obtiene de tres componentes, en la siguiente tabla se indica el porcentaje que representa cada uno:
\begin{table}[H]
\centering
\begin{tabular}{| c | l |} \hline
Rubro & Componente \\ \hline
\multirow{2}{*}{Teoría (70\%)} & 1. Evaluación Continua (40\%) \\
 & 2. Examen (60\%) \\ \hline
Laboratorio (30\%) & 3. Laboratorio \\ \hline
\end{tabular}
\end{table}
Una vez conocidas las calificaciones de los tres componentes, podremos obtener la calificación del examen parcial:
\begin{table}[H]
    \centering
    \begin{tabular}{| l | c | c | c |} \hline
        Rubro & Calificación & Multiplicador & Puntaje \\ \hline
        Evaluación Cont. &  & $0.4$ & $a) $ \\ \hline
        Examen & & $0.6$ & $b)$ \\ \hline
         & & Teoría ($a + b$) & $c)$ \\ \hline
    \end{tabular}
\end{table}
\begin{table}[H]
    \centering
    \begin{tabular}{| l | c | c | c |} \hline
        Rubro & Calificación & Multiplicador & Puntaje \\ \hline
        Teoría $c)$ & & $0.7$ & $d)$ \\ \hline
        Laboratorio & & $0.3$ & $e)$ \\ \hline
         & & Calif. parcial $d + e$ & \\ \hline        
    \end{tabular}
\end{table}
\noindent
\textbf{Ejemplo.} Luego de la retroalimentación tanto de teoría como de laboratorio,  ya tengo las calificaciones:

\vspace*{0.5cm}
\noindent
Evaluación Continua: $9.5$, Examen: $8.5$, Laboratorio: $10$
\begin{table}[H]
    \centering
    \begin{tabular}{| l | c | c | c |} \hline
        Rubro & Calificación & Multiplicador & Puntaje \\ \hline
        Evaluación Cont. & $9.5$ & $0.4$ & $a) \, 3.8$ \\ \hline
        Examen & $8.5$ & $0.6$ & $b) \, 5.1$ \\ \hline
         & & Teoría ($a + b$) & $c) \, 8.9$ \\ \hline
    \end{tabular}
\end{table}
\begin{table}[H]
    \centering
    \begin{tabular}{| l | c | c | c |} \hline
        Rubro & Calificación & Multiplicador & Puntaje \\ \hline
        Teoría $c)$ & $8.9$ & $0.7$ & $d) \, 6.23$ \\ \hline
        Laboratorio & $10$ & $0.3$ & $e) \, 3$ \\ \hline
         & & Calif. parcial $d + e$ & $9.23$  \\ \hline        
    \end{tabular}
\end{table}
La calificación para el curso de Física en mi historial académico será: \underline{$9.23$}
\end{document}