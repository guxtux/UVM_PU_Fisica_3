\documentclass[12pt, letter]{exam}
\usepackage[utf8]{inputenc}
\usepackage[T1]{fontenc}
\usepackage[spanish]{babel}
\usepackage[autostyle,spanish=mexican]{csquotes}
\usepackage{amsmath}
\usepackage{amsthm}
\usepackage{physics}
\usepackage{tikz}
\usepackage{float}
\usepackage{siunitx}
\usepackage{multicol}
\usepackage{enumitem}
\usepackage[left=2.00cm, right=2.00cm, top=2.00cm, 
     bottom=2.00cm]{geometry}
\usepackage{pdfpages}
\usepackage{wasysym}

% \renewcommand{\questionlabel}{\thequestion}
\decimalpoint

\setlength{\belowdisplayskip}{-0.5pt}

\usepackage{tasks}
\settasks{
    label=\Alph*), 
    label-align=left,
    item-indent={20pt}, 
    column-sep={4pt},
    label-width={16pt},
}

\sisetup{per-mode=symbol}
\footer{}{\thepage}{}
\date{}

\title{Solución a los ejercicios de ejecución - Versión B \\ Cuarto Examen Parcial de Física III  - Grupo 47}

\begin{document}
\maketitle

\setcounter{page}{2}

\begin{questions}
    \section{Movimiento circular.}

    \setcounter{question}{4} \question \label{Ejercicio_02} \textbf{Ejercicio de ejecución: }  Un ventilador gira a 10 rps (rev/seg). Determina la aceleración centrípeta del extremo del aspa, si la distancia desde el centro al extremo es de \SI{30.0}{\centi\meter}.

    \begin{minipage}[t]{0.4\linewidth}
    Datos: 
    \begin{align*}
    f &= 10 \, \text{rps} = \SI{10}{\hertz} \\
    r &= \SI{30.0}{\centi\meter} = \SI{0.3}{\meter} \\
    \omega &= \, ? \\
    v_{T} &= \, \\
    a_{C} &= \, ?
    \end{align*}
    \end{minipage}
    \hspace{1cm}
    \begin{minipage}[t]{0.4\linewidth}
    Expresiones:
    \begin{align*}
    \omega &= 2 \, \pi \, f \\
    v_{T} &= \omega \, r \\
    a_{C} &= \dfrac{v_{T}^{2}}{r}
    \end{align*}
    \end{minipage}

    Sustitución:
    \begin{align*}
    \omega &= 2 \, \pi (\SI{10}{\hertz}) \, \unit{\radian} = \SI[per-mode=fraction]{62.83}{\radian\per\second} \\
    v_{T} &= \left( \SI[per-mode=fraction]{62.83}{\radian\per\second} \right) (\SI{0.3}{\meter}) = \SI[per-mode=fraction]{18.84}{\meter\per\second} \\
    a_{C} &= \dfrac{\left( \displaystyle \SI[per-mode=fraction]{18.84}{\meter\per\second} \right)^{2}}{\SI{0.3}{\meter}} = \SI[per-mode=fraction]{1184.35}{\meter\per\square\second}
    \end{align*}

    \vspace{0.3cm}
    \begin{tasks}(4)
        \task $\displaystyle \SI[per-mode=fraction]{247.68}{\meter\per\square\second}$
        \task $\displaystyle \SI[per-mode=fraction]{523.02}{\meter\per\square\second}$
        \task \fbox{$\displaystyle \SI[per-mode=fraction]{1184.35}{\meter\per\square\second}$}
        \task $\displaystyle \SI[per-mode=fraction]{2188.82}{\meter\per\square\second}$
    \end{tasks}

    \question \textbf{Ejercicio de ejecución: } Convierte \ang{1500} a radianes.
    
     
    \begin{minipage}[t]{0.35\linewidth}
    Datos: 
    \begin{align*}
    \varangle{\ang{1500}}
    \end{align*}
    \end{minipage}
    \hspace{1cm}
    \begin{minipage}[t]{0.4\linewidth}
    Expresión:
    \begin{align*}
    \ang{360} = 2 \, \pi \, \unit{\radian}
    \end{align*}
    \end{minipage}

    Sustitución:
    \begin{align*}
    \dfrac{ \ang{1500} (2 \, \pi)}{\ang{360}} = \SI{26.17}{\radian}
    \end{align*}

    \vspace{0.3cm}
    
    \begin{tasks}(4)
        \task \fbox{\SI{26.17}{\radian}}
        \task \SI{38.78}{\radian}
        \task \SI{61.99}{\radian}
        \task \SI{135.38}{\radian}
    \end{tasks}

    \question \label{Ejercicio_03} \textbf{Ejercicio de ejecución: } Una centrifugadora está girando a 20000 rpm y un cambio de velocidad reduce a 15000 rpm en \SI{25}{\second} ¿Cuál es su aceleración angular?

    \begin{minipage}[t]{0.4\linewidth}
    Datos: 
    \begin{align*}
    f_{f} &= 20000 \, \text{rpm} = \SI{333.33}{\hertz} \\
    f_{0} &= 15000 \, \text{rpm} = \SI{250}{\hertz} \\
    t &= \SI{25}{\second} \\
    \omega_{0} &= \, ? \\
    \omega_{f} &= \, ? \\
    \alpha &= \, ?
    \end{align*}
    \end{minipage}
    \hspace{1cm}
    \begin{minipage}[t]{0.4\linewidth}
    Expresiones:
    \begin{align*}
    \omega &= 2 \, \pi \, f \\
    \alpha &= \dfrac{\omega_{f} - \omega_{0}}{t}
    \end{align*}
    \end{minipage}

    Sustitución:
    \begin{align*}
    \omega_{0} &= 2 \, \pi (\SI{333.33}{\hertz}) \, \unit{\radian} = \SI[per-mode=fraction]{2094.37}{\radian\per\second} \\[0.5em]
    \omega_{f} &= 2 \, \pi (\SI{250}{\hertz}) \, \unit{\radian} = \SI[per-mode=fraction]{1570.79}{\radian\per\second} \\[0.5em]
    \alpha &= \dfrac{ \displaystyle \SI[per-mode=fraction]{1570.79}{\radian\per\second} - \SI[per-mode=fraction]{2094.39}{\radian\per\second} }{\SI{25}{\second}} = - \SI[per-mode=fraction]{20.94}{\radian\per\square\second}
    \end{align*}

    \vspace{0.3cm}
    \begin{tasks}(4)
        \task $\displaystyle -\SI[per-mode=fraction]{11.04}{\radian\per\square\second}$
        \task $\displaystyle -\SI[per-mode=fraction]{16.75}{\radian\per\square\second}$
        \task \fbox{$\displaystyle -\SI[per-mode=fraction]{20.94}{\radian\per\square\second}$}
        \task $\displaystyle -\SI[per-mode=fraction]{58.11}{\radian\per\square\second}$
    \end{tasks}


\end{questions}
\end{document}