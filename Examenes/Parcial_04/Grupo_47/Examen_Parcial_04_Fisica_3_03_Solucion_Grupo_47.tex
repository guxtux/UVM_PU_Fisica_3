\documentclass[12pt, letter]{exam}
\usepackage[utf8]{inputenc}
\usepackage[T1]{fontenc}
\usepackage[spanish]{babel}
\usepackage[autostyle,spanish=mexican]{csquotes}
\usepackage{amsmath}
\usepackage{amsthm}
\usepackage{physics}
\usepackage{tikz}
\usepackage{float}
\usepackage{siunitx}
\usepackage{multicol}
\usepackage{enumitem}
\usepackage[left=2.00cm, right=2.00cm, top=2.00cm, 
     bottom=2.00cm]{geometry}
\usepackage{pdfpages}
\usepackage{wasysym}

% \renewcommand{\questionlabel}{\thequestion}
\decimalpoint

\setlength{\belowdisplayskip}{-0.5pt}

\usepackage{tasks}
\settasks{
    label=\Alph*), 
    label-align=left,
    item-indent={20pt}, 
    column-sep={4pt},
    label-width={16pt},
}

\sisetup{per-mode=symbol}
\footer{}{\thepage}{}
\date{}

\title{Solución a los ejercicios de ejecución - Versión C \\ Cuarto Examen Parcial de Física III  - Grupo 47}

\begin{document}
\maketitle

\setcounter{page}{2}

\begin{questions}
    \section{Movimiento circular.}

    \setcounter{question}{4} \question \label{Ejercicio_02} \textbf{Ejercicio de ejecución: }  Una rueda de \SI{40}{\centi\meter} de radio gira sobre un eje fijo. Su velocidad aumenta uniformemente desde el reposo hasta  alcanzar \num{900} rpm en un tiempo de \SI{20}{\second}. ¿Cuál es la aceleración angular de la rueda?

    \begin{minipage}[t]{0.4\linewidth}
    Datos: 
    \begin{align*}
    r &= \SI{40.0}{\centi\meter} = \SI{0.4}{\meter} \\
    \omega_{0} &= 0 \\
    f_{f} &= 900 \, \text{rpm} = \SI{15}{\hertz} \\
    t &= \SI{20}{\second} \\
    \omega_{f} &= \, ? \\
    \alpha &= \, ??
    \end{align*}
    \end{minipage}
    \hspace{1cm}
    \begin{minipage}[t]{0.4\linewidth}
    Expresiones:
    \begin{align*}
    \omega &= 2 \, \pi \, f \\
    \alpha &= \dfrac{\omega_{f} - \omega_{0}}{t}
    \end{align*}
    \end{minipage}

    Sustitución:
    \begin{align*}
    \omega_{f} &= 2 \, \pi (\SI{15}{\hertz}) \, \unit{\radian} = \SI[per-mode=fraction]{94.24}{\radian\per\second} \\
    \alpha &= \dfrac{ \displaystyle \SI[per-mode=fraction]{94.24}{\radian\per\second} - 0 }{\SI{20}{\second}} = \SI[per-mode=fraction]{4.71}{\radian\per\square\second}
    \end{align*}

    \vspace{0.3cm}
    \begin{tasks}(4)
        \task $\displaystyle \SI[per-mode=fraction]{2.47}{\meter\per\square\second}$
        \task $\displaystyle \SI[per-mode=fraction]{3.02}{\meter\per\square\second}$
        \task \fbox{$\displaystyle \SI[per-mode=fraction]{4.71}{\meter\per\square\second}$}
        \task $\displaystyle \SI[per-mode=fraction]{8.82}{\meter\per\square\second}$
    \end{tasks}

    \question \textbf{Ejercicio de ejecución: } La velocidad angular de un disco disminuye uniformemente de \SI{12.00}{\radian\per\second} a \SI{4.00}{\radian\per\second} en \SI{16}{\second}. Calcula la aceleración angular.    
    
    \begin{minipage}[t]{0.4\linewidth}
    Datos: 
    \begin{align*}
    \omega_{0} &= \SI{12}{\radian\per\second} \\
    \omega_{f} &= \SI{4}{\radian\per\second} \\
    t &= \SI{4}{\second} \\
    \alpha &= \, ?
    \end{align*}
    \end{minipage}
    \hspace{1cm}
    \begin{minipage}[t]{0.4\linewidth}
    Expresiones:
    \begin{align*}
    \alpha &= \dfrac{\omega_{f} - \omega_{0}}{t}
    \end{align*}
    \end{minipage}

    Sustitución:
    \begin{align*}
    \alpha &= \dfrac{ \displaystyle \SI[per-mode=fraction]{4}{\radian\per\second} - \SI[per-mode=fraction]{12}{\radian\per\second} }{\SI{16}{\second}} = - \SI[per-mode=fraction]{0.5}{\radian\per\square\second}
    \end{align*}

    \begin{tasks}(4)
        \task $\displaystyle -\SI[per-mode=fraction]{0.1}{\radian\per\square\second}$
        \task $\displaystyle -\SI[per-mode=fraction]{0.7}{\radian\per\square\second}$
        \task \fbox{$\displaystyle -\SI[per-mode=fraction]{0.5}{\radian\per\square\second}$}
        \task $\displaystyle -\SI[per-mode=fraction]{0.8}{\radian\per\square\second}$
    \end{tasks}
   
    \question \label{Ejercicio_03} \textbf{Ejercicio de ejecución: } Partiendo desde el reposo una piedra abrasiva tiene una aceleración angular constante de \SI{3.2}{\radian\per\square\second}. Calcula el desplazamiento angular de piedra abrasiva \SI{2.7}{\second} después de que comenzó a girar.

    \begin{minipage}[t]{0.4\linewidth}
    Datos: 
    \begin{align*}
    \theta_{0} &= 0 \\
    \omega_{i} &= 0 \\
    \alpha &= \SI{3.2}{\radian\per\square\second} \\
    t &= \SI{2.7}{\second} \\
    \theta &= \, ?
    \end{align*}
    \end{minipage}
    \hspace{1cm}
    \begin{minipage}[t]{0.4\linewidth}
    Expresión:
    \begin{align*}
    \theta = \omega_{0} \, t + \dfrac{1}{2} \, \alpha \,  t^{2}
    \end{align*}
    \end{minipage}

    Sustitución:
    \begin{align*}
    \theta &= \dfrac{1}{2} \left( \SI[per-mode=fraction]{3.2}{\radian\per\square\second} \right) \left( \SI{2.7}{\second} \right)^{2} = 0.5 \left( \SI[per-mode=fraction]{3.2}{\radian\per\square\second} \right) \left( \SI{7.29}{\square\second} \right) = \SI{11.66}{\radian}
    \end{align*}

    \vspace{0.3cm}
    \begin{tasks}(4)
        \task $\displaystyle \SI[per-mode=fraction]{14.29}{\radian}$
        \task $\displaystyle \SI[per-mode=fraction]{20.75}{\radian}$
        \task $\displaystyle \SI[per-mode=fraction]{58.94}{\radian}$
        \task \fbox{$\displaystyle \SI[per-mode=fraction]{11.66}{\radian}$}
    \end{tasks}


\end{questions}
\end{document}