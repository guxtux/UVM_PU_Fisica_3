\documentclass[12pt, letter]{exam}
\usepackage[utf8]{inputenc}
\usepackage[T1]{fontenc}
\usepackage[spanish]{babel}
\usepackage[autostyle,spanish=mexican]{csquotes}
\usepackage{amsmath}
\usepackage{amsthm}
\usepackage{physics}
\usepackage{tikz}
\usepackage{float}
\usepackage{siunitx}
\usepackage{multicol}
\usepackage{enumitem}
\usepackage[left=2.00cm, right=2.00cm, top=2.00cm, 
     bottom=2.00cm]{geometry}
\usepackage{pdfpages}

% \renewcommand{\questionlabel}{\thequestion}
\decimalpoint

\setlength{\belowdisplayskip}{-0.5pt}

\usepackage{tasks}
\settasks{
    label=\Alph*), 
    label-align=left,
    item-indent={20pt}, 
    column-sep={4pt},
    label-width={16pt},
}

\sisetup{per-mode=symbol}
\footer{}{\thepage}{}
\date{}

\title{Solución a los ejercicios de ejecución - Versión A \\ Cuarto Examen Parcial de Física III - Grupo 43}

\begin{document}
\maketitle

\setcounter{page}{2}

\begin{questions}
    \section{Movimiento circular.}

    \setcounter{question}{4} \question \textbf{Ejercicio de ejecución: } Convierte 2.18 radianes a grados.

    \begin{minipage}[t]{0.35\linewidth}
    Datos: 
    \begin{align*}
    \theta = \SI{2.18}{\radian}
    \end{align*}
    \end{minipage}
    \hspace{1cm}
    \begin{minipage}[t]{0.4\linewidth}
    Expresión:
    \begin{align*}
    \ang{360} = 2 \, \pi \, \unit{\radian}
    \end{align*}
    \end{minipage}

    Sustitución:
    \begin{align*}
    \SI{2.18}{\radian} \left( \dfrac{\ang{360}}{2 \, \pi \, \unit{\radian}} \right) = \ang{124.90}
    \end{align*}

    \vspace{0.3cm}
    \begin{tasks}(4)
        \task \ang{100.47}
        \task \fbox{\ang{124.90}}
        \task \ang{249.80}
        \task \ang{392.40}
    \end{tasks}
    \question \label{Ejercicio_02} \textbf{Ejercicio de ejecución: } Un ventilador gira a de 900 rpm (rev/min). Determina la velocidad tangencial del extremo del aspa, si la distancia desde el centro al extremo es de \SI{20.0}{\centi\meter}.

    \begin{minipage}[t]{0.4\linewidth}
    Datos: 
    \begin{align*}
    f &= 900 \, \text{rpm} = \SI{15}{\hertz} \\
    r &= \SI{20.0}{\centi\meter} = \SI{0.2}{\meter} \\
    \omega &= \, ? \\
    v_{T} &= \, ?
    \end{align*}
    \end{minipage}
    \hspace{1cm}
    \begin{minipage}[t]{0.4\linewidth}
    Expresiones:
    \begin{align*}
    \omega &= 2 \, \pi \, f \\
    v_{T} &= \omega \, r
    \end{align*}
    \end{minipage}

    Sustitución:
    \begin{align*}
    \omega &= 2 \, \pi (\SI{15}{\hertz}) \, \unit{\radian} = \SI{94.2}{\radian\per\second} \\
    v_{T} &= \left( \SI{94.2}{\radian\per\second} \right) (\SI{0.2}{\meter}) = \SI{18.8}{\meter\per\second}
    \end{align*}

    \vspace{0.3cm}
    \begin{tasks}(4)
        \task \SI{12.9}{\meter\per\second}
        \task \SI{17.4}{\meter\per\second}
        \task \fbox{\SI{18.8}{\meter\per\second}}
        \task \SI{29.5}{\meter\per\second}
    \end{tasks}
    \question \label{Ejercicio_03} \textbf{Ejercicio de ejecución: } Una centrifugadora acelera desde el reposo hasta \num{20000} rpm en \SI{30}{\second} ¿Cuál es su aceleración angular?

    \begin{minipage}[t]{0.4\linewidth}
    Datos: 
    \begin{align*}
    f &= 20000 \, \text{rpm} = \SI{333.33}{\hertz} \\
    t &= \SI{30}{\second} \\
    \omega_{0} &= 0 \\
    \alpha &= \, ?
    \end{align*}
    \end{minipage}
    \hspace{1cm}
    \begin{minipage}[t]{0.4\linewidth}
    Expresiones:
    \begin{align*}
    \omega &= 2 \, \pi \, f \\
    \alpha &= \dfrac{\omega_{f} - \omega_{0}}{t}
    \end{align*}
    \end{minipage}

    Sustitución:
    \begin{align*}
    \omega_{f} &= 2 \, \pi (\SI{333.33}{\hertz}) \, \unit{\radian} = \SI[per-mode=fraction]{2094.39}{\radian\per\second} \\[0.5em]
    \alpha &= \dfrac{ \SI{2094.39}{\radian\per\second} - 0 }{\SI{30}{\second}} = \SI[per-mode=fraction]{69.81}{\radian\per\square\second}
    \end{align*}

    \vspace{0.3cm}
    \begin{tasks}(4)
        \task \fbox{$\displaystyle \SI[per-mode=fraction]{69.81}{\radian\per\square\second}$}
        \task $\displaystyle \SI[per-mode=fraction]{100.43}{\radian\per\square\second}$
        \task $\displaystyle \SI[per-mode=fraction]{210.04}{\radian\per\square\second}$
        \task $\displaystyle \SI[per-mode=fraction]{350.18}{\radian\per\square\second}$
    \end{tasks}


\end{questions}
\end{document}