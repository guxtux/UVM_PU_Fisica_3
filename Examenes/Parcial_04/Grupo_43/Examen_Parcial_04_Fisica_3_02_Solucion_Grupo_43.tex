\documentclass[12pt, letter]{exam}
\usepackage[utf8]{inputenc}
\usepackage[T1]{fontenc}
\usepackage[spanish]{babel}
\usepackage[autostyle,spanish=mexican]{csquotes}
\usepackage{amsmath}
\usepackage{amsthm}
\usepackage{amssymb}
\usepackage{physics}
\usepackage{tikz}
\usepackage{float}
\usepackage{siunitx}
\usepackage{multicol}
\usepackage{enumitem}
\usepackage[left=2.00cm, right=2.00cm, top=2.00cm, 
     bottom=2.00cm]{geometry}
\usepackage{pdfpages}

% \renewcommand{\questionlabel}{\thequestion}
\decimalpoint

\setlength{\belowdisplayskip}{-0.5pt}

\usepackage{tasks}
\settasks{
    label=\Alph*), 
    label-align=left,
    item-indent={20pt}, 
    column-sep={4pt},
    label-width={16pt},
}

\sisetup{per-mode=symbol}
\footer{}{\thepage}{}
\date{}

\title{Solución a los ejercicios de ejecución - Versión B \\ Cuarto Examen Parcial de Física III - Grupo 43}

\begin{document}
\maketitle

\setcounter{page}{2}

\begin{questions}
    \section{Movimiento circular.}

    \setcounter{question}{4} \question \textbf{Ejercicio de ejecución: } Convierte \ang{140} a radianes.

    \begin{minipage}[t]{0.35\linewidth}
    Datos: 
    \begin{align*}
    \sphericalangle = \ang{140}
    \end{align*}
    \end{minipage}
    \hspace{1cm}
    \begin{minipage}[t]{0.4\linewidth}
    Expresión:
    \begin{align*}
    \ang{360} = 2 \, \pi \, \unit{\radian}
    \end{align*}
    \end{minipage}

    Sustitución:
    \begin{align*}
    \ang{140} \left( \dfrac{2 \, \pi \, \unit{\radian}}{\ang{360}} \right) = \SI{2.44}{\radian}
    \end{align*}

    \vspace{0.3cm}
    \begin{tasks}(4)
        \task $\pi$ \, \unit{\radian}
        \task \fbox{\SI{2.44}{\radian}}
        \task \SI{5.26}{\radian}
        \task \SI{7.95}{\radian}
    \end{tasks}

    \question \label{Ejercicio_02} \textbf{Ejercicio de ejecución: } Las aspas de una licuadora giran a razón de \num{6500} rpm. Cuando el motor se apaga durante la operación, las aspas frenan hasta llegar al reposo en \SI{4.0}{\second}. ¿Cuál es la aceleración angular conforme frenan las aspas?
    
    \begin{minipage}[t]{0.4\linewidth}
    Datos: 
    \begin{align*}
    f &= 6500 \, \text{rpm} = \SI{108.33}{\hertz} \\
    \omega_{f} &= 0 \\
    t &= \SI{4}{\second} \\
    \alpha &= \, ?
    \end{align*}
    \end{minipage}
    \hspace{1cm}
    \begin{minipage}[t]{0.4\linewidth}
    Expresiones:
    \begin{align*}
    \omega &= 2 \, \pi \, f \\
    \alpha &= \dfrac{\omega_{f} - \omega_{0}}{t}
    \end{align*}
    \end{minipage}

    Sustitución:
    \begin{align*}
    \omega &= 2 \, \pi (\SI{108.33}{\hertz}) \, \unit{\radian} = \SI{680.67}{\radian\per\second} \\[0.5em]
    \alpha &= \dfrac{ 0 - \SI{680.67}{\radian\per\second}}{\SI{4}{\second}} = - \SI[per-mode=fraction]{170.16}{\radian\per\square\second}
    \end{align*}

    \vspace{0.3cm}
    \begin{tasks}(4)
        \task $\displaystyle - \SI{43.44}{\radian\per\square\second}$
        \task $\displaystyle - \SI{93.2}{\radian\per\square\second}$
        \task \fbox{$\displaystyle - \SI{170.16}{\radian\per\square\second}$}
        \task $\displaystyle - \SI{258.62}{\radian\per\square\second}$
    \end{tasks}
    \question \label{Ejercicio_03} \textbf{Ejercicio de ejecución: } Una rueda de molino de \SI{0.35}{\meter} de diámetro gira a \num{2500} rpm. ¿Cuál es la aceleración centrípeta de un punto localizado en el borde de la rueda de molino?
    
    \begin{minipage}[t]{0.4\linewidth}
    Datos: 
    \begin{align*}
    d &= \SI{0.35}{\meter} \hspace{0.2cm} \Rightarrow \hspace{0.2cm} r = \SI{0.175}{\meter} \\
    f &= 2500 \, \text{rpm} = \SI{41.66}{\hertz} \\
    \omega &= \, ? \\
    v_{T} &= \, ? \\
    a_{C} &= \, ?
    \end{align*}
    \end{minipage}
    \hspace{1cm}
    \begin{minipage}[t]{0.4\linewidth}
    Expresiones:
    \begin{align*}
    \omega &= 2 \, \pi \, f \\
    v_{T} &= \omega \, r \\
    a_{C} &= \dfrac{\left( v_{T} \right)^{2}}{r}
    \end{align*}
    \end{minipage}

    Sustitución:
    \begin{align*}
    \omega_{f} &= 2 \, \pi (\SI{41.66}{\hertz}) \, \unit{\radian} = \SI[per-mode=fraction]{261.75}{\radian\per\second} \\[0.5em]
    v_{T} &= \left( \SI[per-mode=fraction]{261.75}{\radian\per\second} \right) \left( \SI{0.175}{\meter} \right) = \SI[per-mode=fraction]{45.80}{\meter\per\second} \\[0.5em]
    a_{C} &= \dfrac{\left( \displaystyle \SI[per-mode=fraction]{45.80}{\meter\per\second} \right)^{2}}{\SI{0.175}{\meter}} = \SI[per-mode=fraction]{11990.47}{\meter\per\square\second}
    \end{align*}

    \vspace{0.3cm}
    \begin{tasks}(4)
        \task $\displaystyle \SI[per-mode=fraction]{978.89}{\meter\per\square\second}$
        \task $\displaystyle \SI[per-mode=fraction]{10500.19}{\meter\per\square\second}$
        \task $\displaystyle \SI[per-mode=fraction]{11564.24}{\meter\per\square\second}$
        \task \fbox{$\displaystyle \SI[per-mode=fraction]{11990.47}{\meter\per\square\second}$}
    \end{tasks}


\end{questions}
\end{document}