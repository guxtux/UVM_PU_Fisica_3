\documentclass[12pt, letter]{exam}
\usepackage[utf8]{inputenc}
\usepackage[T1]{fontenc}
\usepackage[spanish]{babel}
\usepackage[autostyle,spanish=mexican]{csquotes}
\usepackage{amsmath}
\usepackage{amsthm}
\usepackage{physics}
\usepackage{tikz}
\usepackage{float}
\usepackage{siunitx}
\usepackage{multicol}
\usepackage{enumitem}
\usepackage[left=2.00cm, right=2.00cm, top=2.00cm, 
     bottom=2.00cm]{geometry}
\usepackage{pdfpages}

% \renewcommand{\questionlabel}{\thequestion}
\decimalpoint

\setlength{\belowdisplayskip}{-0.5pt}

\usepackage{tasks}
\settasks{
    label=\Alph*), 
    label-align=left,
    item-indent={20pt}, 
    column-sep={4pt},
    label-width={16pt},
}

\sisetup{per-mode=symbol}
\footer{}{\thepage}{}
\date{}

\title{Solución a los ejercicios de ejecución \\ Primer Examen ordinario de Física III}

\begin{document}
\maketitle

\setcounter{page}{2}

\begin{questions}
    \section{Sistema Internacional de Unidades y Conversión de unidades.}
    % \setcounter{question}{3} \question \textbf{Ejercicio de ejecución.} ¿Cuántos nanosegundos (\num{d-9}) hay un año de $365$ días?
    % \begin{align*}
    %     365 \, \text{días} \left( \dfrac{\SI{24}{\hour}}{1 \, \text{día}} \right) \left( \dfrac{\SI{60}{\minute}}{\SI{1}{\hour}} \right) \left( \dfrac{\SI{60}{\second}}{\SI{1}{\minute}} \right) \left( \dfrac{\SI{1}{\nano\second}}{\SI{d-9}{\second}} \right) = \SI{3.1536d16}{\nano\second}
    % \end{align*}
    % \begin{tasks}(4)
    %     \task \SI{3.1536d12}{\nano\second}
    %     \task \SI{3.1536d14}{\nano\second}
    %     \task \SI{3.1536d15}{\nano\second}
    %     \task \fbox{\SI{3.1536d16}{\nano\second}}
    % \end{tasks}
    \setcounter{question}{1} \question \textbf{Ejercicio de ejecución.} Si la velocidad del tren bala es de \SI{603}{\kilo\meter\per\hour}. ¿Qué distancia en metros ha recorrido en \SI{1}{\minute}?
    
    \vspace*{0.5cm}
    Hacemos primero la conversión a \unit{\meter\per\second} de la velocidad del tren:
    \begin{align*}
        \SI[per-mode=fraction]{603}{\kilo\meter\per\hour} \left( \dfrac{\SI{1000}{\meter}}{\SI{1}{\kilo\meter}} \right) \left( \dfrac{\SI{1}{\hour}}{\SI{3600}{\second}} \right) = \SI[per-mode=fraction]{167.5}{\meter\per\second}
    \end{align*}
    Luego calculamos la distancia que recorre en \SI{1}{\minute} = \SI{60}{\second}:
    \begin{align*}
        \SI[per-mode=fraction]{167.5}{\meter\per\second} \bigg( \SI{60}{\second} \bigg) = \SI{10050}{\meter}
    \end{align*}
    \begin{tasks}(4)
        \task \SI{10015}{\meter}
        \task \fbox{\SI{10050}{\meter}}
        \task \SI{10055}{\meter}
        \task \SI{11050}{\meter}
    \end{tasks}
    % \setcounter{question}{5} \question \textbf{Ejercicio de ejecución:} Un conductor fue detenido en la carretera, la policía le comenta que rebasó el límite de velocidad que es de \SI{100}{\kilo\meter\per\hour}, el conductor argumenta que manejaba a \break $1.25$ millas/minuto (Recuerda que 1 milla = \SI{1.609}{\kilo\meter}) ¿A qué velocidad en \unit{\kilo\meter\per\hour} iba conduciendo el auto?
    % \begin{align*}
    %     1.25 \, \dfrac{\text{milla}}{\unit{\minute}} \left( \dfrac{\SI{1.609}{\kilo\meter}}{1 \, \text{milla}} \right) \left( \dfrac{\SI{60}{\minute}}{\SI{1}{\hour}} \right) = \SI[per-mode=fraction]{120.67}{\kilo\meter\per\hour}
    % \end{align*}
    % \begin{tasks}(4)
    %     \task \fbox{\SI{120.67}{\kilo\meter\per\hour}}
    %     \task \SI{125.35}{\kilo\meter\per\hour}
    %     \task \SI{132.78}{\kilo\meter\per\hour}
    %     \task \SI{100.00}{\kilo\meter\per\hour}
    % \end{tasks}

    \section{Movimiento rectilíneo uniforme.}

    \setcounter{question}{4} \question  \label{Ejercicio_02} \textbf{Ejercicio de ejecución. } Se tienen $4$ objetos en movimiento con los siguientes datos de velocidad y tiempo: \\[0.3em]
    Objeto 1: $v = \SI{16}{\meter\per\second}, \, t = \SI{5}{\second}$ \\[0.3em]
    Objeto 2: $v = \SI{15}{\meter\per\second}, \, t = \SI{6}{\second}$ \\[0.3em]
    Objeto 3: $v = \SI{24}{\meter\per\second}, \, t = \SI{3}{\second}$ \\[0.3em]
    Objeto 4: $v = \SI{4}{\meter\per\second}, \, t = \SI{22}{\second}$ \\[0.3em]
    ¿Cuál de los objetos es el que recorrió la mayor distancia? 
    
    \vspace*{1cm}
    Para revolver este ejercicio, se calcula la distancia que recorre cada objeto:
    \begin{align*}
    v = \dfrac{d}{t} \hspace{0.5cm} \Rightarrow \hspace{0.5cm} d = v \, t
    \end{align*}
    \begin{table}[H]
        \centering
        \renewcommand{\arraystretch}{2}
        \begin{tabular}{c | c | c}
            Objeto & Sustitución & Distancia \\ \hline
            Objeto 1 & $d = \left( \displaystyle \SI[per-mode=fraction]{16}{\meter\per\second} \right) (\SI{5}{\second})$ & $\SI{80}{\meter}$ \\ \hline
            Objeto 2 & $d = \left( \displaystyle \SI[per-mode=fraction]{15}{\meter\per\second} \right) (\SI{6}{\second})$ & $\SI{90}{\meter}$ \\ \hline
            Objeto 3 & $d = \left( \displaystyle \SI[per-mode=fraction]{24}{\meter\per\second} \right) (\SI{3}{\second})$ & $\SI{72}{\meter}$ \\ \hline
            Objeto 4 & $d = \left( \displaystyle \SI[per-mode=fraction]{4}{\meter\per\second} \right) (\SI{22}{\second})$ & $\SI{88}{\meter}$ \\ \hline
        \end{tabular}
    \end{table}
    Por lo que el Objeto 2 es el que recorre una distancia mayor.
    \begin{tasks}(4)
        \task Objeto 1
        \task \fbox{Objeto 2}
        \task Objeto 3
        \task Objeto 4        
    \end{tasks}
    % \setcounter{question}{6} \question \label{Ejercicio_03} \textbf{Ejercicio de ejecución. } Calcula el tiempo en segundos que tardará un automóvil en desplazarse \SI{3}{\kilo\meter} en línea recta hacia el sur con una velocidad cuya magnitud es de \SI{70}{\kilo\meter\per\hour}.

    % \vspace*{0.25cm}
    % Convertimos la velocidad del automóvil de \unit{\kilo\meter\per\hour} a \unit{\meter\per\second}:
    % \begin{align*}
    %     \SI[per-mode=fraction]{70}{\kilo\meter\per\hour} \left( \dfrac{\SI{1000}{\meter}}{\SI{1}{\kilo\meter}} \right) \left( \dfrac{\SI{1}{\hour}}{\SI{3600}{\second}} \right) = \SI[per-mode=fraction]{19.44}{\meter\per\second}
    % \end{align*}
    % Ahora pasamos a resolver el ejercicio:

    % \begin{minipage}[t]{0.35\linewidth}
    % Datos: 
    % \begin{align*}
    % v &= \SI{19.44}{\meter\per\second} \\
    % d &= \SI{3}{\kilo\meter} = \SI{3000}{\meter} \\
    % t &= \, ?
    % \end{align*}
    % \end{minipage}
    % \hspace{1cm}
    % \begin{minipage}[t]{0.4\linewidth}
    % Expresión:
    % \begin{align*}
    % v &= \dfrac{d}{t} \hspace{0.5cm} \Rightarrow \hspace{0.5cm} t = \dfrac{d}{v} \\[0.5em]
    % \end{align*}
    % \end{minipage}

    % Sustitución:
    % \begin{align*}
    % t &= \dfrac{\SI{3000}{\meter}}{\displaystyle \SI[per-mode=fraction]{19.44}{\meter\per\second}} = \SI{154.28}{\second}
    % \end{align*}
    % \begin{tasks}(4)
    %     \task \SI{23.33}{\second}
    %     \task \SI{72.24}{\second}
    %     \task \fbox{\SI{154.28}{\second}}
    %     \task \SI{175.15}{\second}
    % \end{tasks}

    % \newpage

    \section{Movimiento uniformemente acelerado.}

    \setcounter{question}{7} \question \label{Ejercicio_04} \textbf{Ejercicio de ejecución. } Una persona viaja en motocicleta a una velocidad de \SI{3}{\meter\per\second} y acelera constantemente a razón de \SI{0.4}{\meter\per\square\second}. ¿Qué distancia recorrerá después de 1 minuto?
    
    \begin{minipage}[t]{0.35\linewidth}
    Datos: 
    \begin{align*}
    v_{i} &= \SI{3}{\meter\per\second} \\
    a &= \SI{0.4}{\meter\per\square\second} \\
    t &= \SI{1}{\minute} = \SI{60}{\second} \\
    d &= \, ?
    \end{align*}
    \end{minipage}
    \hspace{1cm}
    \begin{minipage}[t]{0.4\linewidth}
    Expresión:
    \begin{align*}
    d = v_{i} \, t + \dfrac{1}{2} \, a \, t^{2}
    \end{align*}
    \end{minipage}

    Sustitución:
    \begin{align*}
    d &= \left( \SI[per-mode=fraction]{3}{\meter\per\second} \right) \left( \SI{60}{\second} \right) + \dfrac{1}{2} \, \left( \SI[per-mode=fraction]{0.4}{\meter\per\square\second} \right) \left( \SI{60}{\second} \right)^{2} \\[0.5em]
    d &= \SI{180}{\meter} + \dfrac{1}{2} \, \left( \SI[per-mode=fraction]{0.4}{\meter\per\square\second} \right) \left( \SI{3600}{\square\second} \right) \\[0.5em]
    d &= \SI{180}{\meter} + \dfrac{\SI{1440}{\meter}}{2} \\[0.5em]
    d &= \SI{180}{\meter} + \SI{720}{\meter} = \SI{900}{\meter}
    \end{align*}
    \begin{tasks}(4)
        \task \fbox{\SI{900}{\meter}}
        \task \SI{1000}{\meter}
        \task \SI{1500}{\meter}
        \task \SI{2000}{\meter}
    \end{tasks}

    \newpage

    \section{Caída libre y tiro vertical.}

    \setcounter{question}{8} \question \label{Ejercicio_05} \textbf{Ejercicio de ejecución. } En su visita a la Torre Latinoamericana, Tobías dejó caer un muñeco desde el piso \num{44} y tardó \SI{5.81}{\second} en caer al piso (suponiendo que no hubo algún obstáculo durante la caída). Calcula la altura del piso \num{44}.

    \begin{minipage}[t]{0.35\linewidth}
    Datos: 
    \begin{align*}
    v_{i} &= 0 \\
    t &= \SI{5.81}{\second} \\
    g &= \SI{9.81}{\meter\per\square\second} \\
    y &= \, ?
    \end{align*}
    \end{minipage}
    \hspace{1cm}
    \begin{minipage}[t]{0.4\linewidth}
    Expresión:
    \begin{align*}
    y = v_{i} \, t + \dfrac{1}{2} g \, t^{2}
    \end{align*}
    \end{minipage}

    Sustitución:
    \begin{align*}
    y &= \left( \displaystyle \SI[per-mode=fraction]{0}{\meter\per\second} \right) (\SI{5.81}{\second}) + \dfrac{1}{2} \, \left( \displaystyle -\SI[per-mode=fraction]{9.81}{\meter\per\square\second} \right) \left(\SI{5.81}{\second} \right)^{2} \\[0.5em]
    y &= \dfrac{1}{2} \, \left( \displaystyle -\SI[per-mode=fraction]{9.81}{\meter\per\square\second} \right) \left(\SI{33.75}{\square\second} \right) \\[0.5em]
    y &= - \dfrac{\SI{331.14}{\meter}}{2} = - \SI{165.57}{\meter}
    \end{align*}
    Nota: el signo menos indica que la distancia que recorrió el objeto parte del punto inicial en dirección negativa del eje $y$.
    \begin{tasks}(4)
        \task \SI{160.21}{\meter}
        \task \fbox{\SI{165.57}{\meter}}
        \task \SI{307.18}{\meter}
        \task \SI{331.14}{\meter}
    \end{tasks}
    \setcounter{question}{9} \question \label{Ejercicio_06} \textbf{Ejercicio de ejecución. } Se lanza verticalmente hacia arriba una esfera metálica con una velocidad cuya magnitud es de \SI{20}{\meter\per\second}. ¿Qué distancia recorre a los \num{2} segundos?

    \begin{minipage}[t]{0.35\linewidth}
    Datos: 
    \begin{align*}
    v_{i} &= \SI{20}{\meter\per\second} \\
    t &= \SI{2}{\second} \\
    g &= \SI{9.81}{\meter\per\square\second} \\
    y &= \, ?
    \end{align*}
    \end{minipage}
    \hspace{1cm}
    \begin{minipage}[t]{0.4\linewidth}
    Expresión:
    \begin{align*}
    y = v_{i} \, t + \dfrac{1}{2} g \, t^{2}
    \end{align*}
    \end{minipage}

    Sustitución:
    \begin{align*}
    y &= \left( \displaystyle \SI[per-mode=fraction]{20}{\meter\per\second} \right) (\SI{2}{\second}) + \dfrac{1}{2} \, \left( \displaystyle -\SI[per-mode=fraction]{9.81}{\meter\per\square\second} \right) \left(\SI{2}{\second} \right)^{2} \\[0.5em]
    y &= \SI{40}{\meter} - \dfrac{1}{2} \, \left( \displaystyle \SI[per-mode=fraction]{9.81}{\meter\per\square\second} \right) \left(\SI{4}{\square\second} \right) \\[0.5em]
    y &= \SI{40}{\meter} - \dfrac{\SI{39.24}{\meter}}{2} \\[0.5em]
    y &= \SI{40}{\meter} - \SI{19.62}{\meter} = \SI{20.38}{\meter}
    \end{align*}
    \begin{tasks}(4)
        \task \SI{15.09}{\meter}
        \task \fbox{\SI{20.38}{\meter}}
        \task \SI{51.19}{\meter}
        \task \SI{59.62}{\meter}
    \end{tasks}

    \section{Movimiento circular.}

    \setcounter{question}{11} \question \label{Ejercicio_07} \textbf{Ejercicio de ejecución. } La turbina de una planta hidroeléctrica gira a razón de 75 rpm. ¿Cuál es el desplazamiento angular en radianes de la punta de una de las hélices de la turbina durante un minuto?
    \begin{align*}
        1 \, \text{rpm} &= 2 \, \pi \unit{\radian} \\[0.5em]
        \Rightarrow \hspace{0.3cm} 75 \, \text{rpm} &= \, ? \, \unit{\radian} \\[0.5em]
        \Rightarrow \hspace{0.3cm} \dfrac{\left( 75 \, \text{rpm} \right) (2 \, \pi \unit{\radian})}{1 \, \text{rpm}} &= \SI{471.23}{\radian}
    \end{align*}
    \begin{tasks}(4)
        \task \SI{150.00}{\radian}
        \task \SI{235.61}{\radian}
        \task \fbox{\SI{471.23}{\radian}}
        \task \SI{785.39}{\radian}
    \end{tasks}
    % \setcounter{question}{15} \question \label{Ejercicio_08} \textbf{Ejercicio de ejecución. } Un objeto está atado a una cuerda y se mueve en un círculo horizontal de \SI{90}{\centi\meter} de radio. Sin considerar los efectos de la gravedad, el objeto tiene una frecuencia de \num{80} rpm. ¿Cuál es la velocidad lineal que experimenta?

    % \begin{minipage}[t]{0.35\linewidth}
    % Datos: 
    % \begin{align*}
    % r &= \SI{90}{\centi\meter} = \SI{0.9}{\meter} \\
    % f &= \num{80} \, \text{rpm} = \SI{80}{\hertz} \\
    % v_{L} &= \, ?
    % \end{align*}
    % \end{minipage}
    % \hspace{1cm}
    % \begin{minipage}[t]{0.4\linewidth}
    % Expresiones:
    % \begin{align*}
    % T &= \dfrac{1}{f} \\[0.5em]
    % v_{L} &= \dfrac{2 \, \pi \, r}{T}
    % \end{align*}
    % \end{minipage}

    % Sustitución:
    % \begin{align*}
    % T &= \dfrac{1}{\SI{80}{\hertz}} = \SI{0.0125}{\second} \\[0.5em]
    % v_{L} &= \dfrac{2 \, \pi \left( \SI{0.9}{\meter} \right)}{\SI{0.0125}{\second}} = \dfrac{\SI{5.65}{\meter}}{\SI{0.0125}{\second}} = \SI[per-mode=fraction]{452.38}{\meter\per\second}
    % \end{align*}
    % \begin{tasks}(4)
    %     \task \SI{7.53}{\meter\per\second}
    %     \task \SI{47.37}{\meter\per\second}
    %     \task \fbox{\SI{452.38}{\meter\per\second}}
    %     \task \SI{753.98}{\meter\per\second}
    % \end{tasks}
    \setcounter{question}{12} \question \label{Ejercicio_09} \textbf{Ejercicio de ejecución. } Una hélice gira inicialmente con una velocidad angular cuya magnitud es de \SI{10}{\radian\per\second} y recibe una aceleración constante cuya magnitud es de \SI{3}{\radian\per\square\second}. ¿Cuál será la magnitud de su velocidad angular después de \num{7} segundos?

    \begin{minipage}[t]{0.35\linewidth}
    Datos: 
    \begin{align*}
    \omega_{i} &= \SI{10}{\radian\per\second} \\
    \alpha &= \SI{3}{\radian\per\square\second} \\
    t &= \SI{7}{\second} \\
    \omega_{f} &= \, ?
    \end{align*}
    \end{minipage}
    \hspace{1cm}
    \begin{minipage}[t]{0.4\linewidth}
    Expresión:
    \begin{align*}
    \omega_{f} = \omega_{i} + \alpha \, t
    \end{align*}
    \end{minipage}

    Sustitución:
    \begin{align*}
    \omega_{f} = \SI[per-mode=fraction]{10}{\radian\per\second} + \left( \SI[per-mode=fraction]{3}{\radian\per\square\second} \right) \left( \SI{7}{\second} \right) \\[0.5em]
    \omega_{f} = \SI[per-mode=fraction]{10}{\radian\per\second} + \SI[per-mode=fraction]{21}{\radian\per\second} = \SI[per-mode=fraction]{31}{\radian\per\second}
    \end{align*}
    \begin{tasks}(4)
        \task \SI{30}{\radian\per\second}
        \task \fbox{\SI{31}{\radian\per\second}}
        \task \SI{35}{\radian\per\second}
        \task \SI{36}{\radian\per\second}
    \end{tasks}

    \section{Leyes de Newton.}

    \setcounter{question}{15} \question \label{Ejercicio_10} \textbf{Ejercicio de ejecución. } Calcula la masa de una caja en kilogramos, si al recibir una fuerza cuya magnitud es de \SI{300}{\newton} le produce una aceleración con una magnitud de \SI{150}{\centi\meter\per\square\second}.
    
    \begin{minipage}[t]{0.35\linewidth}
    Datos: 
    \begin{align*}
    F &= \SI{300}{\newton} \\
    a &= \SI{150}{\centi\meter\per\square\second} \\
    m &= \, ?
    \end{align*}
    \end{minipage}
    \hspace{1cm}
    \begin{minipage}[t]{0.4\linewidth}
    Expresión:
    \begin{align*}
    F = m \, a \hspace{0.3cm} \Rightarrow \hspace{0.3cm} m = \dfrac{F}{a}
    \end{align*}
    \end{minipage}

    Se expresa la aceleración en \unit{\meter\per\second}:
    \begin{align*}
        a &= \SI[per-mode=fraction]{150}{\centi\meter\per\square\second} \left( \dfrac{\SI{1}{\meter}}{\SI{100}{\centi\meter}} \right) = \SI[per-mode=fraction]{1.5}{\meter\per\square\second}
    \end{align*}
    Sustitución:
    \begin{align*}
    m = \dfrac{\SI{300}{\newton}}{\displaystyle \SI[per-mode=fraction]{1.5}{\meter\per\square\second}} = \num{200} \, \dfrac{\displaystyle \unit[per-mode=fraction]{\kilo\gram\meter\per\square\second}}{\displaystyle \unit[per-mode=fraction]{\meter\per\square\second}} = \SI{200}{\kilo\gram}
    \end{align*}
        
    \begin{tasks}(4)
        \task \SI{2}{\kilo\gram}
        \task \SI{10}{\kilo\gram}
        \task \SI{100}{\kilo\gram}
        \task \fbox{\SI{200}{\kilo\gram}}
    \end{tasks}

    \section{Ley de gravitación universal.}

    \setcounter{question}{17} \question \label{Ejercicio_11} \textbf{Ejercicio de ejecución. } Una barra metálica cuyo peso tiene una magnitud de \SI{800}{\newton} se acerca a otra de \SI{1200}{\newton} hasta que la distancia entre sus centros es de \SI{80}{\centi\meter}. ¿Con qué magnitud de fuerza gravitacional se atraen?

    \begin{minipage}[t]{0.35\linewidth}
    Datos: 
    \begin{align*}
    W_{1} &= \SI{800}{\newton} \\
    W_{2} &= \SI{1200}{\newton} \\
    r &= \SI{80}{\centi\meter} = \SI{0.8}{\meter} \\
    g &= \SI{9.81}{\meter\per\square\second} \\
    G &= \SI{6.67d-11}{\newton\square\meter\per\square\kilo\gram} \\
    F &= \, ?
    \end{align*}
    \end{minipage}
    \hspace{1cm}
    \begin{minipage}[t]{0.4\linewidth}
    Expresión:
    \begin{align*}
    F = \dfrac{G \, m_{1} \, m_{2}}{r^{2}}
    \end{align*}
    \end{minipage}

    Convertimos los pesos de las barras de Newtons, a su masa en kilogramos:
    \begin{align*}
    W &= m \, g \hspace{0.3cm} \Rightarrow \hspace{0.3cm} m = \dfrac{W}{g} \\
    m_{1} &= \dfrac{\SI{800}{\newton}}{\displaystyle \SI[per-mode=fraction]{9.81}{\meter\per\square\second}} = \num{81.54} \dfrac{\displaystyle \unit[per-mode=fraction]{\kilo\gram\meter\per\square\second}}{\displaystyle \unit[per-mode=fraction]{\meter\per\square\second}} = \SI{81.54}{\kilo\gram} \\[1em]
    m_{2} &= \dfrac{\SI{1200}{\newton}}{\displaystyle \SI[per-mode=fraction]{9.81}{\meter\per\square\second}} = \num{122.32} \dfrac{\displaystyle \unit[per-mode=fraction]{\kilo\gram\meter\per\square\second}}{\displaystyle \unit[per-mode=fraction]{\meter\per\square\second}} = \SI{122.32}{\kilo\gram}
    \end{align*}

    Sustitución:
    \begin{align*}
    F &= \dfrac{\left( \displaystyle \SI[per-mode=fraction]{6.67d-11}{\newton\square\meter\per\square\kilo\gram} \right) \left( \SI{81.54}{\kilo\gram} \right) \left( \SI{122.32}{\kilo\gram} \right)}{\left( \SI{0.8}{\meter} \right)^{2}} \\[0.5em]
    F &= \dfrac{\SI{6.652d-7}{\newton\per\square\meter}}{\SI{0.64}{\square\meter}} = \SI{1.039d-6}{\newton}
    \end{align*}
    
    \begin{tasks}(4)
        \task \SI{1.039d-5}{\newton}
        \task \fbox{\SI{1.039d-6}{\newton}}
        \task \SI{8.317d-7}{\newton}
        \task \SI{1.000d-8}{\newton}
    \end{tasks}

    \setcounter{section}{11}

    \section{Calor, Trabajo y Energía.}

    \setcounter{question}{33} \question \label{Ejercicio_12} \textbf{Ejercicio de ejecución. } Calcula la magnitud de la velocidad de una caja cuya masa es de \SI{4}{\kilo\gram} y tiene una energía cinética de \SI{100}{\joule}.

    \begin{minipage}[t]{0.35\linewidth}
    Datos: 
    \begin{align*}
    m &= \SI{4}{\kilo\gram} \\
    E_{k} &= \SI{100}{\joule} \\
    v &= \, ?
    \end{align*}
    \end{minipage}
    \hspace{1cm}
    \begin{minipage}[t]{0.4\linewidth}
    Expresión:
    \begin{align*}
    E_{k} = \dfrac{1}{2} \, m \, v^{2} \hspace{0.3cm} \Rightarrow \hspace{0.3cm} v = \sqrt{\dfrac{2 \, E_{k}}{m}}
    \end{align*}
    \end{minipage}

    Sustitución:
    \begin{align*}
    v &= \sqrt{\dfrac{2 \left( \SI{100}{\joule} \right)}{\SI{4}{\kilo\gram}}} = \sqrt{\dfrac{\SI{200}{\newton\meter}}{\SI{4}{\kilo\gram}}} = \sqrt{50 \, \dfrac{\displaystyle \unit[per-mode=fraction]{\kilo\gram\square\meter\per\square\second}}{\unit{\kilo\gram}}} = \sqrt{\SI[per-mode=fraction]{50}{\square\meter\per\square\second}} \\[0.5em]
    v &= \SI[per-mode=fraction]{7.07}{\meter\per\second}
    \end{align*}
    \begin{tasks}(4)
        \task \SI{4.07}{\meter\per\second}
        \task \SI{5.07}{\meter\per\second}
        \task \SI{6.07}{\meter\per\second}
        \task \fbox{\SI{7.07}{\meter\per\second}}
    \end{tasks}

    \setcounter{question}{34} \question \label{Ejercicio_13} \textbf{Ejercicio de ejecución. } La novela de ciencia ficción \textit{Fahrenheit 451} del escritor Rad Bradbury, el título hace referencia a la temperatura en la que el papel de los libros se inflama. ¿Cuánto equivale esa temperatura $(451 ^{\circ}\text{F})$ en grados Celsius?

    Expresión: $\unit{\degreeCelsius} = \dfrac{5}{9} \left( ^{\circ}\text{F} - 32 \right)$

    Sustitución:
    \begin{align*}
        \unit{\degreeCelsius} = \dfrac{5}{9} \left( 451 - 32 \right) = \dfrac{5}{9} \left( 419\right) = \SI{232.77}{\degreeCelsius}
    \end{align*}
    \begin{tasks}(4)
        \task \SI{200.46}{\degreeCelsius}
        \task \fbox{\SI{232.77}{\degreeCelsius}}
        \task \SI{350.63}{\degreeCelsius}
        \task \SI{500.45}{\degreeCelsius}
    \end{tasks}
    % \setcounter{question}{43} \question \label{Ejercicio_14} \textbf{Ejercicio de ejecución. } A una temperatura de \SI{15}{\degreeCelsius} una varilla de hierro tiene una longitud de \SI{5}{\meter}. ¿Cuál será la longitud de la varilla al aumentar la temperatura a \SI{25}{\degreeCelsius}?

    % \begin{minipage}[t]{0.35\linewidth}
    % Datos: 
    % \begin{align*}
    % T_{i} &= \SI{15}{\degreeCelsius} \\
    % T_{f} &= \SI{25}{\degreeCelsius} \\
    % L_{i} &= \SI{5}{\meter} \\
    % \alpha_{Fe} &= \SI{11.7d-6}{\degreeCelsius}^{-1} \\
    % L_{f} &= \, ?
    % \end{align*}
    % \end{minipage}
    % \hspace{1cm}
    % \begin{minipage}[t]{0.4\linewidth}
    % Expresión:
    % \begin{align*}
    %     \Delta L &= \alpha \, L_{i} \, \Delta T \\[0.5em]
    %     L_{f} - L_{i} &= \alpha \, L_{i} \left( T_{f} - T_{i} \right) \\[0.5em] 
    %     L_{f} &= L_{i} \left[ 1 + \alpha \, \left( T_{f} - T_{i} \right) \right]
    % \end{align*}
    % \end{minipage}
    
    % Sustitución:
    % \begin{align*}
    % L_{f} &= \left( \SI{5}{\meter} \right) \left[ 1 + \left( \SI{11.7d-6}{\degreeCelsius}^{-1} \right) \left( \SI{25}{\degreeCelsius} - \SI{10}{\degreeCelsius} \right)\right] \\[0.5em]
    % L_{f} &= \left( \SI{5}{\meter} \right) \left[ 1 + \left( \SI{11.7d-6}{\degreeCelsius}^{-1} \right) \left( \SI{15}{\degreeCelsius}\right) \right] \\[0.5em]
    % L_{f} &= \left( \SI{5}{\meter} \right) \left[ 1 + \num{1.755d-4} \right] \\[0.5em]
    % L_{f} &= \left( \SI{5}{\meter} \right) \left( \num{1.0001755} \right) \\[0.5em]
    % L_{f} &= \SI{5.0008775}{\meter}
    % \end{align*}

    % \begin{tasks}(4)
    %     \task \SI{5.8775}{\meter}
    %     \task \SI{5.08775}{\meter}
    %     \task \SI{5.008775}{\meter}
    %     \task \fbox{\SI{5.0008775}{\meter}}
    % \end{tasks}

    \section{Máquinas y eficiencia.}

    \setcounter{question}{36} \question \label{Ejercicio_15} \textbf{Ejercicio de ejecución. } Una celda solar que está en desarrollo, transforma energía solar a energía eléctrica, si se sabe que se le suministran \SI{748}{\joule} y a la salida devuelve \SI{500}{\joule}, calcula la eficiencia en porcentaje de la celda solar.

    \begin{minipage}[t]{0.35\linewidth}
    Datos: 
    \begin{align*}
    W_{e} &= \SI{748}{\joule} \\
    W_{s} &= \SI{500}{\joule} \\
    \eta &= \, ?
    \end{align*}
    \end{minipage}
    \hspace{1cm}
    \begin{minipage}[t]{0.4\linewidth}
    Expresión:
    \begin{align*}
        \eta = \dfrac{W_{s}}{W_{e}} \times 100 \%
    \end{align*}
    \end{minipage}

    Sustitución:
    \begin{align*}
    \eta = \dfrac{\SI{500}{\joule}}{\SI{748}{\joule}} \times 100 \% = \num{66.84} \%
    \end{align*}
    \begin{tasks}(4)
        \task $\eta = 50.15 \%$
        \task \fbox{$\eta = 66.84 \%$}
        \task $\eta = 72.03 \%$
        \task $\eta = 100.00 \%$
    \end{tasks}

\end{questions}

\end{document}

