\documentclass[12pt, letter]{exam}
\usepackage[utf8]{inputenc}
\usepackage[T1]{fontenc}
\usepackage[spanish]{babel}
\usepackage[autostyle,spanish=mexican]{csquotes}
\usepackage{amsmath}
\usepackage{amsthm}
\usepackage{physics}
\usepackage{tikz}
\usepackage{float}
\usepackage{siunitx}
\usepackage{multicol}
\usepackage{enumitem}
\usepackage[left=2.00cm, right=2.00cm, top=2.00cm, 
     bottom=2.00cm]{geometry}
\usepackage{pdfpages}

% \renewcommand{\questionlabel}{\thequestion}
\decimalpoint

\setlength{\belowdisplayskip}{-0.5pt}

\usepackage{tasks}
\settasks{
    label=\Alph*), 
    label-align=left,
    item-indent={20pt}, 
    column-sep={4pt},
    label-width={16pt},
}

\sisetup{per-mode=symbol}
\footer{}{\thepage}{}

\begin{document}
\includepdf[pages={1}]{Caratula_Examen_Parcial_02_PU_Fisica_3_02.pdf}

\newpage

\begin{questions}
    \section{Movimiento uniformemente acelerado.}

    \question \label{Problema_01} \textbf{Problema de ejecución:} Un camión de pasajeros con una velocidad de \SI{20}{\kilo\meter\per\hour} se lanza cuesta abajo de una pendiente y adquiere una velocidad de \SI{70}{\kilo\meter\per\hour} en \num{1} minuto. Si se considera que su aceleración fue constante. Calcula la aceleración del camión en \unit{\meter\per\square\second}.
    \begin{tasks}(4)
        \task \SI{0.23}{\meter\per\square\second}
        \task \SI{50}{\meter\per\square\second}
        \task \SI{1}{\meter\per\square\second}
        \task \SI{0.83}{\meter\per\square\second}
    \end{tasks}
    \textbf{Solución:} Las velocidades hay que expresarlas en \unit{\meter\per\second}, por lo que:
    \begin{align*}
    \SI[per-mode=fraction]{20}{\kilo\meter\per\hour} \left( \dfrac{\SI{1000}{\meter}}{\SI{1}{\kilo\meter}} \right) \left( \dfrac{\SI{1}{\hour}}{\SI{3600}{\second}} \right) = \dfrac{\SI{20000}{\meter}}{\SI{3600}{\second}} = \SI[per-mode=fraction]{5.55}{\meter\per\second} \\[0.5em]
    \SI[per-mode=fraction]{70}{\kilo\meter\per\hour} \left( \dfrac{\SI{1000}{\meter}}{\SI{1}{\kilo\meter}} \right) \left( \dfrac{\SI{1}{\hour}}{\SI{3600}{\second}} \right) = \dfrac{\SI{70000}{\meter}}{\SI{3600}{\second}} = \SI[per-mode=fraction]{19.44}{\meter\per\second}
    \end{align*}
    \textbf{Datos:}
    \begin{align*}
    v_{i} &= \SI[per-mode=fraction]{5.55}{\meter\per\second} \\[0.5em]
    v_{f} &= \SI[per-mode=fraction]{19.44}{\meter\per\second} \\[0.5em]
    t &= \SI{1}{\minute} = \SI{60}{\second}
    \end{align*}
    \textbf{Expresión:}
    \begin{align*}
    a = \dfrac{v_{f} - v_{i}}{t}
    \end{align*}
    \textbf{Sustitución:}
    \begin{align*}
    a = \dfrac{\displaystyle \SI[per-mode=fraction]{19.44}{\meter\per\second} - \SI[per-mode=fraction]{5.55}{\meter\per\second}}{\SI{60}{\second}} = \dfrac{\displaystyle \SI[per-mode=fraction]{13.89}{\meter\per\second}}{\SI{60}{\second}} = \SI[per-mode=fraction]{0.231}{\meter\per\square\second}
    \end{align*}
    La respuesta correcta es el inciso A).
    \question Un tráiler parte del reposo $\mathbf{(v_{i} = 0)}$ y experimenta una aceleración cuya magnitud es de \SI{3}{\meter\per\square\second} $\mathbf{(a)}$ durante \num{5} minutos $\mathbf{(t)}$. Si queremos obtener la velocidad del tráiler $\mathbf{(v_{f})}$, ¿cuál será la expresión que nos resuelve el problema? \textbf{Nota: } Solo identifica la expresión, no se pide obtener el valor.
    \begin{tasks}(4)
        \task $d = v_{i} \, t + \dfrac{a \, t^{2}}{2}$
        \task $d = \left( \dfrac{v_{f} + v_{i}}{2} \right) \, t$
        \task $v_{f} = v_{i} + a \, t$
        \task $v_{f}^{2} = v_{i}^{2} + 2\, a \, d$
    \end{tasks}
    La expresión del inciso C) $\mathbf{v_{f} = v_{i} + a \, t}$ nos sirve, ya que el lado derecho de la expresión contiene las variables que nos indica el enunciado.
    \question Revisa el siguiente enunciado: \enquote{Un coche tiene una velocidad inicial de \SI{10}{\meter\per\second} $\mathbf{(v_{i})}$ y experimenta una aceleración cuya magnitud es de \SI{2.5}{\meter\per\square\second} $\mathbf{(a)}$, la cual dura \num{20} segundos $\mathbf{(t)}$}. Se pide calcular el desplazamiento del coche $\mathbf{(d)}$, ¿cuál de las siguientes expresiones nos resuelve el problema? \textbf{Nota: } Solo identifica la expresión, no se pide obtener el valor.
    \begin{tasks}(4)
        \task $v = \dfrac{d}{t}$
        \task $d = v_{i} \, t + \dfrac{a \, t^{2}}{2}$
        \task $d = \dfrac{v_{f}^{2} - v_{i}^{2}}{2 \, a}$
        \task $d = \left( \dfrac{v_{f} + v_{i}}{2} \right) \, t$
    \end{tasks}
    La expresión del inciso B) $\mathbf{d = v_{i} \, t + \dfrac{a \, t^{2}}{2}}$ es la que nos sirve, ya que las variables del lado derecho son las que se mencionan en el enunciado.

    \section{Caída libre.}
    
    \question Revisa el siguiente enunciado \enquote{Una canica se deja caer $\mathbf{(v_{i} = 0)}$ desde una ventana y tarda en llegar al suelo \num{6} segundos $\mathbf{(t)}$.}, ¿A qué altura se dejó caer la canica? $\mathbf{(y)}$. 
    \\
    ¿Cuál de las siguientes expresiones nos resuelve el problema? \textbf{Nota: } Solo identifica la expresión, no se pide obtener el valor.
    \begin{tasks}(4)
        \task $2 \, g \, y = v_{f}^{2} - v_{i}^{2}$
        \task $v_{f} = v_{i} + g \, t$
        \task $y = v_{i} \, t + \dfrac{1}{2} g \, t^{2}$
        \task $y = \dfrac{1}{2} \big( v_{i} + v_{f} \big) \, t$
    \end{tasks}
    La expresión del inciso C) $\mathbf{y = v_{i} \, t + \dfrac{1}{2} g \, t^{2}}$ es la que nos resuelve el problema, ya que en el lado derecho de la expresión se indican las variables que menciona el enunciado.
    \question \label{Problema_02} \textbf{Problema de ejecución:} Considerando que ya identificaste la expresión para el problema anterior de la canica, ¿cuál es la altura a la que se dejó caer?
    \begin{tasks}(4)
        \task \SI{122.62}{\meter}
        \task \SI{24.52}{\meter}
        \task \SI{58.86}{\meter}
        \task \SI{176.58}{\meter}
    \end{tasks}
    \textbf{Datos:}
    \begin{align*}
    v_{i} = \SI{0}{\meter\per\second} \hspace{1cm} t = \SI{6}{\second} \hspace{1cm} g = \SI{9.81}{\meter\per\square\second}
    \end{align*}
    \textbf{Expresión:}
    \begin{align*}
    y = v_{i} \, t + \dfrac{1}{2} g \, t^{2} \hspace{0.5cm} \Rightarrow \hspace{0.5cm} y = \dfrac{1}{2} g \, t^{2}
    \end{align*}
    \textbf{Sustitución:}
    \begin{align*}
    y &= \dfrac{1}{2} \left( \SI{9.81}{\meter\per\square\second} \right) \left( \SI{6}{\second} \right)^{2} =  \dfrac{1}{2} \left( \SI{9.81}{\meter\per\square\second} \right) \left( \SI{36}{\square\second} \right) = \dfrac{1}{2} \left( \SI{353.16}{\meter} \right) = \SI{176.58}{\meter}
    \end{align*}
    La respuesta correcta es el inciso D).
    \question Un objeto que experimenta una caída libre, se estudia considerando que el objeto está sujeto a una aceleración de tipo:
    \begin{tasks}(4)
        \task Ascendente.
        \task Descendente.
        \task \underline{\textbf{Constante}}.
        \task Nula.
    \end{tasks}

    \newpage

    \section{Tiro vertical.}

    \question Completa la siguiente frase: Cuando un objeto lanzado hacia arriba, alcanza la altura máxima cuando su velocidad es \rule{2cm}{0.3mm}
    \begin{tasks}(4)
        \task \underline{\textbf{Es cero}}.
        \task Mínima.
        \task Máxima.
        \task $\dfrac{v_{i}}{2}$
    \end{tasks}
    \question \label{Problema_03} \textbf{Problema de ejecución: }Se lanza verticalmente hacia arriba una esfera metálica con una velocidad cuya magnitud es de \SI{20}{\meter\per\second}. ¿Qué distancia máxima que recorre?
    \begin{tasks}(4)
        \task \SI{20.38}{\meter}
        \task \SI{40.77}{\meter}
        \task \SI{28.30}{\meter}
        \task \SI{10.19}{\meter}
    \end{tasks}
    \textbf{Datos:}
    \begin{align*}
    v_{i} = \SI{20}{\meter\per\second} \hspace{1cm} g = - \SI{9.81}{\meter\per\square\second}
    \end{align*}
    \textbf{Expresión:}
    \begin{align*}
    y_{\text{máx}} = - \dfrac{v_{i}^{2}}{2 \, g}
    \end{align*}
    \textbf{Sustitución:}
    \begin{align*}
    y_{\text{máx}} = - \dfrac{\displaystyle \left( \SI{20}{\meter\per\second} \right)^{2}}{\displaystyle (2) \left( - \SI{9.81}{\meter\per\square\second} \right)} = \dfrac{\displaystyle \SI{400}{\square\meter\per\square\second}}{\displaystyle \SI{19.62}{\meter\per\square\second}} = \SI{20.38}{\meter}
    \end{align*}
    El inciso A) es el correcto.
    \question Cuando un objeto se lanza verticalmente hacia arriba, se sabe que conforme se deplaza su aceleración \rule{2cm}{0.3mm}
    \begin{tasks}(4)
        \task Es cero.
        \task \underline{\textbf{Es constante}}.
        \task Aumenta.
        \task Disminuye.
    \end{tasks}
    \question El tiempo de vuelo de un objeto en tiro vertical se obtiene con la siguiente expresión:
    \begin{tasks}(4)
        \task $t_{\text{vuelo}} = - \dfrac{2 \, v_{f}^{2}}{g}$
        \task $t_{\text{vuelo}} = - \dfrac{2 \, v_{f}}{g}$
        \task $t_{\text{vuelo}} = - \dfrac{2 \, v_{i}^{2}}{g}$
        \task $t_{\text{vuelo}} = - \dfrac{2 \, v_{i}}{g}$
    \end{tasks}
    El inciso D) es el correcto, ya que el tiempo de vuelo es el doble del tiempo de subida del objeto a la altura máxima.
    

    \section{Fuerza y Leyes de Newton.}

    \question ¿Qué es una fuerza?
    \begin{tasks}
        \task Es la masa de un cuerpo.
        \task \underline{\textbf{Es la interacción entre dos cuerpos}}.
        \task Es la reacción a distancia entre dos cuerpos.
        \task Es la suma del peso y la masa.
    \end{tasks}
    \question Relaciona las columnas con las letras y los números.
    
    Son las leyes de Newton:
    
    \begin{minipage}[t]{0.4\linewidth}
    \begin{parts}
        \part Primera ley.
        \part Segunda ley.
        \part Tercera ley.
    \end{parts}
    \end{minipage}
    \hspace{-0.5cm}
    \begin{minipage}[t]{0.5\linewidth}
        \begin{enumerate}[label=\arabic*)]
            \itemsep0em
            \item De acción y reacción.
            \item De gravitación universal.
            \item De inercia.
            \item Relación entre masa y aceleración.
        \end{enumerate}
    \end{minipage}
    \begin{tasks}(4)
        \task a-1, \, b-2, \, c-3
        \task a-2, \, b-3, \, c-1
        \task a-3, \, b-4, \, c-2
        \task a-3, \, b-4, \, c-1
    \end{tasks}
    La respuesta correcta es el inciso D):
    \\
    Primera ley: Ley de la inercia. \\
    Segunda ley: Relación entre la masa y la aceleración. \\
    Tercera ley: De acción y reacción.
    
    \question De la segunda ley de Newton, la expresión que relaciona la fuerza con la aceleración es:
    \begin{tasks}(4)
        \task $F = m \, a$
        \task $F = \dfrac{m}{a}$
        \task $F = \dfrac{a}{m}$
        \task $F = m^{2} \, a$
    \end{tasks}
    La respuesta correcta es el inciso A) $F = m \, a$
    \question ¿Cuál es la masa en la Tierra de un objeto que pesa \SI{1000}{\newton} (Newtons)?
    \begin{tasks}(4)
        \task \SI{400.30}{\kilo\gram}
        \task \SI{253.87}{\kilo\gram}
        \task \SI{101.93}{\kilo\gram}
        \task \SI{100.00}{\kilo\gram}
    \end{tasks}
    \textbf{Datos:}
    \begin{align*}
    \text{Peso} = \SI{1000}{\newton} \hspace{1cm} g = \SI{9.81}{\meter\per\square\second}
    \end{align*}
    \textbf{Expresión:}
    \begin{align*}
    \text{Peso} = \text{Masa} \cdot g \hspace{1cm} \text{Masa} = \dfrac{\text{Peso}}{g}
    \end{align*}
    \textbf{Sustitución:}
    \begin{align*}
    \text{Masa} = \dfrac{\SI{1000}{\newton}}{\SI{9.81}{\meter\per\square\second}} = \SI{101.93}{\kilo\gram}
    \end{align*}
    Recordemos que el Newton se define como:
    \begin{align*}
    \SI{1}{\newton} = \SI[per-mode=fraction]{1}{\kilo\gram\meter\per\square\second} \hspace{0.5cm} \Rightarrow \hspace{0.5cm} \dfrac{\displaystyle \unit[per-mode=fraction]{\kilo\gram\meter\per\square\second}}{\displaystyle \unit[per-mode=fraction]{\meter\per\square\second}} = \dfrac{\unit{\kilo\gram\meter\square\second}}{\unit{\meter\square\second}} = \unit{\kilo\gram}
    \end{align*}
    El inciso C) es la respuesta correcta.
    \question Un objeto de \SI{500}{\kilo\gram} se coloca en la superficie de la Luna terrestre, si el valor de la aceleración debida a la gravedad en la Luna es de \SI{1.62}{\meter\per\square\second}, ¿Cuánto pesa en Newtons ese objeto en la superficie de la Luna?
    \begin{tasks}(4)
        \task \SI{-810}{\newton}
        \task \SI{1962}{\newton}
        \task \SI{810}{\newton}
        \task \SI{324}{\newton}
    \end{tasks}
    \textbf{Datos:}
    \begin{align*}
    m =  \SI{500}{\kilo\gram} \hspace{1cm} g_{\text{Luna}} = \SI[per-mode=fraction]{1.62}{\meter\per\square\second}
    \end{align*}
    \textbf{Expresión:}
    \begin{align*}
    \text{Peso} = m \cdot g
    \end{align*}
    \textbf{Sustitución:}
    \begin{align*}
    \textbf{Peso} = \left( \SI{500}{\kilo\gram} \right) \left( \SI[per-mode=fraction]{1.62}{\meter\per\square\second} \right) = \SI[per-mode=fraction]{810}{\kilo\gram\meter\per\square\second} = \SI{810}{\newton}
    \end{align*}
    El inciso C) es la respuesta correcta.
    \question Sabemos que la fuerza es una cantidad vectorial, del siguiente vector de fuerza con una magnitud de \SI{10}{\newton} y que tiene una dirección de \ang{60}, ¿cuánto vale la componente $F_{x}$ en la dirección del eje de las abscisas?
    \begin{tasks}(4)
        \task \SI{9.46}{\newton}
        \task \SI{15}{\newton}
        \task \SI{7.5}{\newton}
        \task \SI{5}{\newton}
    \end{tasks}
    \textbf{Datos:}
    \begin{align*}
    F = \SI{10}{\newton} \hspace{1cm} \theta = \ang{60}
    \end{align*}
    \textbf{Expresión:}
    \begin{align*}
    F_{x} = \cos \theta \cdot F
    \end{align*}
    \textbf{Sustitución:}
    \begin{align*}
    F_{x} = \left( \cos \ang{60} \right) \left( \SI{10}{\newton} \right) = \left( 0.5 \right) \left( \SI{10}{\newton} \right) = \SI{5}{\newton}
    \end{align*}
    El inciso D) es la respuesta correcta.
    \question Resolviendo en una mesa de fuerzas las componentes de un sistema de vectores, se tiene que las componentes del vector resultante son $R_{x} = \SI{10}{\newton}$ y $R_{y} = \SI{10}{\newton}$. ¿Cuál es el ángulo que forma $\va{R}$ con respecto al eje $x$.
    \begin{tasks}(4)
        \task \ang{51.34}
        \task \ang{45}
        \task \ang{90}
        \task \ang{60.38}
    \end{tasks}
    \textbf{Datos:}
    \begin{align*}
    R_{x} = \SI{10}{\newton} \hspace{1cm} R_{y} = \SI{10}{\newton}
    \end{align*}
    \textbf{Expresión:}
    \begin{align*}
    \theta_{R} = \tan^{-1} \left( \dfrac{R_{y}}{R_{x}} \right)
    \end{align*}
    \textbf{Sustitución:}
    \begin{align*}
        \theta_{R} = \tan^{-1} \left( \dfrac{\SI{10}{\newton}}{\SI{10}{\newton}} \right) = \tan^{-1} (1) = \ang{45}
    \end{align*}
    El inciso B) es la respuesta correcta.
    \question Supongamos que vamos de regreso a casa en un automóvil y presenta una falla, comenzamos a empujarlo al taller mecánico más cercano. Cuando el automóvil comienza a moverse: ¿Cómo es la fuerza que ejerces sobre el automóvil en comparación con la que éste ejerce sobre ti? 
    \begin{tasks}
        \task La fuerza que ejerce el automóvil es mayor en magnitud y en la misma dirección a la que el automóvil ejerce sobre ti.
        \task La fuerza que ejerce el automóvil es menor en magnitud y en dirección perpendicular a la que el automóvil ejerce sobre ti.
        \task \textbf{La fuerza que ejerce el automóvil es igual en magnitud y opuesta en dirección a la que el automóvil ejerce sobre ti}.
        \task La fuerza que ejerce el automóvil es igual en magnitud y en dirección oblicua a la que el automóvil ejerce sobre ti.
    \end{tasks}

    \section{Leyes de Kepler y el Sistema Solar.}
    
    \question \rule{2cm}{0.1mm} es el séptimo planeta del Sistema Solar considerando su posición a partir del Sol.
    \begin{tasks}(4)
        \task Neptuno.
        \task \underline{\textbf{Urano}}.
        \task Saturno.
        \task Júpiter.
    \end{tasks}
    \question De la primera ley de Kepler ¿cuál es el espacio geométrico que explica la trayectoria de un planeta?
    \begin{tasks}(4)
        \task Parábola.
        \task Círculo.
        \task Hipérbola.
        \task \underline{\textbf{Elipse}}.
    \end{tasks}
    
\end{questions}

\newpage

\textbf{\huge{Formulario.}}
\begin{table}[H]
    \centering
    \setlength{\tabcolsep}{40pt}
    \renewcommand{\arraystretch}{2.5}
    \begin{tabular}{c  c}
        \multicolumn{2}{c}{Movimiento Uniformemente Acelerado} \\
        $v_{f} = v_{i} + a \, t$ & $d = \dfrac{1}{2} \big( v_{i} + v_{f} \big) \, t$ \\
        $d = v_{i} \, t + \dfrac{1}{2} a \, t^{2}$ & $2 \, a \, d = v_{f}^{2} - v_{i}^{2}$ \\ \hline
        \multicolumn{2}{c}{Caída libre y Tiro vertical} \\
        $v_{f} = v_{i} + g \, t$ & $y = \dfrac{1}{2} \big( v_{i} + v_{f} \big) \, t$ \\
        $y = v_{i} \, t + \dfrac{1}{2} g \, t^{2}$ & $2 \, g \, y = v_{f}^{2} - v_{i}^{2}$ \\ 
        $y_{\text{máx}} = - \dfrac{v_{i}^{2}}{2 \, g}$ & $t_{\text{subida}} = - \dfrac{v_{i}}{g}$ \\ \hline
        \multicolumn{2}{c}{Vectores} \\
        $\abs{\va{R}} = \sqrt{R_{x}^{2} + R_{y}^{2}}$ & $\theta_{R} = \tan^{-1} \left( \dfrac{R_{y}}{R_{x}} \right)$ \\
        $R_{x} = \cos \theta \cdot \abs{\va{R}}$ & $R_{y} = \text{sen} \, \theta \cdot \abs{\va{R}}$ \\ \hline
        \multicolumn{2}{c}{Cantidades} \\
        $g = \SI{9.81}{\meter\per\square\second}$ & $\SI{1}{\kilo\meter} = \SI{1000}{\meter}$  \\
        $\SI{1}{\hour} = \SI{3600}{\second}$ &
\end{tabular}
\end{table}

\end{document}