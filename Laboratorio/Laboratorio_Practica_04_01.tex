\documentclass[14pt]{extarticle}
\usepackage[utf8]{inputenc}
\usepackage[T1]{fontenc}
\usepackage[spanish,es-lcroman]{babel}
\usepackage{amsmath}
\usepackage{amsthm}
\usepackage{physics}
\usepackage{tikz}
\usetikzlibrary{positioning}
\usepackage{calc}
\usepackage{float}
\usepackage[autostyle,spanish=mexican]{csquotes}
\usepackage[per-mode=symbol]{siunitx}
\usepackage{gensymb}
\usepackage{multicol}
\usepackage{enumitem}
\usepackage[left=2.00cm, right=2.00cm, top=2.00cm, 
     bottom=2.00cm]{geometry}

%\renewcommand{\questionlabel}{\thequestion)}
\decimalpoint
\sisetup{bracket-numbers = false}

\newlength{\depthofsumsign}
\setlength{\depthofsumsign}{\depthof{$\sum$}}
\newcommand{\nsum}[1][1.4]{% only for \displaystyle
    \mathop{%
        \raisebox
            {-#1\depthofsumsign+1\depthofsumsign}
            {\scalebox
                {#1}
                {$\displaystyle\sum$}%
            }
    }
}

\title{\vspace*{-2cm} Práctica 4 - Plano inclinado \\  Física III\vspace{-5ex}}
\date{}

\begin{document}
\maketitle


\section{Datos para la práctica.}

\begin{itemize}
\itemsep0em 
\item  \textbf{Práctica:} 4.
\item \textbf{Unidad:} Uno
\item \textbf{Temática:} Movimiento uniformemente acelerado.
\item \textbf{Nombre de la práctica:} Plano inclinado.
\item \textbf{Número de sesiones que se requieren para la práctica:} Dos.
\end{itemize}

\section{Objetivo general.}

Determinar la relación entre la distancia recorrida y el tiempo que tarda en desplazarse un balín sobre un plano inclinado.

\section{Objetivos específicos.}

Obtener la velocidad de un balín que se desliza sobre un plano inclinado a partir del registro de la distancia recorrida y el tiempo que requiere para ello.

\section{Hipótesis.}

El plano inclinado con el ángulo mayor permite que la velocidad del balín sea mayor en comparación con la inclinación de los otros ángulos.

\section{Planteamiento del problema.}

El plano inclinado es una situación clásica en física que se utiliza para analizar el movimiento de objetos bajo la influencia de la gravedad y la presencia de una superficie inclinada. Se puede describir de manera general de la siguiente manera:

Descripción del Sistema:

\begin{enumerate}
\item Objeto en Reposo o en Movimiento: se considera un objeto que puede estar en reposo sobre el plano inclinado o que se mueve a lo largo de él.
\item Plano Inclinado: El plano inclinado es una superficie plana que forma un ángulo $\theta$ con la horizontal. Puede ser liso o rugoso, y se asume que no hay fricción inicialmente.
\item Ejes de Coordenadas: Establecer un sistema de coordenadas cartesianas con ejes $x$ e $y$. Generalmente, el eje $x$ se alinea con la superficie inclinada y el eje $y$ es perpendicular a ella.
\end{enumerate}

\section{Marco teórico.}

Responde las siguientes preguntas apoyándote con la consulta de libros que deberás de incluir en el apartad de Bibliografía.

\begin{enumerate}
\item ¿Qué es el movimiento uniformemente acelerado (MUA)?
\item ¿Qué científico utilizó el plano inclinado para observar el MUA?
\item En términos de la física que ya conoces, explica el por qué se mueve un balín sobre un riel inclinado.
\end{enumerate}

\section{Material.}

\begin{enumerate}
\item Riel de \SI{150}{\centi\meter}
\item Plano inclinado.
\item Balín o canica.
\item Cronómetro (puede ser el del celular)
\end{enumerate}

\section{Montaje.}

\begin{enumerate}
\item Se coloca la base del plano inclinado a un ángulo de \ang{10}, sujetando el tornillo mariposa de tal manera que la base quede fija.
\item Colocamos el riel sobre el plano inclinado.
\item Se va a registrar el tiempo que tarda en deslizarse una distancia de \SI{10}{\centi\meter}, para ello, un integante del equipo debe de soltar el balín en el extremo inicial del riel, otro integrante deberá de colocar un bolígrafo o un lápiz en la marca de los \SI{10}{\centi\meter}. Cuando se suelte el balín, deberá de avisarse para que el cronómetro se inicie y cuando el balín llegue a la marca de distancia, se deberá de avisar para deterner el cronómetro.
\item El tiempo que se registre en el cronómetro, es el que se deberá de anotar en la tabla de datos.
\item Para tener una medida uniforme en cada distancia, deberán de repetir 5 (cinco veces) el registro de tiempo.
\item Se avanza a la siguiente marca de distancia, a \SI{20}{\centi\meter} y se repiten las cinco mediciones de tiempo.
\item Se continua hasta completar una distancia de \SI{100}{\centi\meter}.
\item Cuando se haya completado la tabla, se cambia el ángulo de la mesa a \ang{15} y se repite todo el procedimiento, esto considera que hay que elaborar una tabla de distancias y tiempos nuevamente.
\item El registro considera también el cambio de ángulo a \ang{20} y \ang{25}, cada uno, con su respectiva tabla.
\end{enumerate}

\section{Resultados.}

\begin{table}[H]
\centering
Tabla para un ángulo de \ang{10}

\begin{tabular}{| c | >{\centering}p{1cm} | >{\centering}p{1cm} | >{\centering}p{1cm} | >{\centering}p{1cm} | >{\centering}p{1cm} | c |} \hline
 & \multicolumn{5}{c|}{Tiempos} & \\ \hline
Distancia & $t_{1}$ & $t_{2}$ & $t_{3}$ & $t_{4}$ & $t_{5}$ & $t$ Promedio \\ \hline
\SI{10}{\centi\meter} & & & & & & \\ \hline
\SI{20}{\centi\meter} & & & & & & \\ \hline
\SI{30}{\centi\meter} & & & & & & \\ \hline
\SI{40}{\centi\meter} & & & & & & \\ \hline
\SI{50}{\centi\meter} & & & & & & \\ \hline
\SI{60}{\centi\meter} & & & & & & \\ \hline
\SI{70}{\centi\meter} & & & & & & \\ \hline
\SI{80}{\centi\meter} & & & & & & \\ \hline
\SI{90}{\centi\meter} & & & & & & \\ \hline
\SI{100}{\centi\meter} & & & & & & \\ \hline
\end{tabular}
\end{table}

Se deben de presentar cuatro tablas, una para cada ángulo: \ang{10}. \ang{15}, \ang{20} y \ang{25}.

\section{Análisis de resultados.}

Se debe de elaborar una gráfica por cada tabla de distancia y tiempo promedio, de tal manera que en el eje vertical se tenga la distancia y en el eje horizontal, el tiempo.

Los puntos experimentales se deben de marcar, no hay que unirlos.

Con la regla deberás de trazar una recta que se acerque lo más posible a todos los puntos, será un trabajo que hay que realizar con cuidado, ya que habrá algunos puntos que queden por arriba y otros que queden por abajo.

Una vez trazada la recta, hay que calcular la pendiente de la misma, para ello, ocupamos la expresión que ya conocemos:
\begin{align*}
m = \dfrac{d_{f} - d_{i}}{t_{f} - t_{i}}
\end{align*}
donde $d_{f}$ representa un punto final de la recta, $d_{i}$ corresponde a un punto inicial, para los valores de tiempo, tendemos un tiempo final $t_{f}$ y un tiempo inicial $t_{i}$, respectivamente. El valor de $m$ quedará expresado en \unit{\meter\per\second}, ya que la distancia la hemos expresado en metros.

Los valores de velocidad para cada plano inclinado los presentamos en la siguiente tabla:
\begin{table}[H]
    \centering
    \begin{tabular}{| c | c |} \hline
        Ángulo & Velocidad [\unit{\meter\per\second}] \\ \hline
        \ang{10} & \\ \hline
        \ang{15} & \\ \hline
        \ang{20} & \\ \hline
        \ang{25} & \\ \hline        
    \end{tabular}
\end{table}

\section{Conclusiones.}

Deberás de responder las siguientes preguntas:
\begin{enumerate}
\item ¿Se logró el objetivo de la práctica?
\item ¿La hipótesis planteada se confirma?
\item ¿Para qué ángulo la velocidad del balín es mayor?
\end{enumerate}
Recuerda que debes de extender tus respuestas de manera clara, evitando respuestas breves que no permitan reflejar el trabajo que realizaste.

\end{document}