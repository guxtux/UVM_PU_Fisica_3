\documentclass[14pt]{extarticle}
\usepackage[utf8]{inputenc}
\usepackage[T1]{fontenc}
\usepackage[spanish,es-lcroman]{babel}
\usepackage{amsmath}
\usepackage{amsthm}
\usepackage{physics}
\usepackage{tikz}
\usepackage{float}
\usepackage[autostyle,spanish=mexican]{csquotes}
\usepackage[per-mode=symbol]{siunitx}
\usepackage{gensymb}
\usepackage{multicol}
\usepackage{enumitem}
\usepackage[left=2.00cm, right=2.00cm, top=2.00cm, 
     bottom=2.00cm]{geometry}

%\renewcommand{\questionlabel}{\thequestion)}
\decimalpoint
\sisetup{bracket-numbers = false}

\renewcommand*{\theenumi}{\thesection.\arabic{enumi}}
\renewcommand*{\theenumii}{\theenumi.\arabic{enumii}}

\title{\vspace*{-2cm} Reporte Práctica 6 - Ley de Ohm\\  Física III \vspace{-5ex}}
\date{}

\begin{document}
\maketitle

\section{Datos.}

Deberás de incluir los datos experimentales que se midieron.

\subsection{Primera vez.}

Valores de resistencias.
\begin{table}[H]
\centering
\begin{tabular}{| c | c | } \hline
Resistencia & R $(\Omega)$ \\ \hline
$R_{1}$ & \\ \hline
$R_{2}$ & \\ \hline
$R_{3}$ & \\ \hline
\end{tabular}
\end{table}

Valor de voltaje directo.

Tipo de pila: \rule{2cm}{0.1mm} Voltaje: \rule{2cm}{0.1mm}

Valor de voltaje alterno.

Voltaje contacto: \rule{2cm}{0.1mm}

\subsection{Segunda vez.}

Valores de resistencias.
\begin{table}[H]
\centering
\begin{tabular}{| c | c | } \hline
Resistencia & R $(\Omega)$ \\ \hline
$R_{1}$ & \\ \hline
$R_{2}$ & \\ \hline
$R_{3}$ & \\ \hline
\end{tabular}
\end{table}

Valor de voltaje directo.

Tipo de pila: \rule{2cm}{0.1mm} Voltaje: \rule{2cm}{0.1mm}

\section{Análisis de datos.}

\subsection{Cálculo de corriente en las resistencias.}

Utilizando el valor de voltaje de la pila y con la ley de Ohm, calcula el valor de la corriente en cada resistencia, ocupa la siguiente tabla (en caso de valores muy pequeños de corriente, será conveniente que uses la notación científica para el valor de corriente):

\subsection*{Primera medición.}

Voltaje pila: \rule{2cm}{0.1mm}
\begin{table}[H]
\centering
\begin{tabular}{| c | c | c | } \hline
Resistencia & R $(\Omega)$ & I fórmula $(\si{\ampere})$ \\ \hline
$R_{1}$ & & \\ \hline
$R_{2}$ & & \\ \hline
$R_{3}$ & & \\ \hline
\end{tabular}
\end{table}

\subsection*{Segunda medición.}

Voltaje pila: \rule{2cm}{0.1mm}
\begin{table}[H]
\centering
\begin{tabular}{| c | c | c | } \hline
Resistencia & R $(\Omega)$ & I fórmula $(\si{\ampere})$ \\ \hline
$R_{1}$ & & \\ \hline
$R_{2}$ & & \\ \hline
$R_{3}$ & & \\ \hline
\end{tabular}
\end{table}

Del multímetro que ocupaste en el Laboratorio, revisa si el rango de medición para corriente directa, permitiría medir (o detectar) la corriente que tienes en cada tabla, tendrás que regresar al Laboratorio y pedir la caja para leer la información.

\section{Conclusiones.}

Responde las siguientes preguntas apoyándote con los apartados que desarrollaste para completar tu reporte:
\begin{enumerate}
\item ¿Se logró el objetivo de la práctica?
\item ¿Se corrobora la hipótesis planteada?
\item Si una pila nos dice que el valor nomimal es de \SI{1.5}{\volt}, ¿por qué tenemos un valor ya sea por debajo o por arriba de ese valor nomimal?
\item Lo que nos entrega la CFE son entre \SI{110}{\volt} - \SI{120}{\volt}, ¿por qué tenemos una variación en la lectura del multímetro?
\end{enumerate}

Recuerda que debes de actualizar en el material de tu reporte: el uso del protoboard y del adaptador para los dos pilas AA.

\end{document}