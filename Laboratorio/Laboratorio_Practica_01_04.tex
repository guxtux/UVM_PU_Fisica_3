\documentclass[14pt]{extarticle}
\usepackage[utf8]{inputenc}
\usepackage[T1]{fontenc}
\usepackage[spanish,es-lcroman]{babel}
\usepackage{amsmath}
\usepackage{amsthm}
\usepackage{physics}
\usepackage{tikz}
\usepackage{float}
\usepackage[autostyle,spanish=mexican]{csquotes}
\usepackage[per-mode=symbol]{siunitx}
\usepackage{gensymb}
\usepackage{multicol}
\usepackage{enumitem}
\usepackage[left=2.00cm, right=2.00cm, top=2.00cm, 
     bottom=2.00cm]{geometry}

%\renewcommand{\questionlabel}{\thequestion)}
\decimalpoint
\sisetup{bracket-numbers = false}

\title{\vspace*{-2cm} Práctica 1 - Análisis preliminar \\  Física III\vspace{-5ex}}
\date{}

\begin{document}
\maketitle

Luego de haber realizado el montaje experimental para la Ley de Hooke, deberás de preparar un análisis preliminar de los datos experimentales, con los siguientes puntos:

\begin{enumerate}
\item En la tabla de valores $F$ [\unit{\gram}] y $x$ [\unit{\centi\meter}] agrega una columna en donde vas a calcular el valor de $k = F/x$, las unidades quedarán en \unit{\gram\per\centi\meter}, quedando la tabla de la siguiente manera:
\begin{table}[H]
\centering
\begin{tabular}{c | c | c}
$F$ \, [\unit{\gram}] & $x$ \, [\unit{\centi\meter}] & $k$ \, [\unit{\gram\per\centi\meter}] \\ \hline
\vdots & \vdots & \vdots \\
\end{tabular}
\end{table}
\item Elabora una gráfica de $F$, la fuerza en gramos (eje y) contra $x$, el desplazamiento en centímetros (eje x), presenta cada dato con un símbolo, por ejemplo "+".
\item Traza una recta que \enquote{toque} la mayor cantidad de puntos experimentales.
\item Considera dos puntos que estén separados y muy cerca de la recta, obtén la pendiente $m$ de la recta que acabas de trazar.
\begin{align*}
m = \dfrac{\Delta \, y}{\Delta x} = \dfrac{y_{f} - y_{i}}{x_{f} - x_{i}}
\end{align*}
El valor de $m$ representa el valor de $k$ la constante del resorte, anota ese valor aquí:

$m = $ \rule{2cm}{0.3mm} [\unit{\gram\per\centi\meter}]
\item En una nueva tabla agrega el valor de la pendiente m de la recta, y para cada registro obtén el valor absoluto de la diferencia de $m - k$:
\begin{table}[H]
\centering
\begin{tabular}{c | c | c | c}
$F$ \, [\unit{\gram}] & $x$ \, [\unit{\centi\meter}] & $k$ \, [\unit{\gram\per\centi\meter}] & $\abs{m - k}$ \\ \hline
\vdots & \vdots & \vdots & \vdots \\
\end{tabular}
\end{table}
\item Señala en la tabla y en la gráfica el(los) valor(es) cuyo valor de $\abs{m - k}$ sea cero o muy cercano a cero.
\item Responde las siguientes preguntas:
\begin{enumerate}
\item ¿Observaste algo en particular en el resorte con la fuerza de mayor magnitud? ¿Tardó más tiempo en estabilizarse para hacer la medición?
\item ¿Por qué no todos los valores de la cuarta columna de la última tabla son cero?
\item ¿Cómo mejorarías el montaje experimental?
\item ¿Se cumplió el objetivo de la práctica 1?
\item ¿Las hipótesis de nuestra práctica son correctas? En caso de que no lo sean, explica el por qué.
\end{enumerate}
\item Una vez que hayas respondido todo lo anterior, deberás de anotarlo en tu archivo donde tienes el marco teórico que ya presentaste para la práctica 1.
\item Deberás de enviar tu archivo actualizado en la asignación por Teams.
\item En la siguiente clase se te pedirá que expongas tu análisis preliminar, por lo que deberás de preparar la información solicitada.
\item En caso de que elabores tu gráfica en tu cuaderno o en hoja milimétrica, toma un foto o escanea la imagen para que adjuntes esa imagen en la asignación.
\item Recuerda que esta actividad cuenta para la calificación.
\item El trabajo de análisis es INDIVIDUAL.
\end{enumerate}

\end{document}