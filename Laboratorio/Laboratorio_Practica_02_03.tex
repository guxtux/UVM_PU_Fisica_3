\documentclass[14pt]{extarticle}
\usepackage[utf8]{inputenc}
\usepackage[T1]{fontenc}
\usepackage[spanish,es-lcroman]{babel}
\usepackage{amsmath}
\usepackage{amsthm}
\usepackage{physics}
\usepackage{tikz}
\usepackage{float}
\usepackage[autostyle,spanish=mexican]{csquotes}
\usepackage[per-mode=symbol]{siunitx}
\usepackage{gensymb}
\usepackage{multicol}
\usepackage{multirow}
\usepackage{wasysym}
\usepackage{enumitem}
\usepackage[left=2.00cm, right=2.00cm, top=2.00cm, 
     bottom=2.00cm]{geometry}
\usepackage{hyperref}

%\renewcommand{\questionlabel}{\thequestion)}
\decimalpoint
\sisetup{bracket-numbers = false}

\title{\vspace*{-2cm} Práctica 2 - Mediciones en Física \vspace{-5ex}}
\date{}

\renewcommand*{\theenumi}{\thesection.\arabic{enumi}}
\renewcommand*{\theenumii}{\theenumi.\arabic{enumii}}
% \renewcommand*{\theenumi}{\thesubsection.\arabic{enumi}}
\setlist[enumerate]{font=\bfseries}

\begin{document}
\maketitle

\noindent
\textbf{Nombre:} \rule{8cm}{0.3mm} \textbf{Fecha:} \rule{4cm}{0.3mm} \\[0.5em]
\textbf{Grupo: } \rule{3cm}{0.3mm} \hspace{2cm} \textbf{Puntos/Firma:} \rule{5cm}{0.3mm}

\section{Lista de mediciones realizadas.}

Anota en la siguiente tabla los objetos y mediciones realizadas con su incertidumbre:
\begin{table}[H]
\centering
\begin{tabular}{| c | p{5cm} | p{5cm} |} \hline
Objeto & \multicolumn{1}{|c|}{Instrumento} & \multicolumn{1}{c|}{Magnitud ($m \pm \Delta m$)} \\ \hline
\multirow{6}{*}{Hoja carta} & & Largo: \\ \cline{3-3}
 & & Ancho:\\ \cline{2-3}
 & & Largo: \\ \cline{3-3}
 & & Ancho: \\ \cline{2-3}
 & & Largo: \\ \cline{3-3}
 & & Ancho:\\ \hline
\multirow{6}{*}{Roldana 1} & \multirow{2}{*}{Vernier} & \diameter \, externo: \\ \cline{3-3}
  & & \diameter \, interno: \\ \cline{2-3}
  & \multirow{2}{*}{Flexómetro}& \diameter \, externo: \\ \cline{3-3}
  & & \diameter \, interno: \\ \cline{2-3}
  & \multirow{2}{*}{Regla \SI{30}{\centi\meter}}& \diameter \, externo: \\ \cline{3-3}
  & & \diameter \, interno: \\ \hline
\multirow{6}{*}{Roldana 2} & \multirow{2}{*}{Vernier} & \diameter \, externo: \\ \cline{3-3}
& & \diameter \, interno: \\ \cline{2-3}
& \multirow{2}{*}{Flexómetro}& \diameter \, externo: \\ \cline{3-3}
& & \diameter \, interno: \\ \cline{2-3}
& \multirow{2}{*}{Regla \SI{30}{\centi\meter}}& \diameter \, externo: \\ \cline{3-3}
& & \diameter \, interno: \\ \hline
\multirow{2}{*}{Balín} & Vernier & Diámetro: \\ \cline{2-3}
 & Micrómetro & Diámetro: \\ \hline
\multirow{2}{*}{Moneda $10$ pesos} & & Diámetro: \\ \cline{3-3}
 & & Espesor: \\ \hline
\end{tabular}
\end{table}

\newpage

\begin{table}[H]
\centering
\begin{tabular}{| c | p{5cm} | p{5cm} |} \hline
Objeto & \multicolumn{1}{|c|}{Instrumento} & \multicolumn{1}{c|}{Magnitud ($m \pm \Delta m$)} \\ \hline
\multirow{3}{*}{Vaso precipitado} & & \diameter externo: \\ \cline{3-3}
 & & \diameter interno: \\ \cline{3-3}
 & & Profundidad: \\ \hline
 \multirow{3}{*}{Tubo ensayo} & & \diameter externo: \\ \cline{3-3}
 & & \diameter interno: \\ \cline{3-3}
 & & Profundidad: \\ \hline
Lata de aluminio & Micrómetro & Espesor: \\ \hline
\multirow{3}{*}{Tapa de PET} & & \diameter externo: \\ \cline{3-3}
 & & \diameter interno: \\ \cline{3-3}
 & & Profundidad: \\ \hline
\end{tabular}
\end{table}

\end{document}