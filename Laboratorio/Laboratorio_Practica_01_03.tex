\documentclass[14pt]{extarticle}
\usepackage[utf8]{inputenc}
\usepackage[T1]{fontenc}
\usepackage[spanish,es-lcroman]{babel}
\usepackage{amsmath}
\usepackage{amsthm}
\usepackage{physics}
\usepackage{tikz}
\usepackage{float}
\usepackage[autostyle,spanish=mexican]{csquotes}
\usepackage[per-mode=symbol]{siunitx}
\usepackage{gensymb}
\usepackage{multicol}
\usepackage{enumitem}
\usepackage[left=2.00cm, right=2.00cm, top=2.00cm, 
     bottom=2.00cm]{geometry}

%\renewcommand{\questionlabel}{\thequestion)}
\decimalpoint
\sisetup{bracket-numbers = false}

\title{\vspace*{-2cm} Actividad para la Práctica 1 \\  Física III\vspace{-5ex}}
\date{}

\begin{document}
\maketitle

\noindent
\textbf{Nombre:} \rule{8cm}{0.3mm}
\\[1em]
\textbf{Fecha:} \rule{4cm}{0.3mm}
\vspace{1cm}

Grafica los puntos experimentales de la masa y el volumen de un bloque de hierro. Obtén la expresión matemática entre las variables.

\begin{figure}[H]
    \centering
    \begin{tikzpicture}[scale=2]
        \draw (0, 0) -- (4.5, 0) node [above, pos=1.1] {\small{$V \, [\unit{\cubic\centi\meter}]$}};
        \draw (0, 0) -- (0, 7) node [left, pos=1.1] {\small{$m \, [\unit{\gram}]$}};

        \draw[step=0.5, black, thin] (0, 0) grid (4.5, 7);
        
        \foreach [evaluate={\j=int(\x*10)}] \x in {1, 2, 3, 4}
        {    
            \draw (\x, 0.2) -- (\x, -0.2);
            \node at (\x, -0.4) {\small{$\x$}};
        }
        \foreach [evaluate={\j=int(\x*5)}] \x in {1, 2, 3, 4, 5, 6, 7}
        {
            \draw (-0.2, \x) -- (0.2, \x);
            \node at (-0.6, \x) {\small{$\j$}};
        }
        
        % \draw [fill, color=blue](1, 1.6) circle (0.1cm); \pause
        % \draw [fill, color=blue](2, 3.2) circle (0.1cm); \pause
        % \draw [fill, color=blue](3, 4.8) circle (0.1cm); \pause
        % \draw [fill, color=blue](4, 6.4) circle (0.1cm); \pause
        % \draw [dashed, color=red] (0, 0) -- (4, 6.4);
    \end{tikzpicture}
\end{figure}

\end{document}