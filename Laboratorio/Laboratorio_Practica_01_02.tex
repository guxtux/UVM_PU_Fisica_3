\documentclass[14pt]{beamer}
\usepackage{./Estilos/BeamerUVM}
\usepackage{./Estilos/ColoresLatex}
\input{Preambulos/preambulo_Beamer_Madrid_default}
% \usefonttheme{serif}
\usepackage[clock]{ifsym}

\sisetup{per-mode=symbol}
\resetcounteronoverlays{saveenumi}

\title{\Large{Práctica 1 - Ley de Hooke} \\ \normalsize{Física III}}
\date{}

\renewcommand\cellset{\renewcommand\arraystretch{0.7}%
\setlength\extrarowheight{0pt}}

\addtobeamertemplate{frametitle}{}{%
\begin{tikzpicture}[remember picture,overlay]
\coordinate (logo) at ([xshift=-1.5cm,yshift=-0.8cm]current page.north east);
% \fill[devryblue] (logo) circle (.9cm);
% \clip (logo) circle (.75cm);
\node at (logo) {\includegraphics[width=2.1cm]{Imagenes/logo_UVM.png}};
\end{tikzpicture}}

\begin{document}
\maketitle

\section*{Contenido}
\frame{\frametitle{Contenido} \tableofcontents[currentsection, hideallsubsections]}

\section{Graficación}
\frame{\tableofcontents[currentsection, hideothersubsections]}
\subsection{Contexto inicial}

\begin{frame}
\frametitle{Relación entre variables}
En la ciencia cuando se estudian los fenómenos naturales (y artificiales), se comprueba que en ellos, hay dos (o más) magnitudes \textocolor{lava}{relacionadas} entre sí.
\end{frame}
\begin{frame}
\frametitle{Cambio entre las magnitudes}
Esto implica que al variar una de las magnitudes, la otra también cambia.
\end{frame}
\begin{frame}
\frametitle{Relación como magnitud}
Cuando las magnitudes están relacionadas, se dice que una es \textocolor{blue}{función} de la otra.
\end{frame}
\begin{frame}
\frametitle{Tipos de funciones}
Existen diferentes maneras en las cuales se relacionan las magnitudes físicas.
\end{frame}
\begin{frame}
\frametitle{Tipos de funciones}
Es decir que existen varios tipos de funciones que relacionan las magnitudes.
\end{frame}

\subsection{Proporción directa}

\begin{frame}
\frametitle{¿Qué es la proporción directa?}
Supongamos que tenemos dos magnitudes están relacionadas, \pause de modo que al duplicar el valor de una de ellas, \pause el valor de la otra también se duplica.
\end{frame}
\begin{frame}
\frametitle{¿Qué es la proporción directa?}
Al triplicar el valor de la primera, \pause el valor de la segunda queda multiplicada por tres.
\end{frame}
\begin{frame}
\frametitle{La proporción directa}
Siempre que sucede este cambio entre las magnitudes, decimos que entre ambas existe una \textocolor{denim}{proporción directa}.
\end{frame}
\begin{frame}
\frametitle{Datos experimentales}
En un experimento se registraron en la siguiente tabla, los valores de volumen ($V$) y de masa ($m$) de un bloque de hierro:
\end{frame}
\begin{frame}
\frametitle{Datos experimentales}
\begin{table}
    \renewcommand{\arraystretch}{1}
    \centering
    \begin{tabular}{c | c}
    Volumen [\unit{\cubic\centi\meter}] & Masa [\unit{\gram}] \\ \hline
    $V_{1} = 1$ & $m_{1} = 8$ \\ \hline
    $V_{2} = 2$ & $m_{2} = 16$ \\ \hline
    $V_{3} = 3$ & $m_{3} = 24$ \\ \hline
    $V_{4} = 4$ & $m_{4} = 32$ \\ \hline
    \end{tabular}
\end{table}
\end{frame}
\begin{frame}
\frametitle{Resultado preliminar}
La masa del bloque de hierro es directamente proporcional a su volumen.
\end{frame}
\begin{frame}
\frametitle{Expresión matemática}
Para denotar la relación entre el volumen y la masa del bloque de hierro, escribimos la relación como:
\pause
\begin{align*}
m \propto V
\end{align*}
\end{frame}
\begin{frame}
\frametitle{Aclaración importante}
La relación anterior \textocolor{carmine}{NO es una igualdad}, solo nos indica que existe una proporción directa.
\end{frame}

\subsection{Constante de proporcionalidad}

\begin{frame}
\frametitle{Obteniendo la constante de proporcionalidad}
Observan el cambio entre los valores de las masas y los volúmenes de la tabla, se tiene que:
\end{frame}
\begin{frame}
\frametitle{La constante de proporcionalidad}
\vspace*{-0.5cm}
\begin{eqnarray*}
\begin{aligned}
\dfrac{m_{1}}{V_{1}} &= \pause \dfrac{\SI{8}{\gram}}{\SI{1}{\cubic\centi\meter}} = \pause \SI{8}{\gram\per\cubic\centi\meter} \\[0.5em] \pause
\dfrac{m_{2}}{V_{2}} &= \pause \dfrac{\SI{16}{\gram}}{\SI{2}{\cubic\centi\meter}} = \pause \SI{8}{\gram\per\cubic\centi\meter} \\[0.5em] \pause
\dfrac{m_{3}}{V_{3}} &= \pause \dfrac{\SI{24}{\gram}}{\SI{3}{\cubic\centi\meter}} = \pause \SI{8}{\gram\per\cubic\centi\meter} \\[0.5em]
\vdots
\end{aligned}
\end{eqnarray*}
\end{frame}
\begin{frame}
\frametitle{Resultado del cociente}
Encontramos que al variar el volumen $V$ del bloque, su masa $m$ también cambia, \pause pero el \textocolor{vividcerise}{cociente} entre $m$ y $V$ permanece \textocolor{viridian}{constante}.
\end{frame}
\begin{frame}
\frametitle{Reescribiendo la expresión}
La expresión la podemos escribir como:
\pause
\begin{align*}
\dfrac{m}{V} = K
\end{align*}
donde $K$ es la llamada \textocolor{vividburgundy}{constante de proporcionalidad}.
\end{frame}
\begin{frame}
\frametitle{Primera conclusión}
Cuando dos magnitudes son directamente proporcionales, el cociente entre ellas permanece sin cambios, y recibe el nombre de constante de proporcionalidad entre las magnitudes.
\end{frame}

\section{Representación gráfica}
\frame{\tableofcontents[currentsection, hideothersubsections]}
\subsection{Método gráfico}

\begin{frame}
\frametitle{La representación gráfica}
Una manera alterna para analizar la dependencia entre dos magnitudes es mediante el \textocolor{tuscanred}{método gráfico}.
\end{frame}
\begin{frame}
\frametitle{Manejando los ejes del gráfico}
Para nuestro análisis en el eje de las ordenadas tendremos la magnitud de la masa, \pause mientras que para el eje de las abscisas ocuparemos el volumen.
\end{frame}
\begin{frame}
\frametitle{Base de la gráfica}
\begin{figure}
    \centering
    \begin{tikzpicture}[scale=0.7]
        \draw (0, 0) -- (4.5, 0) node [above, pos=1.3] {\small{$V \, [\unit{\cubic\centi\meter}]$}};
        \draw (0, 0) -- (0, 7) node [left, pos=1.1] {\small{$m \, [\unit{\gram}]$}};

        \draw[step=0.5, black, thin] (0, 0) grid (4.5, 7);
        
        \foreach [evaluate={\j=int(\x*10)}] \x in {1, 2, 3, 4}
        {    
            \draw (\x, 0.2) -- (\x, -0.2);
            \node at (\x, -0.4) {\small{$\x$}};
        }
        \foreach [evaluate={\j=int(\x*5)}] \x in {1, 2, 3, 4, 5, 6, 7}
        {
            \draw (-0.2, \x) -- (0.2, \x);
            \node at (-0.6, \x) {\small{$\j$}};
        }
        \pause
        % \draw [thick, color=ao] (0, 4) -- (7, 4); \pause

        % \node at (5, 3) {\small{No hay cambio de posición}}; \pause
        % \node at (5, 2.2) {\small{La pendiente de la recta vale $0$}};
        \draw [fill, color=blue](1, 1.6) circle (0.1cm); \pause
        \draw [fill, color=blue](2, 3.2) circle (0.1cm); \pause
        \draw [fill, color=blue](3, 4.8) circle (0.1cm); \pause
        \draw [fill, color=blue](4, 6.4) circle (0.1cm); \pause
        \draw [dashed, color=red] (0, 0) -- (4, 6.4);
    \end{tikzpicture}
\end{figure}
\end{frame}

\end{document}