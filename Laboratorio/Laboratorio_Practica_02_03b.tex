\documentclass[14pt]{extarticle}
\usepackage[utf8]{inputenc}
\usepackage[T1]{fontenc}
\usepackage[spanish,es-lcroman]{babel}
\usepackage{amsmath}
\usepackage{amsthm}
\usepackage{physics}
\usepackage{tikz}
\usepackage{float}
\usepackage[autostyle,spanish=mexican]{csquotes}
\usepackage[per-mode=symbol]{siunitx}
\usepackage{gensymb}
\usepackage{multicol}
\usepackage{multirow}
\usepackage{wasysym}
\usepackage{enumitem}
\usepackage[left=2.00cm, right=2.00cm, top=2.00cm, 
     bottom=2.00cm]{geometry}
\usepackage{hyperref}

%\renewcommand{\questionlabel}{\thequestion)}
\decimalpoint
\sisetup{bracket-numbers = false}

\title{\vspace*{-2cm} Práctica 2 - Mediciones en Física \\ Análisis preliminar\vspace{-5ex}}
\date{}

\renewcommand*{\theenumi}{\thesection.\arabic{enumi}}
\renewcommand*{\theenumii}{\theenumi.\arabic{enumii}}
% \renewcommand*{\theenumi}{\thesubsection.\arabic{enumi}}
\setlist[enumerate]{font=\bfseries}

\begin{document}
\maketitle

\section{Trabajo a realizar.}

Con las mediciones que completaste en la tercera clase de Laboratorio, deberás de presentar las siguientes tablas con las magnitudes y la incertidumbre de cada objeto medido, será necesario que te apoyes con las hojas que llevaste en cada clase.

La incertidumbre $\pm \Delta m$ que hay que asociar a cada instrumento es la mitad de la mínima escala, que registraste en una de las tablas. Cada medición debe de llevar su incertidumbre, si falta algún valor eso te va a descontar puntos en esta actividad.

\subsection{Hoja tamaño carta.}

Anota las magnitudes con su incertidumbre de la hoja tamaño carta en \unit{\centi\meter}.
\begin{table}[H]
\centering
\begin{tabular}{| c | p{3cm} | p{3cm} |} \hline
Instrumento & \multicolumn{1}{|c|}{Largo $\pm \Delta \text{Largo}$} & \multicolumn{1}{c|}{Ancho $\pm \Delta \text{Ancho}$}\\\hline
Regla \SI{30}{\centi\meter} & & \\ \hline
Regla \SI{1}{\meter} & & \\ \hline
Flexómetro & & \\ \hline
\end{tabular}
\end{table}

\subsection{Roldanas.}

Anota las magnitudes y su incertidumbre de las roldanas en \unit{\centi\meter}. Recuerda que abreviamos el diámetro por el símbolo \diameter.

\textbf{Roldana 1.}
\begin{table}[H]
\centering
\begin{tabular}{| c | c | c |} \hline
Instrumento & \diameter exterior $\pm \diameter \text{ext.}$ & \diameter interior $\pm \diameter \text{ext.}$\\ \hline
Regla \SI{30}{\centi\meter} & & \\ \hline
Flexómetro & & \\\hline
Vernier (en \unit{\centi\meter}) & & \\ \hline
\end{tabular}
\end{table}

\textbf{Roldana 2.}
\begin{table}[H]
\centering
\begin{tabular}{| c | c | c |} \hline
Instrumento & \diameter exterior $\pm \diameter \text{int.}$ & \diameter interior $\pm \diameter \text{int.}$\\ \hline
Regla \SI{30}{\centi\meter} & & \\ \hline
Flexómetro & & \\ \hline
Vernier(en \unit{\centi\meter}) & & \\ \hline
\end{tabular}
\end{table}

\subsection{Balín o canica.}

Indica el diámetro y su incertidumbre del balín o canica en \unit{\centi\meter}.
\begin{table}[H]
\centering
\begin{tabular}{| c | p{3cm} |} \hline
Instrumento & \multicolumn{1}{c|}{Diámetro $\pm \text{diámetro}$} \\\hline
Vernier (en \unit{\centi\meter}) & \\ \hline
Micrómetro (en \unit{\milli\meter}) & \\ \hline
\end{tabular}
\end{table}

\subsection{Objetos restantes.}

En la siguiente tabla anota el instrumento que utilizaste, así como el valor de las magnitudes y su incertidumbre.

\begin{table}[H]
\centering
\begin{tabular}{| c | p{5cm} | p{5cm} |} \hline
Objeto & \multicolumn{1}{|c|}{Instrumento} & \multicolumn{1}{c|}{Magnitud ($m \pm \Delta m$)} \\ \hline
\multirow{2}{*}{Moneda de $10$ pesos} & & \diameter externo: \\ \cline{3-3}
 & & Espesor \\ \hline
\multirow{3}{*}{Vaso precipitado} & & \diameter externo: \\ \cline{3-3}
    & & \diameter interno: \\ \cline{3-3}
    & & Profundidad: \\ \hline
    \multirow{3}{*}{Tubo ensayo} & & \diameter externo: \\ \cline{3-3}
    & & \diameter interno: \\ \cline{3-3}
    & & Profundidad: \\ \hline
Lata de aluminio & Micrómetro & Espesor: \\ \hline
\multirow{3}{*}{Tapa de PET} & & \diameter externo: \\ \cline{3-3}
    & & \diameter interno: \\ \cline{3-3}
    & & Profundidad: \\ \hline
\end{tabular}
\end{table}

\textbf{Se espera que las mediciones que reportas las hayas realizado de manera individual.}
\\[0.5em]
Responde las siguientes preguntas:
\begin{enumerate}
\item ¿Qué instrumento representó el manejo más sencillo?
\item ¿Qué instrumento representó el manejo más complicado?
\item ¿Qué instrumento tiene la mayor precisión?
\end{enumerate}

\section{Conclusiones.}

Para las conclusiones se te pide que las tablas para la hoja, las roldanas y el diámetro del balín, indiques cuál es la medida que se acerca más al valor real del objeto. Deberás de justificar tu respuesta apoyándote con todos los elementos que se trabajaron: naturaleza del instrumento, incertidumbre asociada, manejo del aparato, etc.

\section{Manejo y disposición de desechos.}

El sobrante de las latas se envió al bote de basura inorgánica reciclable. El papel para limpiar la lata también se envió al bote de basura para papel. 

\end{document}