\documentclass[14pt]{extarticle}
\usepackage[utf8]{inputenc}
\usepackage[T1]{fontenc}
\usepackage[spanish,es-lcroman]{babel}
\usepackage{amsmath}
\usepackage{amsthm}
\usepackage{physics}
\usepackage{tikz}
\usetikzlibrary{positioning}
\usepackage{calc}
\usepackage{float}
\usepackage[autostyle,spanish=mexican]{csquotes}
\usepackage[per-mode=symbol]{siunitx}
\usepackage{gensymb}
\usepackage{multicol}
\usepackage{enumitem}
\usepackage[left=2.00cm, right=2.00cm, top=2.00cm, 
     bottom=2.00cm]{geometry}

%\renewcommand{\questionlabel}{\thequestion)}
\decimalpoint
\sisetup{bracket-numbers = false}

\newlength{\depthofsumsign}
\setlength{\depthofsumsign}{\depthof{$\sum$}}
\newcommand{\nsum}[1][1.4]{% only for \displaystyle
    \mathop{%
        \raisebox
            {-#1\depthofsumsign+1\depthofsumsign}
            {\scalebox
                {#1}
                {$\displaystyle\sum$}%
            }
    }
}

\title{\vspace*{-2cm} Práctica 5 - Electrostática \\  Física III\vspace{-5ex}}
\date{}

\begin{document}
\maketitle

\section{Datos para la práctica.}

\begin{itemize}
\itemsep0em 
\item  \textbf{Práctica:} 5.
\item \textbf{Unidad:} Dos.
\item \textbf{Temática:} Electricidad.
\item \textbf{Nombre de la práctica:} Electrostática.
\item \textbf{Número de sesiones que se requieren para la práctica:} Dos.
\end{itemize}

\section{Objetivo general.}

Generar una carga eléctrica negativa mediante la fricción.

\section{Objetivos específicos.}

Entender el principio de funcionamiento y el manejo de:
\begin{enumerate}
\item El generador de Van de Graff.
\item La máquina de Wimshurst.
\end{enumerate}

\section{Hipótesis.}

Las cargas eléctricas que se generan por fricción son negativas (electrones).

\section{Planteamiento del problema.}

Todos los elementos de la naturaleza están compuestos de átomos y una de las partículas principales de todos los átomos son los electrones, los cuales se pueden desplazar de un átomo a otro, incluso entre materiales diferentes, formando \enquote{corrientes eléctricas} que recorren miles de kilómetros por segundo.

La unidad para medir la corriente eléctrica es el \textbf{Ampere}, que equivale aproximadamente a un flujo de \num{6.25d18} electrones cada segundo.

Todos los materiales conocidos, en mayor o menor grado, permiten el flujo de la corriente eléctrica a través de ellos, sin embargo, en todos los casos, también presentan una \enquote{resistencia} (o impedancia) al paso de dicha corriente. Mientras menos resistencia eléctrica presente un material, se considera un mejor conductor y mientras más resistencia presente será un mejor aislante.

Los mejores conductores de electricidad son los metales como el oro, la plata, el cobre o el aluminio y los mejores aislantes son el vidrio, la mica y algunos materiales sintéticos, por ejemplo, el PVC. Entre los dos extremos están todos los otros materiales que conocemos como semiconductores, su conductividad o resistencia puede variar dependiendo de diversos factores. Por ejemplo, el agua salada es mucho mejor conductor que el agua pura, la arcilla es mejor conductor que la arena o el concreto, la madera es mejor conductora cuando está verde que cuando está seca, y la piel humana es mejor conductora cuando está húmeda.

Hay muchos fenómenos físicos y químicos que incitan la formación de corrientes eléctricas. La forma más elemental de generar electricidad estática es frotando determinados materiales: por ejemplo, al frotar un peine de plástico con un paño, o nuestro cuerpo con ciertos vestidos o tapetes, o al rozar el viento seco y frío el automóvil en que viajamos. En cada caso, el peine, nuestro cuerpo o el automóvil se van cargando lentamente con electricidad estática, superando el \enquote{nivel normal} de la superficie terrestre o de los objetos circundantes.

Debido a que ningún átomo se puede quedar sin electrones ni soportar más de los que le corresponden, la corriente eléctrica siempre tiende a circular. Si no existe ninguna fuerza externa (voltaje) que impulse a los electrones o si éstos no tienen un camino para regresar y completar el circuito, la corriente eléctrica simplemente \enquote{no circula}. La única excepción al movimiento circular de la corriente la constituye la electricidad estática, que consiste en el desplazamiento o la acumulación de partículas (iones) de ciertos materiales que tienen la capacidad de almacenar una carga eléctrica positiva o negativa.

\section{Marco teórico.}

Responde las siguientes preguntas:
\begin{enumerate}
\item ¿Cuál es el principio de funcionamiento del generador de Van de Graff?
\item ¿Cuál es el principio de funcionamiento de la máquina de Wimshurst?
\item ¿Qué científico nombró a las cargas como positivas y negativas?
\item ¿Cómo identificarías si la carga eléctrica que se genera es positiva o negativa?
\item ¿En qué consisten los métodos de generación de carga por contacto, inducción y fricción? Menciona ejemplos de cada uno de ellos.
\end{enumerate}

\section{Material.}

\begin{enumerate}[label=\alph*)]
\item 1 generador de Van de Graff.
\item 1 máquina de Wimshurst.
\item Globos.
\end{enumerate}

\section{Montaje experimental.}

\begin{enumerate}
\item Atiende las indicaciones del Profesor para la operación y manejo del generador de Van de Graff y de la máquina de Wimshurst.
\item En tu cuaderno describe el fenómeno que se presenta cuando una persona toca la esfera del generador de Van de Graff.
\item Responde la pregunta: ¿cuál de los dos aparatos produce una carga eléctrica de mayor intensidad? ¿de qué manera apoyas tu respuesta?
\item Infla un globo en tu cabello, intenta colocarlo en la pared. Describe qué es lo que sucede, recuerda que tu enunciado debe de apoyarse con la física involucrada.
\end{enumerate}

\end{document}