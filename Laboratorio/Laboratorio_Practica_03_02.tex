\documentclass[14pt]{extarticle}
\usepackage[utf8]{inputenc}
\usepackage[T1]{fontenc}
\usepackage[spanish,es-lcroman]{babel}
\usepackage{amsmath}
\usepackage{amsthm}
\usepackage{physics}
\usepackage{tikz}
\usetikzlibrary{positioning}
\usepackage{calc}
\usepackage{float}
\usepackage[autostyle,spanish=mexican]{csquotes}
\usepackage[per-mode=symbol]{siunitx}
\usepackage{gensymb}
\usepackage{multicol}
\usepackage{enumitem}
\usepackage[left=2.00cm, right=2.00cm, top=2.00cm, 
     bottom=2.00cm]{geometry}

%\renewcommand{\questionlabel}{\thequestion)}
\decimalpoint
\sisetup{bracket-numbers = false}

\newlength{\depthofsumsign}
\setlength{\depthofsumsign}{\depthof{$\sum$}}
\newcommand{\nsum}[1][1.4]{% only for \displaystyle
    \mathop{%
        \raisebox
            {-#1\depthofsumsign+1\depthofsumsign}
            {\scalebox
                {#1}
                {$\displaystyle\sum$}%
            }
    }
}

\title{\vspace*{-2cm} Práctica 3 - Vectores \\ Actividad en clase - Física III\vspace{-5ex}}
\date{}

\begin{document}
\maketitle

\textbf{Nombre:} \rule{7cm}{0.1mm} \textbf{Fecha:} \rule{3cm}{0.1mm} 

\vspace*{0.4cm}
\textbf{Grupo:} \rule{3cm}{0.1mm}

\vspace{1cm}

\large
Esta actividad otorgará \textbf{5 puntos} siempre y cuando se hayan completado todas las actividades de la clase y solo contará en la calificación con la firma. \hspace{2cm} \textbf{Puntos/Firma Profesor:} \rule{4cm}{0.1mm}

\vspace*{0.5cm}
Realiza la descomposición de los siguientes vectores.

\begin{minipage}[T]{0.4\linewidth}
Vector $\va{A}$:
\begin{figure}[H]
    \centering
    \begin{tikzpicture}[scale=2.5]
        \draw [-stealth] (0, 0) -- (2, 0);
        \draw [-stealth] (0, 0) -- (0, 1.5);
        \draw [-stealth, thick] (0, 0) -- (1.63, 1.14) node [above, midway, rotate=35] {$\va{A} = \SI{2}{\newton}$};
        \draw [-stealth] (0.5, 0) arc(0:35:0.5);
        \node at (1.2, 0.2) {$\theta = \ang{35}$};
    \end{tikzpicture}
\end{figure}
\end{minipage}
\hspace{0.3cm}
\begin{minipage}[t]{0.4\linewidth}
Vector $\va{B}$:
\begin{figure}[H]
    \centering
    \begin{tikzpicture}[scale=1.5]
        \draw [-stealth] (-2, 0) -- (0.5, 0);
        \draw [-stealth] (0, 0) -- (0, 3.5);
        \draw [-stealth, thick] (0, 0) -- (-2, 3.46) node [above, midway, rotate=-60] {$\va{B} = \SI{4}{\newton}$};
        \draw [-stealth] (0.5, 0) arc(0:120:0.5);
        \node at (1.2, 0.2) {$\theta = \ang{120}$};
    \end{tikzpicture}
\end{figure}
\end{minipage}

\vspace*{0.3cm}
Revisando la mesa de fuerzas, responde las siguientes preguntas:
\begin{enumerate}
\item ¿Importa la orientación que dejes a la mesa de fuerza para obtener las componentes de un vector?
\item ¿Cuál es la mínima escala del transportador de la mesa de fuerzas?
\item Si el anillo que está al centro de la mesa de fuerzas no está centrado al poste, ¿qué nos dice del sistema de vectores que se está representando?
\end{enumerate}
\end{document}