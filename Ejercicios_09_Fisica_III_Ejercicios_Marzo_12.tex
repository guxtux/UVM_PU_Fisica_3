\documentclass[14pt]{extarticle}
\usepackage[utf8]{inputenc}
\usepackage[T1]{fontenc}
\usepackage[spanish,es-lcroman]{babel}
\usepackage{amsmath}
\usepackage{amsthm}
\usepackage{physics}
\usepackage{tikz}
\usepackage{float}
\usepackage[autostyle,spanish=mexican]{csquotes}
\usepackage[per-mode=symbol]{siunitx}
\usepackage{gensymb}
\usepackage{multicol}
\usepackage{enumitem}
\usepackage[left=2.00cm, right=2.00cm, top=2.00cm, 
     bottom=2.00cm]{geometry}

%\renewcommand{\questionlabel}{\thequestion)}
\decimalpoint
\sisetup{bracket-numbers = false}

\renewcommand*{\theenumi}{\thesection.\arabic{enumi}}
\renewcommand*{\theenumii}{\theenumi.\arabic{enumii}}

\title{\vspace*{-2cm} Actividad Grupo 43\\  Evaluación Continua - Física III\vspace{-5ex}}
\date{13 de marzo de 2024}

\begin{document}
\maketitle

\textbf{Nombre:} \rule{8cm}{0.1mm}

\vspace*{0.75cm}
\textbf{Importante: } Esta actividad otorgará hasta \textbf{5 puntos}, un punto por cada ejercicio. Deberás de entregar el desarrollo completo de cada ejercicio, si se reporta el resultado directo, sin presentar el desarrollo, sin manejo de unidades en cada paso, no se contará como ejercicio resuelto.Podrás apoyarte con tus notas.

\section{Ejercicios.}

\begin{enumerate}
\item Si la velocidad de una bicicleta cambia hasta alcanzar una velocidad de \SI{8}{\meter\per\second} en \SI{20}{\second} con una aceleración de \SI{0.3}{\meter\per\square\second}. Calcula su velocidad original.
\item Una bala es disparada hacia arriba con una velocidad de \SI{50}{\meter\per\second}. 

Calcula: a) su velocidad a los $4$ segundos de haber sido disparada, b) la altura máxima que alcanzará y c) en cuánto tiempo lo hará.
\item ¿Qué fuerza ejerce el motor de un automóvil de \SI{1300}{\kilo\gram} para detenerlo completamente después de \SI{7}{\second} si iba a una velocidad de \SI{65}{\kilo\meter\per\hour}?
\item ¿Qué fuerza necesita aplicar una grúa para subir $8$ niveles de \SI{2.5}{\meter} cada uno, si desarrolla un trabajo de \SI{1.25d5}{\joule}?
\item Completa la siguiente tabla, escribiendo en el espacio el valor de la temperatura correspondiente. Se deben de incluir las operaciones necesarias.
\begin{table}[H]
    \centering
    \begin{tabular}{| p{6.5cm} | >{\centering\arraybackslash}m{1.5cm} | >{\centering\arraybackslash}m{1.5cm} | >{\centering\arraybackslash}m{1.5cm} |} \hline
        \multicolumn{1}{|c|}{Temperatura} & \unit{\degreeCelsius} & ${}^{\circ} \, F$ & \unit{\kelvin} \\ \hline
        Temperatura de ebullición del oro & & $3129$ & \\ \hline
        Temperatura de ebullición del n-butanol & $117.4$ &  & \\ \hline
        Temperatura corporal del cuerpo humano & & $98.6$ & \\ \hline
        Temperatura de ebullición del agua en la ciudad de Puebla & & & $366$ \\ \hline
    \end{tabular}
\end{table}
\end{enumerate}
\end{document}