\documentclass[14pt]{extarticle}
\usepackage[utf8]{inputenc}
\usepackage[T1]{fontenc}
\usepackage[spanish,es-lcroman]{babel}
\usepackage{amsmath}
\usepackage{amsthm}
\usepackage{physics}
\usepackage{tikz}
\usepackage{float}
\usepackage[autostyle,spanish=mexican]{csquotes}
\usepackage[per-mode=symbol]{siunitx}
\usepackage{gensymb}
\usepackage{multicol}
\usepackage{enumitem}
\usepackage[left=2.00cm, right=2.00cm, top=2.00cm, 
     bottom=2.00cm]{geometry}

%\renewcommand{\questionlabel}{\thequestion)}
\decimalpoint
\sisetup{bracket-numbers = false}

\title{\vspace*{-2cm} Ejercicios de Caída Libre \\  Evaluación Continua - Física III\vspace{-5ex}}
\date{}

\begin{document}
\maketitle

\textbf{Nombre:} \rule{7cm}{0.1mm} \textbf{Fecha:} \rule{3cm}{0.1mm} 

\vspace*{0.4cm}
\textbf{Grupo:} \rule{3cm}{0.1mm} \hspace{1.5cm} \textbf{Firma Profesor:} \rule{4cm}{0.1mm}

\vspace{1cm}

Los siguientes ejercicios corresponden al Tema de caída libre, la evaluación consistirá en dos partes:
\begin{enumerate}[label=\alph*)]
\item En clase: se revisarán las expresiones que deberán de utilizarse para responder cada ejercicio. Esta actividad te otorgará \textbf{5 puntos} en Evaluación Continua, siempre y cuando la actividad ésta hoja esté firmada por el Profesor.
\item En casa: \textbf{al resolver en esta misma hoja} cada uno de los cinco ejercicios ya con las expresiones a utilizar con el procedimiento utilizado y las respuestas a las incógnitas, deberás de devolver \textbf{en la siguiente clase} la hoja. Esta actividad te otorgará otros \textbf{5 puntos} en Evaluación Continua.
\end{enumerate}

Los enunciados son:
\begin{enumerate}
\item Una niña deja caer una muñeca desde una ventana que está a \SI{60}{\meter} de altura sobre el suelo. ¿Qué tiempo tardará en caer? ¿Con qué magnitud de velocidad choca contra el suelo?
\item Una maceta cae desde la azotea de un edificio y tarda en llegar al suelo $4$ segundos. ¿Cuál es la altura del edificio? ¿Cuál es la magnitud de la velocidad con que la maceta llegará al suelo?
\item Se lanza verticalmente hacia abajo una pelota al vacío con una velocidad inicial de \SI{5}{\meter\per\second} ¿Qué magnitud de la velocidad llevará a los 3 segundos de su caída? ¿Qué distancia recorrerá entre los segundos $3$ y $4$?
\item Se tira una canica verticalmente hacia abajo con una velocidad inicial cuya magnitud es de \SI{8}{\meter\per\second}. ¿Qué magnitud de velocidad llevará a los $4$ segundos de su caída? ¿Qué distancia recorre en ese tiempo?
\item Un niño deja caer una piedra a un pozo de \SI{20}{\meter} de profundidad. Calcula con qué velocidad llegará al fondo si tarda $8$ segundo en hacerlo.
\end{enumerate}
\textbf{Los puntos de Evaluación Continua solo contarán en cada clase.}

\end{document}