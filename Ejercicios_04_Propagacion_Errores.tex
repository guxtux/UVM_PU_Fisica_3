\documentclass[14pt]{extarticle}
\usepackage[utf8]{inputenc}
\usepackage[T1]{fontenc}
\usepackage[spanish,es-lcroman]{babel}
\usepackage{amsmath}
\usepackage{amsthm}
\usepackage{physics}
\usepackage{tikz}
\usepackage{float}
\usepackage[autostyle,spanish=mexican]{csquotes}
\usepackage[per-mode=symbol]{siunitx}
\usepackage{gensymb}
\usepackage{multicol}
\usepackage{enumitem}
\usepackage[left=2.00cm, right=2.00cm, top=2.00cm, 
     bottom=2.00cm]{geometry}

%\renewcommand{\questionlabel}{\thequestion)}
\decimalpoint
\sisetup{bracket-numbers = false}

\title{\vspace*{-2cm} Ejercicios de Propagación de Errores \\  Evaluación Continua - Física III\vspace{-5ex}}
\date{}

\begin{document}
\maketitle

Resuelve los siguientes ejercicios ocupando el tema de propagación de errores o incertidumbres, deberás de realizar a detalle cada procedimiento.

Esta actividad te otorgará cuatro puntos.

\begin{enumerate}
\item (\textbf{1 punto.}) Se tienen tres varillas y cada una ha sido medida con regla de diferente graduación; las medidas de las longitudes son: $(\num{4.37} \pm\SI{0.005}{\meter})$, $(\num{138.5} \pm \SI{0.05}{\centi\meter})$, $(\unit{7.877} \pm \SI{0.0005}{\meter})$. Si las varillas se colocan una a continuación de otra ¿Cuál sería la longitud total y su incertidumbre que se tendría?
\item (\textbf{1 punto.}) Un trozo rectangular de aluminio mide $\num{5.10} \pm \SI{0.01}{\centi\meter}$ de longitud y $\num{1.90} \pm \SI{0.01}{\centi\meter}$ de anchura. Calcula su área y la incertidumbre del área.
\item (\textbf{2 puntos.}) Al comer una bolsa de galletas con chispas de chocolate, nos damos cuenta de que cada una es un disco circular con diámetro de $\num{8.50} \pm \SI{0.02}{\centi\meter}$ y espesor de $\num{0.050} \pm \SI{0.005}{\centi\meter}$.
\begin{enumerate}
\item Calcula el volumen promedio de una galleta y la incertidumbre del volumen.
\item Obtén la razón diámetro/espesor y la incertidumbre de dicha razón.
\end{enumerate}
\end{enumerate}

\end{document}