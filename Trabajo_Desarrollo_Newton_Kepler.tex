\documentclass[14pt]{beamer}
\usepackage{./Estilos/BeamerUVM}
\usepackage{./Estilos/ColoresLatex}
\input{Preambulos/preambulo_Beamer_Madrid_default}
% \usefonttheme{serif}
\usepackage[clock]{ifsym}
\usetikzlibrary{plotmarks}

\sisetup{per-mode=symbol}
\resetcounteronoverlays{saveenumi}

\title{\Large{Trabajo de Desarrollo} \\ \normalsize{Física III}}
\date{}

% Macro para agregar el logo de UVM en cada slide de la presentación

\addtobeamertemplate{frametitle}{}{%
\begin{tikzpicture}[remember picture,overlay]
\coordinate (logo) at ([xshift=-1.5cm,yshift=-0.8cm]current page.north east);
% \fill[devryblue] (logo) circle (.9cm);
% \clip (logo) circle (.75cm);
\node at (logo) {\includegraphics[width=2.1cm]{Imagenes/logo_UVM.png}};
\end{tikzpicture}}

\begin{document}
\maketitle
  
\section*{Contenido}
\frame{\frametitle{Contenido} \tableofcontents[currentsection, hideallsubsections]}

\section{Trabajo de desarrollo}
\frame{\tableofcontents[currentsection, hideothersubsections]}
\subsection{Indicaciones}

\begin{frame}
\frametitle{Trabajo de desarrollo}
Deberás de realizar un trabajo con los siguientes puntos:
\pause
\setbeamercolor{item projected}{bg=bananayellow,fg=ao}
\setbeamertemplate{enumerate items}{%
\usebeamercolor[bg]{item projected}%
\raisebox{1.5pt}{\colorbox{bg}{\color{fg}\footnotesize\insertenumlabel}}%
}
\begin{enumerate}[<+->]
\item Un revisión biográfica de Isaac Newton y de Johannes Kepler.
\seti
\end{enumerate}
\end{frame}
\begin{frame}
\frametitle{Trabajo de desarrollo}
\setbeamercolor{item projected}{bg=bananayellow,fg=ao}
\setbeamertemplate{enumerate items}{%
\usebeamercolor[bg]{item projected}%
\raisebox{1.5pt}{\colorbox{bg}{\color{fg}\footnotesize\insertenumlabel}}%
}
\begin{enumerate}[<+->]
\conti
\item Para Isaac Newton deberás de presentar de manera general, las aportaciones que hizo para la mecánica, en particular, las llamadas Leyes de Newton, respondiendo ¿de qué manera logró establecer o construir esas leyes?
\seti
\end{enumerate}
\end{frame}
\begin{frame}
\frametitle{Trabajo de desarrollo}
Así como en la matemática y en la óptica. ¿Qué aportaciones realizó?
\end{frame}
\begin{frame}
\frametitle{Trabajo de desarrollo}
\setbeamercolor{item projected}{bg=bananayellow,fg=ao}
\setbeamertemplate{enumerate items}{%
\usebeamercolor[bg]{item projected}%
\raisebox{1.5pt}{\colorbox{bg}{\color{fg}\footnotesize\insertenumlabel}}%
}
\begin{enumerate}[<+->]
\conti
\item Para Johannes Kepler, presentarás su trabajo conocido como las Leyes de Kepler, ¿en qué consisten? ¿qué fenómenos explican?
\seti
\end{enumerate}
\end{frame}

\subsection{Puntos que otorga}

\begin{frame}
\frametitle{Trabajo de desarrollo}
\setbeamercolor{item projected}{bg=bananayellow,fg=ao}
\setbeamertemplate{enumerate items}{%
\usebeamercolor[bg]{item projected}%
\raisebox{1.5pt}{\colorbox{bg}{\color{fg}\footnotesize\insertenumlabel}}%
}
\begin{enumerate}[<+->]
\conti
\item Esta actividad de desarrollo aportará hasta $7$ puntos de evaluación continua.
\item La evaluación se hará con una rúbrica, pero el nivel de calidad y exigencia del trabajo será mayor.
\seti
\end{enumerate}
\end{frame}
\begin{frame}
\frametitle{Importante consideración}
Todo trabajo se presentará como de elaboración propia, \pause en el caso donde se identifique que se hayan utilizado herramientas tipo chatGPT, y otras, el trabajo \textocolor{red}{será anulado}.
\end{frame}

\subsection{Fecha de entrega}

\begin{frame}
\frametitle{Trabajo de desarrollo}
\setbeamercolor{item projected}{bg=bananayellow,fg=ao}
\setbeamertemplate{enumerate items}{%
\usebeamercolor[bg]{item projected}%
\raisebox{1.5pt}{\colorbox{bg}{\color{fg}\footnotesize\insertenumlabel}}%
}
\begin{enumerate}[<+->]
\conti
\item La entrega del trabajo será por Teams, el plazo para el envío es el día 22 de noviembre a las 8 pm.
\seti
\end{enumerate}
\end{frame}
\begin{frame}
\frametitle{Trabajo de desarrollo}
\setbeamercolor{item projected}{bg=bananayellow,fg=ao}
\setbeamertemplate{enumerate items}{%
\usebeamercolor[bg]{item projected}%
\raisebox{1.5pt}{\colorbox{bg}{\color{fg}\footnotesize\insertenumlabel}}%
}
\begin{enumerate}[<+->]
\conti
\item No se recibirán trabajos extemporáneos, a menos que se presente la evidencia de problemas técnicos durante el envío y que se haya notificado oportunamente al Profesor.
\seti
\end{enumerate}
\end{frame}
\begin{frame}
\frametitle{Trabajo de desarrollo}
\setbeamercolor{item projected}{bg=bananayellow,fg=ao}
\setbeamertemplate{enumerate items}{%
\usebeamercolor[bg]{item projected}%
\raisebox{1.5pt}{\colorbox{bg}{\color{fg}\footnotesize\insertenumlabel}}%
}
\begin{enumerate}[<+->]
\conti
\item Esta actividad \textocolor{ao}{deberá de incluirse} en la Bitácora de Evaluación Continua.
\end{enumerate}
\end{frame}
\end{document}